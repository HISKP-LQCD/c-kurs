\documentclass{article}[12pt]
\usepackage[utf8]{inputenc}
\usepackage[T1]{fontenc}
\usepackage[ngerman]{babel}
\usepackage[dvipsnames]{xcolor}
\usepackage{lipsum}

\usepackage{amsfonts}
\usepackage[intlimits]{amsmath}
\usepackage{cite}
\usepackage{epsfig}

\usepackage[usenames,dvipsnames]{pstricks}
\usepackage{pstricks-add}
\usepackage{epsfig}
\usepackage{pst-grad} % For gradients
\usepackage{pst-plot} % For axes

\addtolength{\hoffset}{-1.5cm}
\addtolength{\textwidth}{3cm}
\usepackage{listings}
\usepackage{color}
\definecolor{mygreen}{rgb}{0,0.6,0}
\definecolor{mygray}{rgb}{0.5,0.5,0.5}
\definecolor{mymauve}{rgb}{0.58,0,0.82}
\PassOptionsToPackage{svgnames}{xcolor}
\usepackage{tcolorbox}
\usepackage{lipsum}
\tcbuselibrary{skins,breakable}
\usetikzlibrary{shadings,shadows}

\lstset{ %
  backgroundcolor=\color{white},   % choose the background color; you must add \usepackage{color} or \usepackage{xcolor}; should come as last argument
  basicstyle=\footnotesize,        % the size of the fonts that are used for the code
  breakatwhitespace=false,         % sets if automatic breaks should only happen at whitespace
  breaklines=true,                 % sets automatic line breaking
  captionpos=b,                    % sets the caption-position to bottom
  commentstyle=\color{mygreen},    % comment style
  deletekeywords={...},            % if you want to delete keywords from the given language
  escapeinside={\%*}{*)},          % if you want to add LaTeX within your code
  extendedchars=true,              % lets you use non-ASCII characters; for 8-bits encodings only, does not work with UTF-8
  frame=single,                    % adds a frame around the code
  keepspaces=true,                 % keeps spaces in text, useful for keeping indentation of code (possibly needs columns=flexible)
  keywordstyle=\color{blue},       % keyword style
  language=C,                      % the language of the code
  morekeywords={*,...},            % if you want to add more keywords to the set
  numbers=left,                    % where to put the line-numbers; possible values are (none, left, right)
  numbersep=5pt,                   % how far the line-numbers are from the code
  numberstyle=\tiny\color{mygray}, % the style that is used for the line-numbers
  rulecolor=\color{black},         % if not set, the frame-color may be changed on line-breaks within not-black text (e.g. comments (green here))
  showspaces=false,                % show spaces everywhere adding particular underscores; it overrides 'showstringspaces'
  showstringspaces=false,          % underline spaces within strings only
  showtabs=false,                  % show tabs within strings adding particular underscores
  stepnumber=1,                    % the step between two line-numbers. If it's 1, each line will be numbered
  stringstyle=\color{mymauve},     % string literal style
  tabsize=2,                       % sets default tabsize to 2 spaces
  title=\lstname                   % show the filename of files included with \lstinputlisting; also try caption instead of title
}

\usepackage{amssymb}

\newenvironment{myexampleblock}[1]{%
    \tcolorbox[beamer,%
    noparskip,breakable,
    colback=White,colframe=ForestGreen,%
    colbacklower=LimeGreen!75!White,%
    title=#1]}%
    {\endtcolorbox}

\newenvironment{myalertblock}[1]{%
    \tcolorbox[beamer,%
    noparskip,breakable,
    colback=White,colframe=Bittersweet,%
    colbacklower=Peach!75!White,%
    title=#1]}%
    {\endtcolorbox}

\newenvironment{myblock}[1]{%
    \tcolorbox[beamer,%
    noparskip,breakable,
    colback=White,colframe=RoyalBlue,%
    colbacklower=TealBlue!75!White,%
    title=#1]}%
    {\endtcolorbox}

\newenvironment{myexampleprogram}[1]{%
    \tcolorbox[beamer,%
    noparskip,breakable,
    colback=White,colframe=Goldenrod,%
    colbacklower=Yellow!75!White,%
    title=#1]}%
    {\endtcolorbox}
%--------
%\usepackage[magyar]{babel}
\title{Zufallszahlgenerator}
\begin{document}
\maketitle
Zufällige nummer sind sehr wichtig für uns. In numerischen Physik wir verwenden Zufallszahlen ganz often. Zum Beispiel 
in die einfachsten method für Integration von mehreren variablen Funkctions müssen wir Monte Carlo algorithm verwenden. Die
enthalten Statements, die vom Werte der zufälligen Variablen abhängt. In dieser Aufgabe wir müssen eine Zufallszahlgenerator
schreiben. Es gibt mehrere methode dazu, aber alle stammt aus linear Kongruenzen:
\begin{equation}
I_{j+1}=a I_{j} \left( \mathrm{mod} m\right).
\label{basics}
\end{equation}
Diesen Instruktion macht eine neue Zufallszahl ($I_{j+1}$)  aus einem original ($I_j$). Der qualität der Zufallszahl generator
hängt von dem Eingangsparameters $(a,m)$ ab. Eine gute Zufallszahlgenerator hat große Period, die Zeit zwischen den beiden gleichen
Zufallszahlen muss sehr groß sein. Park und Miller hat die folgenden parameter für a und m gewählt:
\begin{equation}
a=16807, m=2^{31}-1=2147483647. 
\end{equation}
Leider direkte implementation der Zufallszahlgenerator mit deisen parameters ist nicht möglich in \texttt{C}. Der Grund ist
dass wir nicht Zahlen größer als $m$ speichern können. Glücklicherweise gibt es eine Möglichkeit das Problem umzugehen. 
Wir faktorizieren $m$:
\begin{equation}
m= aq + r; r= m \left(\mathrm{mod}a\right); q= \left[m/a\right]
\end{equation}
Damit können wir die Gleichung (\ref{basics}) auch bewerten mit ($q,r$):
\begin{equation}
a I_j \left( \mathrm{mod} m\right)=  
\left\{ \begin{array}{rc}
a\left(I_j \mathrm{mod} q\right) -r \left[I_j/q\right] & \mathrm{wenn~es~}>0~\mathrm{ist} \\ 
a\left(I_j \mathrm{mod} q\right) -r \left[I_j/q\right] + m & \mathrm{andernfalls} \\ 
\end{array}\right.
\label{algo}
\end{equation}
wo $r=2836,q=127773$ ist. Implementieren Sie das Zufallszahlgenerator nach eq.\ref{algo}. Ändern Sie die Algorithm
um die Zahlen zwischen 0 und 1 zu sein werden.
Es ist wichtig zu test unser Ergebnis. Wir müssen kontrollieren die Verteilung (Distribution) der Zufallszahlgenerator.
Wir möchten gleichmäßige Vertailung erreichen. Wir werden eine Histogramms machen aus dem Zufallszahlen. Wir teilen 
das Interval in $n$ Teilen und zahlen wie oft die Zufallszahl in jedem Interval sinkt. Machen Sie ein Histogramm aus 
den Verfügbaren daten und prüfen wie gleichmäßig der Verteilung ist!
\end{document}
