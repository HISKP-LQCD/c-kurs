In Unix wir identifizieren ein Datein mit seinem Name. Wir können Funktionen aus dem Standardbibliothek verwenden, um
detallierte Informationen über ihnen zu erhalten. In diesem Fall die \texttt{C} Sprache bietet uns den \texttt{stat} Funktion 
und ihren Rückgabewert die \texttt{stat} Strukture. 
\begin{myexampleblock}{Definition: \texttt{struct stat}}
\begin{lstlisting}
struct tm{
  _dev_t st_dev:   /*Laufwerksnummer*/
  _ino_t st_ino;   /*Inode: eine grundlegende Datenstruktur zur Verwaltung von Dateisystemen mit unixartigen Betriebssystemen. */
  unsigned short st_mode; /*Modus, Berechtigungen*/
  short st_nlink;  /*Anzahl der Referenzen auf dem Datei*/
  short st_uid;    /*USER-ID*/
  short st_gid;    /*GROUP-ID*/
  _dev_t st_rdev:  /*Laufwerksnummer*/
  _off_t st_size;  /*Dateigroesse in Bytes */
  time_t st_atime; /*Datum des letzten Zugriffes*/
  time_t st_mtime; /*Datum des letzten Aenderung*/
  time_t st_ctime; /*Erstelldatum*/
}
\end{lstlisting}
\vspace{-0.4cm}
Das Definition befindet sich im Standardbibliothek \texttt{sys/stat.h}.
\end{myexampleblock}
\begin{myexampleblock}{Definition: \texttt{struct passwd}}
\begin{lstlisting}
struct passwd {
 char *pw_name; /*Anwaenders login Name*/
 uid_t pw_uid; /* UID id*/
 gid_t pw_gid; /* GID id*/
 char *pw_dir; /*Login Direktory*/
 ...
}; 
\end{lstlisting}
\vspace{-0.4cm}
Das Definition befindet sich im Standardbibliothek \texttt{passwd.h}.
Die Punkten stehen für Elementen, die uns nicht interessiert.
\end{myexampleblock}
\begin{myexampleblock}{Definition: \texttt{struct group}}
\begin{lstlisting}
struct group {
  char *gr_name;
  gid_t gr_gid;
  char **gr_mem;
  ...
};
\end{lstlisting}
\vspace{-0.4cm}
Das Definition befindet sich im Standardbibliothek \texttt{grp.h}.
Die Punkten stehen für Elementen, die uns nicht interessiert.
\end{myexampleblock}
\begin{myexampleblock}{Funktion: \texttt{stat}}
\begin{lstlisting}
int stat(char * path, struct stat * buf);
\end{lstlisting}
\vspace{-.4cm}
Information erhalten über das Datein bennant als $path$. Die Informations werden im $buf$ gespeichert.
Der Rückgabewert ist 0, wenn alles geklappt hat, und -1 bei einem Fehler.
\end{myexampleblock}
Bevor wir mit einem Datein arbeiten, müssen wir sicher sein, ob wir die entsprechende Rechte haben, um ihn zu 
erstellen oder einzulesen. Die Felden \texttt{uid} und \texttt{gid} in der \texttt{stat} Strukture geben Informationen 
über die Besitzer des Dateins. Der Supervisor hat immer \texttt{(0,0)} als \texttt{(uid,gid)}. Normalweise der erste Anwänder 
hat 1000 \texttt{uid}. Für die weitere Anwänders das \texttt{uid} wird um eins erhöht. 
Wir können auch das \texttt{uid} mit einem Anwänder's Name verbunden. Dafür gibt es zum Bespiel der \texttt{getpwuid} 
Funktion im Standardbibliothek. Die Anwänders in den unixartigen Betriebsystemen haben ihre login Name und gehören
auch zu einer Gruppe. Wenn in einem Computer mehrere Programmierer arbeiten, Sie können ihre Arbeit leicht teilen. 
Wir können das \texttt{gid} mit einem Name mit Hilfe von \texttt{getgrgid} Funktion verbunden.
Es kann nötig sein zu prüfen, ob die Datein beschädigt ist, oder nicht.  Bespielweise, wir können ihre Größe
überprüfen. Lass uns annehmen, dass wir die folgende Datei (example.txt) haben
\begin{lstlisting}{example.txt}
Hello world
\end{lstlisting}
Wir haben die folgende \texttt{C} Code geschrieben, um Informationen über diesen Datei erhalten:
\begin{lstlisting}
#include <stdio.h>
#include <stdlib.h>
#include <time.h>
#include <pwd.h>
#include <grp.h>
#include <sys/stat.h>
int main(int argc, char *argv[])
{
   struct stat *sb;
   struct passwd *pwuser;
   struct group *grpnam;
   int i;
   if (argc < 2)
   {
       printf("Usage: %s: file ...\n", argv[0]);
       exit(EXIT_FAILURE);
   }
   sb= (struct stat *)malloc(sizeof(struct stat));
   if (sb == NULL){
      printf("Memory allocation failure\n");
      exit(1);
   }
   for ( i = 1; i < argc; i++){
      if (stat(argv[i], sb) == -1){
          exit(EXIT_FAILURE);
      }
      if ( (pwuser = getpwuid(sb->st_uid)) == NULL )
      {
          exit(EXIT_FAILURE);
      }
      if ( (grpnam = getgrgid(sb->st_gid)) == NULL )
      {
          exit(EXIT_FAILURE);
      }
      printf("%s:\n", argv[i]);
      printf("\tinode: %u\n", (int)sb->st_ino);
      printf("\towner: %u (%s)\n", sb->st_uid, pwuser->pw_name);
      printf("\tgroup: %u (%s)\n", sb->st_gid, grpnam->gr_name);
      printf("\tperms: %o\n", sb->st_mode & (S_IRWXU | S_IRWXG | S_IRWXO));
      printf("\tsize: %ld\n", sb->st_size);
      printf("\tatime: %s", ctime(&(sb->st_atime)));
      printf("\tmtime: %s", ctime(&(sb->st_mtime)));
      printf("\tctime: %s", ctime(&(sb->st_ctime)));
      printf("\n");
   }
   free(sb);
   return 0;
}
\end{lstlisting}
Dieses Beispiel gibt  viele Neueigkeiten für uns. Zum ersten Mal verwenden wir die Eingabeparameters unseres 
Programms. Zur Erinnerung, der Wert vom \texttt{argc} erhaltet die Anzahl der Parameter, und im \texttt{argv[]} 
(Array von Zeiger auf \texttt{char} Typ) sind die Parameter gespeichert. Standardmäßig das erste Parameter ist das 
Name des Programms, dann kommen die Parameter, die wir hinzufügen haben. In der 13-ten Zeile prüfen wir die Anzahl 
der eingegebenen Parameter. Wenn wir keinen Parameter eingegeben haben, beenden wir das Programm mit einer Gebrauchanweisung.
Für die Anwänders das ist ein große Hilfe. Standartmäßig die Unix Programmen ausführen mit dem \texttt{--help} werden
ähnliche Gebrauchanweisungen zur Verfügung stellen.
In dem Programm wir verwenden jeden Parameter als ein Dateiname und erhalten ihre Eigenschaften mit Hilfe des \texttt{stat} 
Funktion. Es ist wichtig in der 18-ten Zeile Speicher für \texttt{sb} reservieren. Der Funktion \texttt{stat} geht davon aus, 
dass die Adresse, die sie bekommen habe, ist schon reserviert. In diesem Beispiel wir verwenden auch der \texttt{getpwuid} Funktion, 
mit dem wir die \texttt{uid} mit einem login Name verbunden. In unix Betriebsystemen wir können verschiedene Rechten für 
\begin{enumerate}
\item Besitzer, 
\item Gruppe vom Besitzer
\item Die Anderen
\end{enumerate}
geb en. Für ein Datei gibt es Rechte für Schreiben (\texttt{w}), Lesen(\texttt{r}) und Ausführen(\texttt{x}). Aus der Kommandozeile
können wir die Rechten mit \texttt{ls -la} überprüfen. Beispielweise:\\
\begin{verb}
>$ ls -la example.txt
\end{verb}\\
\begin{verb}
-rw-rw-r-- 1 pittler pittler 26 Maer  4 18:36 example.txt
\end{verb}\\
Das bedeutet, dass der Besitzer von \texttt{example.txt} ist der Anwänder von Name \texttt{pittler} aus der \texttt{pittler} Gruppe.
Der Besitzer und ihre Gruppe hat Recht für Lesen und auch Schreiben, die Anderen haben Rechte nur für Lesen.
In unserem Code wir determinieren die Rechten für die drei Oben Kategorie im Zeile 35. Jede Rechte wird ein 3\texttt{bit} binären
Zahl sein. Die Rechte für Lesen gehört zum höchsten \texttt{bit}, und die Rechte von Ausführen gehört zum niedrigsten \texttt{bit}. 
Beispielweise in unserer Ausgabe(Siehst du unter) \texttt{perms: 6} bedeutet Rechte für Schreiben und Lesen.
\begin{lstlisting}
./a.out example.txt
example.txt:
  inode: 11931732
  owner: 1000 (pittler)
  group: 1000 (pittler)
  perms: 664
  size: 12
  atime: Fri Mar  3 12:57:38 2017
  mtime: Fri Mar  3 12:57:31 2017
  ctime: Fri Mar  3 12:57:31 2017
\end{lstlisting}
Die Größe vom Datein kann überraschend sein, weil es nur aus 11 Zeichen besteht. Aber jeden Datein enthält die 
$EOF$(Ende des Dateins) Zeichen. Mit diesem Byte die Größe des Dateins stimmt.
