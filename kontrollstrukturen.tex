\section{Kontrollstrukturen: Verzweigungen und Schleifen}

Bisher haben wir einfache Anweisungen kennen gelernt, die vom Rechner nacheinander ausgeführt werden.
Wir werden nun Kontrollstrukturen einführen, die es erlauben, den Fluss eines Programmes zu beeinflussen.
Beispielsweise kann man, in Abhängigkeit von Werten von Variablen entweder einen, oder einen anderen Block von Anweisungen ausführen.
Solche Kontrollstrukturen nennt man Verzweigungen.
Das kann beispielsweise bedeuten, dass der Programmteil \texttt{A} ausgeführt wird, falls eine Variable $x$ größer als Null ist, und sonst der Programmteil \texttt{B}.
Die Entscheidung wird auf Grundlage logischer Ausdrücke gefällt, siehe den vorangegangenen Abschnitt.

\subsection{Einfache Verzweigung: \texttt{if-else}}

Im Allgemeinen hat das \emph{if-else} Konstrukt folgendes Aussehen:
\begin{lstlisting}[caption={if-else Statement}, belowcaptionskip=0.3em]
if (logischer Ausdruck)
  {
    // falls wahr
    Anweisung1;
    Anweisung2;
    ...;
  }
else // optional
  {
    // sonst
    Anweisung3;
    Anweisung4;
    ...;
  }
\end{lstlisting}
In Abhängigkeit vom logischen Ausdruck wird entweder der erste (nach \emph{if}) oder der zweite (nach \emph{else}) Codeblock ausgeführt.
Diese beiden Codeblöcke sind durch geschweifte Klammern definiert.
Die Klammern können auch weggelassen werden, wenn ein Block nur aus genau einer Anweisung besteht.
Wir empfehlen trotzdem auch in diesem Fall Klammern zu setzen.
Dies verhindert spätere Fehler, und macht den Quelltext lesbarer.
Der logische Ausdruck kann prinzipiell alle Operatoren enthalten, bzw. auch verkettete Ausdrücke von Operatoren, die wir in Tabellen~\ref{vergoper} und \ref{vergoper2} zusammengestellt haben.

\verb|Anweisung1| kann explizit auch wieder ein \emph{if-else} Ausdruck sein. 
Man kann \emph{if-else} also beliebig schachteln.
Außerdem ist der \emph{else} Block optional.
Wird der \emph{else} Block ausgelassen, so wird in dem Fall, in dem der logische Ausdruck zu \emph{falsch} ausgewertet wird, der Block mit Anweisungen nicht ausgeführt.

Nehmen wir beispielsweise an, wir wollen den Absolutwert einer Variablen \verb|a| ausgeben.
Dafür kann man einen \verb|if-else| Ausdruck verwenden.\index{\texttt{if-else}}
Der entsprechende Quelltext könnte so aussehen:
\begin{lstlisting}
#include <stdio.h>

int main()
{
  int a = 4;
  unsigned int absolutevalue = 0;
  if (a > 0) // a ist positiv
    {
      absolutevalue = a;  // Betrag ist direkt gleich a 
    }
  else       // a ist negativ
    {
      // falls a <= 0
      absolutvalue = -a;  // Betrag ist gleich -a
    }
  printf("Der Absolutwert von a ist %u\n", absolutevalue);
  return (0);
}
\end{lstlisting}
In Abhängigkeit davon, ob \verb|a > 0| zu wahr ausgewertet wird oder nicht, wird in diesem Beispiel entweder der Codeblock nach dem Schlüsselwort \verb|if| oder der nach dem Schlüsselwort \verb|else| ausgeführt.
Die Variable \verb|absolutevalue| ist außerhalb beider Blöcke deklariert, und damit in beiden sichtbar.
Die Zuweisung, die \verb|absolutevalue| innerhalb der Blöcke bekommt, ist damit auch nach Ende der Blöcke weiterhin erhalten.

\subsection{Mehrfache Verzweigung: \texttt{switch}}
 
\index{\texttt{switch}}
Ein dem \emph{if-else} Konstrukt verwandtes Konstrukt ist das \emph{switch} Konstrukt.
Es erlaubt eine ganze Liste von Alternativen abzuarbeiten.
Im Allgemeinen sieht das also so aus:
\begin{lstlisting}[caption={switch statement}, belowcaptionskip=0.3em]
switch (int)
  {
    case Wert1:
      Anweisung1;
      ....;
      break; // optional
    case Wert2:
      Anweisung2;
      ....;
      break; // optional
      ....
    default: // optional
      Anweisung3;
      break; // optional
  }
\end{lstlisting}
Das Konstrukt wird mit dem Schlüsselwort \verb|switch| eingeleitet.\index{\texttt{switch}}
Jede der Möglichkeiten beginnt mit dem Schlüsselwort \verb|case|, gefolgt vom möglichen Wert des Ausdruckes und einem Doppelpunkt. 
Falls die entsprechende Möglichkeit realisiert ist, werden alle noch folgenden Anweisungen ausgeführt, bis ein \verb|break| Schlüsselwort kommt, oder der \verb|switch| Block zu Ende ist.\index{\texttt{break}}
Der \verb|default| Zweig wird dann ausgeführt, wenn keiner der anderen Fälle gepasst hat.\index{\texttt{default}}

Im folgende Bespiel wird eine ganze Zahl von der Standardeingabe eingelesen, und zwar mit Hilfe der \verb|scanf| Funktion, die eine \verb|printf| sehr ähnliche Syntax hat.\index{\texttt{printf}}\index{\texttt{scanf}}
Wir diskutieren die Details zu \verb|scanf| später.
Dann wird, in Abhängigkeit vom eingegebenen Wert eine Ausgabe auf dem Bildschirm gemacht.
\begin{myexampleprogram}{Beispiel: \texttt{switch} statement}
\begin{lstlisting}[numbers=left]
#include <stdio.h>

int main()
{
  int n = 0;
  scanf("%d", &n);
  switch (n)
    {
      case 0:
        printf("Der Eingabewert ist 0\n");
        break;
      case 1:
        printf("Der Eingabewert ist 1\n");
      case 2:
        printf("Der Eingabewert ist 1 oder 2\n");
        break;
      case 3:
        printf("Der Eingabewert ist 3\n");
        break;
      default:
        printf("Der Eingabewert ist groesser als 3 oder kleiner als 0\n");
        break;
    }
  return (0);
}
\end{lstlisting}
\end{myexampleprogram}
D.h., wenn $0$ eingegeben wird, wird Zeile $10$ ausgeführt.
Wenn $1$ eingegeben wird, wird Zeile $13$ und $15$ ausgeführt.
Und, wenn keines von $0,1,2,3$ eingegeben wird, also keine der Möglichkeiten passt, so wird der \verb|default| Zweig ausgeführt, und damit Zeile $21$.

\subsection{Schleifen: \texttt{for}}

\index{Schleife!\texttt{for}}\index{\texttt{for}}
Eine wesentlich wichtigere Kontrollstruktur sind sogenannte \emph{for} Schleifen.
Allgemein sieht die \emph{for} Schleife wie folgt aus:
\begin{lstlisting}[caption={for Schleife}, belowcaptionskip=0.3em]
for (Ausdruck1; Ausdruck2; Ausdruck3)
  {
    Anweisung1;
    Anweisung2;
    ...;
  }
\end{lstlisting}
Im Einzelnen bedeutet das:
\begin{enumerate}
\item Zuerst wird \texttt{Ausdruck1} ausgewertet.\\
  In diesem ersten Ausdruck wird typischerweise die Schleifenvaraible deklariert und / oder initialisiert.
  (Es können auch mehrere Variablen deklariert und initialisiert werden.)
\item Dann wird \texttt{Ausdruck2} ausgewertet.\\
  Dieser zweite Ausdruck wird zu Beginn jeder Ausführung der Schleife ausgewertet und wird als logischer Ausdruck interpretiert.
  Er stellt die Abbruchbedingung dar.

  Wenn \texttt{Ausdruck2} zu wahr ausgewertet wird, werden die Anweisungen im Körper der Schleife ausgeführt. 
\item Nach Ausführung des Schleifenkörpers wird \texttt{Ausdruck3} ausgewertet.\\
  Hier werden typischerweise Schleifenvariablen modifiziert.
\end{enumerate}
Als Beispiel nehmen wir an, wir möchten die ganzen Zahlen von $0$ bis $n-1$ aufsummieren.
Dann können wir das mit folgendem Quelltext machen:
\begin{lstlisting}
#include <stdio.h>

int main()
{
  int n = 173;
  int summe = 0;
  for (int i = 0; i < n; i++) {
    summe += i;
  }
  printf("Die Summe der Zahlen von 0 bis %d ist %d\n", n - 1, summe);
  return (0);
}
\end{lstlisting}
Dabei ist \texttt{Ausdruck1} die Deklaration und Initialisierung von \verb|i|
\begin{lstlisting}
  int i = 0;
\end{lstlisting}
Das Abbruchkrieterium ist der logische Ausdruck
\begin{lstlisting}
  i < n;
\end{lstlisting}
Der Körper der Schleife besteht in diesem Fall nur aus der Zeile
\begin{lstlisting}
summe += i;
\end{lstlisting}
die für \verb|i=0,1,... n-1| nacheinander ausgeführt wird.
Diese Zeile wird also $n-1$--Mal ausgeführt, mit jeweils anderen Werten für \verb|summe| und \verb|i|.
Da \verb|summe| außerhalb der Schleife deklariert und initialisiert wurde, ist der entsprechende Wert auch nach der Schleife verfügbar.
\verb|i| dagegen ist in diesem Beispiel nur innerhalb der Schleife verfügbar!
\texttt{Ausdruck3} erhöht bei jedem Aufruf die Schleifenvariable \verb|i| um eins.
\begin{lstlisting}
  i++
\end{lstlisting}
Interessanterweise dürfen auch alle drei Ausdrücke leer sein.
Dann wird die Schleife im Prinzip unendlich oft ausgeführt.
In einem solchen Fall kann man die Schleife mit Hilfe von \verb|break| explizit abbrechen, wie man im folgenden modifizierten Beispiel sieht:\index{\texttt{break}}
\begin{lstlisting}
#include <stdio.h>

int main()
{
  int n = 173;
  int summe = 0;
  int i = 0;
  for (;;) { // korrekt, aber schlechter Stil!
    if (i == n) break;
    summe += i;
    i++;
  }
  printf("Die Summe der Zahlen von 0 bis %d ist %d\n", n - 1, summe);
  return (0);
}
\end{lstlisting}
Der Quelltext führt immer noch die gleiche Aufgabe aus. 
Allerdings stellt das Verwenden einer \texttt{for} Schleife in dieser Weise sehr schlechten Stil dar.
Der Grund ist, dass man am Schleifenkopf nicht mehr erkennen kann, wann die Schleife abgebrochen wird und was die Schleifenvariablen sind.
Der Quelltext wird also wesentlich unleserlicher und fehleranfälliger.

\index{\texttt{break}}
Die Benutzung von \verb|break| kann und ist aber an vielen Stellen sinnvoll.
Soll beispielsweise innerhalb einer Schleife eine externe Funktion aufgerufen werden, so sollte man immer überprüfen, ob diese Funktion auch ohne Fehler ausgeführt wurde.
Falls die Funktion einen Fehler meldet, kann man mit Hilfe von \verb|break| die Ausführung der Schleife unterbrechen:
\begin{lstlisting}
  for (int i = 0; i < n; i++) {
    int errorcode = myFunction(i);
    if( errorcode != 0 ) {
      // take appropriate action
      break;
    }
  }
\end{lstlisting}
Dies lässt sich für Fälle verallgemeinern, in denen unerwartete oder nicht standard Dinge innerhalb der Ausführung einer Schleife auftreten.

\subsection{Schleifen: \texttt{while} und \texttt{do-while}}

\index{\texttt{do-while}}\index{\texttt{while}}\index{Schleife!\texttt{do-while}}\index{Schleife!\texttt{while}}
C kennt zwei weitere Schleifenkonstrukte, nämlich die \verb|while| und die \verb|do-while| Schleife.
\texttt{for} Schleifen haben in der Regel eine fest vorgegebene Anzahl von Durchläufen.
Die \texttt{while} Schleifen Familie ist da flexibler: Die Anzahl an Durchläufen hängt nur von einem logischen Ausdruck ab.
Während die \verb|while| Schleife die Abbruchbedingung vor dem Ausführen des Schleifenkörpers überprüft und im Zweifel abbricht, wird bei der \verb|do-while| Schleife der Körper immer mindestens einmal ausgeführt.
Das macht den großen Unterschied zwischen den beiden aus.
Allgemein hat die \verb|while| Schleife folgende Form:
\begin{lstlisting}[caption={while Schleife}, belowcaptionskip=0.3em]
while (logischer Ausdruck)
  {
    Anweisungen;
  }
\end{lstlisting}
Hier ist der \texttt{logische Ausdruck} die Abbruchbedingung.
So lange der logische Ausdruck zu wahr ausgewertet wird, werden die Anweisungen im Schleifenkörper ausgeführt.
Die Abbruckbedingung steht bei der \verb|do-while| Schleife am Ende, wie man an der allgemeinen Form sieht:
\begin{lstlisting}[caption={do-while Schleife}, belowcaptionskip=0.3em]
do
  {
    Anweisungen;
  }
while (logischer Ausdruck);
\end{lstlisting}
Die Anweisungen werden also mindestens einmal ausgeführt bis der \texttt{logische Ausdruck} das erste Mal ausgewertet wird.

Ausführungen von \verb|while| und \verb|do-while| Schleifen können ebenfalls mit \verb|break| abgebrochen werden.
Natürlich können \verb|while|, \verb|do-while| und \verb|for| Schleifen immer ineinander überführt werden.
Allerdings ist dies manchmal mit Aufwand verbunden und es erscheint fast immer natürlich die eine oder andere Schleifenform zu verwenden.
\endinput
