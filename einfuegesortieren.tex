\begin{myexampleprogram}{ Programme: \texttt{Einfügesortieren}}
%\begin{myexampleprogram}{ Programme: \texttt{Einfügesortieren}}
Für Zusammenfassung der ersten Teil des Buches wir werden eine komplizierten Aufgabe Zeile für Zeile lösen.
Wir werden ganzen Zahlen mit dem Algorithmus vom Einfügesortieren sortieren. Die Aufgabe ist die Folgende:
Wir müssen $int$ Zahlen aus einem File einlesen, und wir müssen ihnen sortieren. Es ist gut zu errinern, 
dass alle Teil des Problems, der gut getrennte Aufgabe hat, müssen wir als ein Funktion verwirklichen.
In diesem Beispiel müssen wir ein Funkcion für readin (Einlesen) und für Sort (sortieren) schreiben.
Jetzt müssen wir über die Eingabe und Rückgabeparameters der Funkcions entscheiden.
Das read Funktion wird das Nummer der Zahlen in dem Datei bestimmen und die Zahlen in einem Array
speichern. Dies kann mit einem int Rückgabewert (Nummer des eingelesenen Zahlens) und mit einem Array
(speicher für die Zahlen) Eingabeparamter erreichen. Der Sort Funktion muss hat zwei Eingabeparameter haben:
die sortierende Liste und ihre Länge.

Wir werden für Einlesen der $scanf()$ Funkcion verwenden. Wir haben auch diesen Funktion für 
Einlesen verwendet. Hier wir nutzen auch ihre Rückgabewert. Wenn diese Rückgabewert $EOF$ ist, 
das bedeutet, dass wir das Ende des Datein erreicht haben. Der Wert von $EOF$ ist im $stdio.h$ definiert, 
und es ist -1.
\begin{lstlisting}
#define MAXNUM 10000
int read( int array[] ){
  unsigned int i;
  for (i = 0; i < MAXNUM; i++){
    if ( scanf("%d\n", &array[i])== EOF )
    break;
  }
  if (i == MAXNUM)
  printf("Sorry, wir konnten leider nicht die ganze Datei einlesen.");
  return i;
}
\end{lstlisting} 
In der ersten Zeile wir haben eine Symbolische Konstante definiert um die maximale Größe des
Arrays festzustellen. Wir haben eine lokale Variable $i$, das das Nummer vom eingelesenen Zahlen speichert.
Wenn die Rückgabewert von scanf EOF ist, wir beenden die Schleife, und geben der Wert von $i$ zurück.
Wenn wir unsere Grenze erreicht haben, drücken wir eine Nachricht aus.
Das war ganz einfach nur 9 Zeilen. Jetzt müssen wir der Funktion für Sortieren schreiben.
\begin{lstlisting}{}{
#define SWAP(x, y) do { typeof(x) swap = x; x = y; y = swap; } while (0)
void Sort( int unsortiert[], int n ){
   int sortiert[MAXNUM];
   int i,j;
   for (i=0; i<n; ++i){
     sortiert[i]=unsortiert[i];
     for (j=i; j>0; --j){
        if (sortiert[j] < sortiert[j-1] ){
           SWAP( sortiert[j], sortiert[j-1] );
        }
        else
           break;
     }
   }
   for (i=0; i<n; ++i)
     unsortiert[i]=sortiert[i];
}
\end{lstlisting}

Zuerst müssen wir erwähnen, dass unsere Sort Funktion hat zwei Eingabeparameter (das Array, das
wir sortieren möchten und ihre Länge) und hat keine Rückgabewert. Wir folgen den Algorithmus
bei dem Codeschreiben. Am Anfang haben wir zwei Arrays, eine sortierte und eine unsortierte. 
Dann kommt die Schleife für die Elementen des unsortierten Arrays. In jedem Schritt wir ziehen 
den aktuallen Element vom unsortierten Array zum Ende des sortierten Arrays (Zeile 7) um. 
Danach suchen wir für ihre rechtige  Position. Wir tauschen es mit dem Vorherigen aus bis es 
auf dem richtigen Position ist. Wenn wir die richtige Position gefunden haben, beenden wir die 
Schleife mit $break$. Wir haben ein Macro definiert um die Tauschen zu erledigen. Das wird nützlich 
in der Zukunft, weil wir den $SWAP$ nicht nur hier (in diesem Funktion)  verwenden können. 
Die Macros werden von dem Preprocessor verarbeiten. Grundsätzlich das Macrodefinition
wird copiert an Stelle von jeden Macroanrufs. Diese Macro hat zwei Eingabeparameter, die Zahlen (die Variablen), 
die wir tauschen wollen. Wir verwendem im Macro auch eine do while Schleife, aber die Anweisungen 
im Schleifenkern werden nur einmal Ausführen. Im macro wir haben eine lokale Variable von Namen $swap$ hergestellt,
deswegen brauchen wir die Schleife. Die Laufdauerzeit der swap Variable ist nur diese drei Anweisungen in dem 
Schleifenkern. Zum Ende wir kopieren die Werten aus dem sortierten Array zum unsortierten, weil der sortierte 
Array auch lokale ist.

\begin{lstlisting}
#include<stdio.h>
#define MAXNUM 10000
int read( int array[] ){
   int i;
   for (i=0;i<MAXNUM;i++){
     if ( scanf("%d\n", &array[i])== EOF )
      break;
   }
   if (i==MAXNUM)
     printf("Sorry, wir konnten leider
             nicht das ganze File gelesen.");
   return i;
}
#define SWAP(x, y) do { typeof(x) swap = x; x = y; y = swap; } while (0)
void Sort( int unsortiert[], int n ){
   int sortiert[MAXNUM];
   int i,j;
   for (i=0; i<n; ++i){
     sortiert[i]=unsortiert[i];
     for (j=i; j>0; --j){
        if (sortiert[j] < sortiert[j-1] ){
           SWAP( sortiert[j], sortiert[j-1] );
        }
        else
           break;
     }
   }
   for (i=0; i<n; ++i)
     unsortiert[i]=sortiert[i];
}
int main(){
   int array[MAXNUM];
   int i;
   int actual_legth=read(array);
   Sort(array, actual_length );
   for (i=0; i<length; ++i)
     printf("%d\n", array[i]); 
}
\end{lstlisting}
Jetzt grundsätzlich wir haben alles um das Programm zu verrvollständigen.
Der $main$ Funktion besteht nur 6 Zeile aus, mann kann es ganz einfach zu verstehen. Das
ist das größte Vorteil der modulierten Programmierung. Um unseren Quelltext verwenden zu können,
wir müssen zuerst es kompilieren.
\begin{lstlisting}
gcc myerste.c -o myerste
\end{lstlisting}
In diesem Fall das Name des ausführbaren Dateins ist $myerste$. Lasst uns annehmen, dass wir die folgende Datein 
unter dem Namen "Datequelle.txt" 
hergestellt haben:
\begin{lstlisting}
12
77
25
43
4
\end{lstlisting}
Dann wir können unsere Programm ausführen im Kommandoziele mit ihrem Name und das Quelle verwenden:
\begin{lstlisting}
./myerste < Datequelle.txt
\end{lstlisting}
Standardmäßig der $scanf$ Funktion lest aus dem Standardinput ein. Um aus dem Datein einlesen, wir müssen es
zum Standardinput richten  (Gleich werden wir lernen, wie können wir direkt aus einem File einlesen). Im bash 
wir können es mit dem < Operator erledigen.
\end{myexampleprogram}
