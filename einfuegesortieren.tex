\section{Anwendung: Einfügesortieren}

Als abschließendes und zusammenfassendes Beispiel für den ersten Teil dieses Skripts diskutieren wir nun noch ein etwas komplizierteres Beispiel.
Dabei wenden wir das bisher gelernte auf das schon erwähnte Sortierproblem an:
Ganze Zahlen $x_1, \ldots, x_n$ sollen eingelesen und dann sortiert ausgegeben werden.
Dafür unterteilen wir das Problem zunächst in kleinere Unterprobleme, das Einlesen der Zahlen und das Sortieren.
Dafür werden wir jeweils eine Funktion schreiben.

Wir beginnen mit dem Einlesen der zu sortierenden Zahlen.
Dafür nutzen wir wieder die Funktion \verb|scanf|.
Die Funktion, die wir \verb|read| nennen, sieht wie folgt aus
\begin{lstlisting}
#include <stdio.h>

int read(int array[], const int n)
{
  int i;
  for (i = 0; i < n; i++)
    {
      if (scanf("%d\n", &array[i]) == EOF)
        {
          break;
        }
    }
  if (i != n)
    printf("Sorry, es konnten nicht alle %d Zeichen eingelesen
                                %werden.\n", n);
  return i;
}
\end{lstlisting}
Die Funktion liest bis zu $n$ Zeichen ein, bzw. bis \verb|scanf| den Rückgabewert \verb|EOF| liefert.
Dieser Rückgabewert, der in \verb|stdio.h| definiert ist, bedeutet, dass \verb|scanf| das Ende des Eingabestroms erreicht hat.
Die eingelesenen Zahlen werde in \verb|array| gespeichert, das by reference übergeben wird.
Der Speicher für \verb|array| muss also von der aufrufenden Funktion bereitgestellt sein.
Der Rückgabewert unserer Funktion \verb|read| ist die Anzahl der eingelesenen Zahlen.

Als nächstes schreiben wir die Funktion zum Sortieren:
\begin{lstlisting}
void sort(int sortiert[], int unsortiert[], const int n)
{
  for (int i = 0; i < n; ++i)
    {
      sortiert[i] = unsortiert[i];
      for (int j = i; j > 0; --j)
        {
          if (sortiert[j] < sortiert[j - 1])
            {
              int swap = sortiert[j];
              sortiert[j] = sortiert[j - 1];
              sortiert[j - 1] = swap;
            }
          else
            break;
        }
    }
}
\end{lstlisting}
Diese Funktion hat drei Eingabeparameter: Das Array, dass die sortierten Zahlen enthalten soll, das Array, dass die zu sortierenden Zahlen enthält, und die Anzahl an zu sortierenden Zahlen.
Wieder muss die aufrufende Funktion dafür Sorge tragen, dass für beide Arrays ausreichend Speicher bereitgestellt wurde.
Überzeugen Sie sich bei den übrigen Zeilen, dass diese wirklich die $n$ Zahlen durch Einfügen in das Array \verb|sortiert| sortiert.

Bleibt noch die \verb|main| Funktion, die wie folgt aussieht:
\begin{lstlisting}
#include <stdio.h>

int read(int array[], const int n);
void sort(int sortiert[], int unsortiert[], const int n);

int main()
{
  const int MAXNUM = 10000;
  int array[MAXNUM], sortiert[MAXNUM];

  // read the numbers
  int actual_length = read(array, MAXNUM);

  // sort the numbers
  sort(sortiert, array, actual_length);

  // print them on the screen
  for (int i = 0; i < actual_length; ++i)
    {
      printf("%d ", sortiert[i]);
    }
  printf("\n");
}
\end{lstlisting}
Hier haben wir schon von der Möglichkeit Gebrauch gemacht, dass man Funktionen deklarieren kann, und die Definition später stattfinden kann.
Man kann also beispielsweise in eine Datei \verb|sort.c| erst den Code für die \verb|main| Funktion kopieren, und im Anschluss daran den Code für die beiden Funktionen.
Übersetzen kann man dann mit
\begin{verbatim}
gcc --std=c99 -o sort sort.c
\end{verbatim}
Der Name der ausführbaren Datei ist dann \verb|sort|.
Diese kann im Prinzip mit
\begin{verbatim}
$> ./sort
\end{verbatim}
von der Konsole ausgeführt werden.
Allerdings brauchen wir erst noch Eingabedaten.
Damit nicht alles von Hand eingegeben werden muss, nutzen wir die Funktionalität von Linux und leiten den Eingabestrom aus einer Datei um.
Wir erzeugen erst eine Textdatei mit Namen \verb|data.dat| mit folgenden Zeilen
\begin{verbatim}
12
77
25
43
4
\end{verbatim}
Damit kann man das Programm von der Kommandozeile wie folgt ausführen:
\begin{verbatim}
$> ./sort < data.dat
4 12 25 43 77
\end{verbatim}
Man kann, natürlich, wie im letzten Abschintt gezeigt, auch direkt aus \texttt{C} aus der Datei lesen.

\subsection{Header und Quelltextdateien}

Das gerade besprochene Beispiel eignet sich gut auf eine Möglichkeit hinzuweisen, die das Entwickeln von Programmen deutlich erleichtern kann.
Es ist nämlich möglich, den Quelltext für ein Programm auf verschiedene Dateien aufzuteilen.
Im obigen Beispiel könnte man beispielsweise die zwei Funktionen \texttt{sort} und \texttt{read} in eigenen Dateien speichern. 
Zweckdienlicherweise speichert man \texttt{read} in einer Datei \texttt{read.c}
\begin{lstlisting}[caption={Datei \texttt{read.c}}, belowcaptionskip=0.3em]
#include <stdio.h>
#include "read.h"
  
int read(int array[], const int n)
{
  // Funktionendefinition wie oben
  return i;
}
\end{lstlisting}
und die Funktion \texttt{sort} in einer Datei \texttt{sort.c}
\begin{lstlisting}[caption={Datei \texttt{sort.c}}, belowcaptionskip=0.3em]
#include "sort.h"
  
void sort(int sortiert[], int unsortiert[], const int n)
{
  // Funktionendefinition wie oben
  return;
}
\end{lstlisting}
Das neue Element in diesen beiden Dateien ist das Einbinden von zwei sogenannten \emph{header} Dateien, \texttt{read.h} und \texttt{sort.h}.\index{Header Dateien}
Das Einbinden erfolgt mit Hilfe von \texttt{include}, was im allgemeinen wie folgt aussieht
\begin{lstlisting}
#include "relative/path/header.h"
\end{lstlisting}
wobei der Pfad relativ zu dem Verzeichnis ist, in dem Übersetzt wird.
Diese beiden header Dateien benötigt man, um die Funktionendeklarationen in der \texttt{main} Funktion bekannt zu machen.
Und man bindet sie auch in den beiden Dateien \texttt{sort.c} und \texttt{read.c}, um die Deklarationen konsistent zu halten.
Die beiden header Dateien enthalten also lediglich die Funktionendeklarationen, also für die Funktion \texttt{read}
\begin{lstlisting}[caption={Datei \texttt{read.h}}, belowcaptionskip=0.3em]
#pragma once
  
int read(int array[], const int n);
\end{lstlisting}
und für die Funktion \texttt{sort}
\begin{lstlisting}[caption={Datei \texttt{sort.h}}, belowcaptionskip=0.3em]
#pragma once
  
void sort(int sortiert[], int unsortiert[], const int n);
\end{lstlisting}
In diesen beiden header Dateien ist das sogenannte Pragma \verb|#pragma once| hinzugekommen.\index{\texttt{\#pragma once}}
Es signalisiert dem Compiler, dass diese header Datei lediglich einmal einzubinden ist.
Auf diese Weise verhindert man fehlerhaftes, wechselseitiges Einbinden von header Dateien, was zu unendlichen Schleifen führen kann.
Die beiden header Dateien kann man nun nutzen, um die beiden Funktionen auch in der \texttt{main} Funktion bekannt zu machen.
Für \texttt{main} schreiben man auch eine eigene Datei, sagen wir \texttt{main.c}
\begin{lstlisting}[caption={Datei \texttt{main.c}}, belowcaptionskip=0.3em]
#include<stdio.h>
#include "read.h"
#include "sort.h"

int main() {
  // Funktionendefinition wie oben
  return 0;
}
\end{lstlisting}
Bleibt noch zu erklären, wie man das Programm übersetzt, wenn es in verschiedenen Dateien gespeichert ist.
Dafür kennt der Compiler die Möglichkeit, eine Datei nur zu übersetzen und nicht zu verlinken.
Zum Beispiel für die drei Dateien geht das wie folgt

\vspace*{0.5cm}
\begin{verbatim}
>$  gcc -std=c99 -Wpedantic -c main.c -o main.o
>$  gcc -std=c99 -Wpedantic -c sort.c -o sort.o
>$  gcc -std=c99 -Wpedantic -c read.c -o read.o
\end{verbatim}
\vspace*{0.5cm}

Das flag \texttt{-c} zeigt dem Compiler an, dass nur übersetzt werden soll.
Es wird eine sogenannte Objektdatei erzeugt, die typischerweise \texttt{.o} als Endung hat.\index{Objektdatei}
Das heißt, dass die beiden Funktionen jetzt zwar deklariert, aber noch nicht definiert.
Dementsprechend kann man eine Objektdatei auch nicht ausführen, man muss vorher noch linken.\index{Linken}
Dies geschieht mit

\vspace*{0.5cm}
\begin{verbatim}
>$  gcc main.o sort.o read.o -o main.exe
\end{verbatim}
\vspace*{0.5cm}

In diesem Schritt wird dann die Deklaration in \texttt{main.o} mit den Definitionen in \texttt{read.o} und \texttt{sort.o} verknüpft.
Ein weiterer Vorteil von mehreren Dateien ist, dass man immer nur den Teil neu übersetzen muss, den man geändert hat.
Dies ist unter Umständen ein immenser Zeitvorteil, wenn die Projekte etwas anspruchsvoller werden.
Man kann dies mit Hilfe des Programms \texttt{make} auch automatisieren.\index{make}

\endinput
