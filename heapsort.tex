\documentclass[tikz,12pt]{article}
\usepackage[utf8]{inputenc}
\usepackage[T1]{fontenc}
\usepackage[ngerman]{babel}
\usepackage[dvipsnames]{xcolor}
\usepackage{lipsum}
\usepackage{algorithm}
\usepackage{algorithmicx}
\usepackage{algpseudocode}
\usepackage{tikz}
\usepackage{pgfplots}

\usepackage{amsfonts}
\usepackage[intlimits]{amsmath}
\usepackage{cite}
\usepackage{epsfig}

\usepackage[usenames,dvipsnames]{pstricks}
\usepackage{pstricks-add}
\usepackage{epsfig}
\usepackage{wasysym}
\usepackage{pst-grad} % For gradients
\usepackage{pst-plot} % For axes

\addtolength{\hoffset}{-1.5cm}
\addtolength{\textwidth}{3cm}
\usepackage{listings}
\usepackage{color}
\definecolor{mygreen}{rgb}{0,0.6,0}
\definecolor{mygray}{rgb}{0.5,0.5,0.5}
\definecolor{mymauve}{rgb}{0.58,0,0.82}
\PassOptionsToPackage{svgnames}{xcolor}
\usepackage{tcolorbox}
\usepackage{lipsum}
\tcbuselibrary{skins,breakable}
\usetikzlibrary{shadings,shadows,graphs,shapes.geometric,arrows,fit,matrix,positioning}

\tikzset
{
    treenode/.style = {circle, draw=black, align=center, minimum
      size=1cm},
%    subtree/.style  = {isosceles triangle, draw=black, align=center,
%      minimum height=0.5cm, minimum width=1cm, shape border rotate=90,
%      anchor=north}
}

\usepackage{siunitx}
\lstset{ %
  backgroundcolor=\color{white},   % choose the background color; you must add \usepackage{color} or \usepackage{xcolor}; should come as last argument
  basicstyle=\footnotesize,        % the size of the fonts that are used for the code
  breakatwhitespace=false,         % sets if automatic breaks should only happen at whitespace
  breaklines=true,                 % sets automatic line breaking
  captionpos=b,                    % sets the caption-position to bottom
  commentstyle=\color{mygreen},    % comment style
  deletekeywords={...},            % if you want to delete keywords from the given language
  escapeinside={\%*}{*)},          % if you want to add LaTeX within your code
  extendedchars=true,              % lets you use non-ASCII characters; for 8-bits encodings only, does not work with UTF-8
  frame=single,                    % adds a frame around the code
  keepspaces=true,                 % keeps spaces in text, useful for keeping indentation of code (possibly needs columns=flexible)
  keywordstyle=\color{blue},       % keyword style
  language=C,                      % the language of the code
  morekeywords={*,...},            % if you want to add more keywords to the set
  numbers=left,                    % where to put the line-numbers; possible values are (none, left, right)
  numbersep=5pt,                   % how far the line-numbers are from the code
  numberstyle=\tiny\color{mygray}, % the style that is used for the line-numbers
  rulecolor=\color{black},         % if not set, the frame-color may be changed on line-breaks within not-black text (e.g. comments (green here))
  showspaces=false,                % show spaces everywhere adding particular underscores; it overrides 'showstringspaces'
  showstringspaces=false,          % underline spaces within strings only
  showtabs=false,                  % show tabs within strings adding particular underscores
  stepnumber=1,                    % the step between two line-numbers. If it's 1, each line will be numbered
  stringstyle=\color{mymauve},     % string literal style
  tabsize=2,                       % sets default tabsize to 2 spaces
  title=\lstname                   % show the filename of files included with \lstinputlisting; also try caption instead of title
}
\newenvironment{myexampleblock}[1]{%
    \tcolorbox[beamer,%
    noparskip,breakable,
    colback=White,colframe=ForestGreen,%
    colbacklower=LimeGreen!75!White,%
    title=#1]}%
    {\endtcolorbox}

\newenvironment{myalertblock}[1]{%
    \tcolorbox[beamer,%
    noparskip,breakable,
    colback=White,colframe=Bittersweet,%
    colbacklower=Peach!75!White,%
    title=#1]}%
    {\endtcolorbox}

\newenvironment{myblock}[1]{%
    \tcolorbox[beamer,%
    noparskip,breakable,
    colback=White,colframe=RoyalBlue,%
    colbacklower=TealBlue!75!White,%
    title=#1]}%
    {\endtcolorbox}

\newenvironment{myexampleprogram}[1]{%
    \tcolorbox[beamer,%
    noparskip,breakable,
    colback=White,colframe=Goldenrod,%
    colbacklower=Yellow!75!White,%
    title=#1]}%
    {\endtcolorbox}

\usepackage{amssymb}


\begin{document}

\section{Heap-Sort}

Während der Vorlesung wurde eine Verfahren eingeführt, um Zahlenreihen der Länge $N$ zu sortieren, das sogenannte Sortieren durch Einfügen.
Wir hatten auch gesehen, dass dieses Verfahren im schlechtesten Fall $\mathcal{O}(N^2)$ Vergleichs- und Vertauschungs-Operationen benötigt.
Wir werden nun ein Verfahren einführen, dass im schlechtesten Fall $\mathcal{O}(N\log(N))$ Vergleichs- und Vertauschungs-Operationen benötigt.
Es wird \emph{Heap-Sort} genannt, weil es auf einer Datenstruktur basiert, die man Halde oder \emph{Heap} nennt. 

Definieren wir zunächst, was man einen Heap nennt.
Eine Folge von Schlüsseln $F = k_1, k_2, \ldots, k_{N}$ nennt man einen Heap, wenn
\[
k_i \ \leq\ k_{\lfloor i/2 \rfloor}\,,\qquad 1\leq i \leq N\,.
\]
Anders ausgedrückt bedeutet dies $k_i \geq k_{2i}$ und $k_i\geq k_{2i+1}$, sofern $2i\leq N$ bzw. $2i+1\leq N$. 
Die Schlüssel $k_{2i}$ und $k_{2i+1}$ nennt man Nachfolger von $k_i$. 
Wichtig ist, dass zwar $k_i \geq k_{2i}$ und $k_i\geq k_{2i+1}$, aber der Heap keine Relation zwischen $k_{2i}$ und $k_{2i+1}$ impliziert. 
Per Definition eines Heaps ist natürlich $k_1 \geq k_i$ für alle $1 < i \leq N$. 

Unter den Schlüsseln $k_1, \ldots, k_N$ können Sie sich zunächst einfach ganze Zahlen vorstellen.
Aber im Prinzip können es allgemeine Daten sein, für die die Operation $\leq$ definiert ist.
Beispielsweise ist $F=8,6,7,3,4,5,2,1$ ein Heap mit $N=8$ im Sinne obiger Definition, wie man leicht verifiziert. 
Man stellt einen Heap am einfachsten mit Hilfe eines sogenannten balancierten, binären Baumes dar. 
Man nennt einen solchen binären Baum balanciert, wenn seine Höhe minimal ist für die Anzahl $N$ an Elementen.
Für das Beispiel sieht der binäre Baum wie folgt aus:
\begin{center}
  \begin{tikzpicture}[->,>=stealth',level/.style={sibling distance =
        5cm/#1,
        level distance = 2.5cm},scale=.8, transform shape]
    \node [treenode] {$k_1$:\\ Wert $8$}
    child
    {
      node [treenode] {$k_2$:\\ Wert $6$}
      child
      {
        node [treenode] {$k_4$:\\ Wert $3$}
        child
        {
          node [treenode] {$k_8$:\\ Wert $1$}
        }
      }
      child
      {
        node [treenode] {$k_5$:\\ Wert $4$}
      }
    }
    child
    {
      node [treenode] {$k_3$:\\ Wert $7$}
      child
      {
        node [treenode] {$k_6$:\\ Wert $5$}
      }
      child
      {
        node [treenode] {$k_7$:\\ Wert $2$}
      }
    };
  \end{tikzpicture}
\end{center}
Wir werden für balancierte, binäre Bäume der Konvention folgen, dass immer von links aufgefüllt wird.
Falls also ein Vertex keinen linken Nachfolger hat, so hat der Vertex auch keinen rechten Nachfolger.

Man sieht in diesem Beispiel auch sofort, dass jeder Unterbaum eines Heaps auch selbst wieder ein Heap ist. 
In unserem Beispiel bleiben beispielsweise nach Entfernen des ersten Elements zwei Teilheaps zurück, als $F_1 = 6,3,4,1$ und $F_2 = 7,5,2$.
Zum Sortieren entfernen wir das erste Element, also $k_1$ aus dem Heap.
Dann ersetzen wir es durch das Element mit dem größten Index, im Beispiel also $k_8$. 
Damit erhalten wir einen neuen Binärbaum, der allerdings die Heap Eigenschaften nicht erfüllt, denn $1$ is nicht größer als $6$ und $7$. 
\begin{center}
  \begin{tikzpicture}[->,>=stealth',level/.style={sibling distance =
        5cm/#1,
        level distance = 2.5cm},scale=.8, transform shape]
    \node [treenode] {$k_1$:\\ Wert $1$}
    child
    {
      node [treenode] {$k_2$:\\ Wert $6$}
      child
      {
        node [treenode] {$k_4$:\\ Wert $3$}
      }
      child
      {
        node [treenode] {$k_5$:\\ Wert $4$}
      }
    }
    child
    {
      node [treenode] {$k_3$:\\ Wert $7$}
      child
      {
        node [treenode] {$k_6$:\\ Wert $5$}
      }
      child
      {
        node [treenode] {$k_7$:\\ Wert $2$}
      }
    };
  \end{tikzpicture}
\end{center}
Der neue Binärbaum wird wieder zu einem Heap, indem man den neuen Schlüssel $k_1$ \emph{versickern} lassen.
Dies geschieht, indem wir ihn immer wieder mit dem größeren seiner beiden Nachfolger vertauschen.
Im Beispiel erhält man also nach dem ersten Versickerungsschritt
\begin{center}
  \begin{tikzpicture}[->,>=stealth',level/.style={sibling distance =
        5cm/#1,
        level distance = 2.5cm},scale=.8, transform shape]
    \node [treenode] {$k_1$:\\ Wert $7$}
    child
    {
      node [treenode] {$k_2$:\\ Wert $6$}
      child
      {
        node [treenode] {$k_4$:\\ Wert $3$}
      }
      child
      {
        node [treenode] {$k_5$:\\ Wert $4$}
      }
    }
    child
    {
      node [treenode] {$k_3$:\\ Wert $1$}
      child
      {
        node [treenode] {$k_6$:\\ Wert $5$}
      }
      child
      {
        node [treenode] {$k_7$:\\ Wert $2$}
      }
    };
  \end{tikzpicture}
\end{center}
und nach dem zweiten
\begin{center}
  \begin{tikzpicture}[->,>=stealth',level/.style={sibling distance =
        5cm/#1,
        level distance = 2.5cm},scale=.8, transform shape]
    \node [treenode] {$k_1$:\\ Wert $7$}
    child
    {
      node [treenode] {$k_2$:\\ Wert $6$}
      child
      {
        node [treenode] {$k_4$:\\ Wert $3$}
      }
      child
      {
        node [treenode] {$k_5$:\\ Wert $4$}
      }
    }
    child
    {
      node [treenode] {$k_3$:\\ Wert $5$}
      child
      {
        node [treenode] {$k_6$:\\ Wert $1$}
      }
      child
      {
        node [treenode] {$k_7$:\\ Wert $2$}
      }
    };
  \end{tikzpicture}
\end{center}
nach dem die Heapeigenschaft wieder hergestellt ist.
Diese Schritte wiederholt man, bis der Heap keine Schlüssel mehr enthält. 
Für ganze Zahlen könnte man das Versickern mit Hilfe eines Arrays wie folgt realisieren:
\begin{algorithmic}[1]
  \Procedure{versickere}{$a, i, m$}
  \State \textbf{input} array $a$, integer $i,m$
  \State \textbf{output} heap $a$
  \While{$2i+1 \leq m$}
  \Comment $a[i]$ hat linken Nachfolger
  \State $j=2i+1$
  \Comment $a[j]$ ist linker Nachfolger
  \If{$j < m$}
  \If{$a[j] < a[j+1]$}
  \State $j\leftarrow j+1$
  \Comment $a[j]$ ist jetzt groesster Nachfolger
  \EndIf
  \EndIf
  \If{$a[i] < a[j]$}
  \Comment Vertausche mit groesserem Nachfolger
  \State $a[j]\leftrightarrow\ a[i]$
  \State $i = j$
  \Else
  \State $i = m$
  \Comment Heapbedingung wieder erfuellt
  \EndIf
  \EndWhile
  \EndProcedure
\end{algorithmic}
%\begin{lstlisting}
%  // versickere a[i] bis hoechstens a[m]
%  void versickere(int a[], int i, const int m) {
%    while(2*i+1 <= m) { // a[i] hat linken Nachfolger
%      int j = 2*i+1;    // a[j] ist linker Nachfolger
%      if(j < m) {
%        if(a[j] < a[j+1]) j++;
%        // a[j] ist jetzt groesster Nachfolger
%      }
%      if(a[i] < a[j]) { // Vertausche mit groesserem Nachfolger
%        int t = a[i];
%        a[i] = a[j];
%        a[j] = t;
%        i = j;
%      }
%      else i = m;
%    } // Heapbedingung wieder erfuellt
%    return;
%  }
%\end{lstlisting}
Hierbei ist $m$ die Länge des Arrays und $i$ der Index des Elements, vom dem das Versickern starten soll.
Sobald wir also eine Folge mit Heapbedingung haben, können wir durch wiederholtes Anwenden dieses Verfahrens die Folge sortieren.
Allerdings müssen wir die Folge dafür erstmal in einen Heap verwanden. 
Dies kann man erreichen, indem man die Schlüssel $k_{\lfloor N/2\rfloor}$ bis $k_1$ versickern. 
Dadurch werden schrittweise immer größere Heaps aufgebaut, bis am Ende schließlich der Ausgangsheap vorliegt. 
Eine mögliche Implementierung könnte wie folgt aussehen:
\begin{algorithmic}[1]
  \Procedure{Heapsort}{$a, N$}
  \State \textbf{Input} $a,N$
  \State \textbf{Output} $a$
  \For{$i = N/2-1, N/2-2, \ldots, 0$}
  \State \textsc{versickere}$(a, i, N-1)$
  \EndFor
  \For{$i=N-1, N-2, \ldots, 1$}
  \State $a[i]\leftrightarrow a[0]$
  \State \textsc{versickere}$(a, 0, i-1)$
  \EndFor
  \EndProcedure
\end{algorithmic}
%\begin{lstlisting}
%  void HeapSort(int a[], const int N) {
%    // erzeuge Heap
%    for(int i = (N/2) - 1; i >= 0; i--) {
%      versickere(a, i, N-1);
%    }
%    // Sortiere durch versickern
%    for(int i = N-1; i > 0; i--) {
%      int t = a[i];
%      a[i] = a[0];
%      a[0] = t;
%      versickere(a, 0, i-1);
%    }
%    return;
%  }
%\end{lstlisting}
Implementieren Sie den Heap-Sort Algorithmus für eine Folge reeller Zahlen variabler Länge.

\textbf{Bemerkung}: Das Verfahren Heap-Sort sortiert eine Folge im schlechtesten Fall in $\mathcal{O}(N\log N)$ Schritten.
Das $\log N$ kommt aus der Höhe des Baumes.

Wir werden nun anstelle des Arrays eine spezielle Art sogenannter verketteter Listen verwenden.
Der Vorteil dieser verketteten Liste ist, dass keine Daten mehr kopiert werden müssen, sondern lediglich Zeiger auf solche Daten.
Das kann man zwar für Heapsort auch mit Indexarrays erreichen, aber als Übung betrachten wir nun verkettete Listen.
Betrachtet man einen Vertex in einem binären Baum, so ist er gekennzeichnet durch einen oder keinen Vorgängervertex, keinen, einen oder zwei Nachfolgervertizes und ein Datum.
Folgender Datentyp 
\begin{lstlisting}
typedef struct element
{
  Datum datum;            // die eigentlichen Daten
  struct element *left;   // Zeiger auf linken Nachfolger
  struct element *right;  // Zeiger auf rechten Nachfolger
  struct element *parent; // Zeiger auf den Vorgaenger
} element;
\end{lstlisting}
implementiert diese Eigenschaften.
Die eigentlichen Daten sind in \texttt{datum} vom Typ \texttt{Datum} gespeichert.
Wir werden den \texttt{NULL}-Zeiger verwenden, um keinen Vorgänger oder Nachfolger zu kennzeichnen.
Dementsprechend wird für die Wurzel des Baumes \texttt{parent = NULL;} gesetzt.
Blätter des Baumes mit nur einem oder keinem Nachfolger haben \texttt{left = NULL} und/oder \texttt{right = NULL}.
\texttt{Datum} kann ein beliebiger Datentyp sein, für den eine Vergleichsoperation $\leq$ definiert ist.

Im Prinzip können wir nun oben vorgestellten Algorithmus auf diesen Datentyp umstellen.
Dafür brauchen wir zunächst eine Funktion, die den Vergleich durchführt. 
In Pseudo-Code wäre das das folgende
\begin{algorithmic}[1]
  \Procedure{compare}{$e1, e2$}
  \State\textbf{input} Elemente $e1, e2$ vom Typ \texttt{element}
  \State \textbf{output} $0$ oder $1$
  \If{\texttt{e1.datum} $\leq$ \texttt{e2.datum}} 
  \State \textbf{return 1}
  \Else
  \State \textbf{return 0}
  \EndIf
  \EndProcedure
\end{algorithmic}
Für das Versickern lassen brauchen wir außer der Vergleichsoperation noch das Vertauschen von zwei Elementen. 
Betrachten wir nocheinmal folgenden Baum, der die Heapeigenschaft nicht erfüllt
\begin{center}
  \begin{tikzpicture}[->,>=stealth',level/.style={sibling distance =
        5cm/#1,
        level distance = 2.5cm},scale=.8, transform shape]
    \node [treenode] {$k_1$:\\ Wert $1$}
    child
    {
      node [treenode] {$k_2$:\\ Wert $6$}
      child
      {
        node [treenode] {$k_4$:\\ Wert $3$}
      }
      child
      {
        node [treenode] {$k_5$:\\ Wert $4$}
      }
    }
    child
    {
      node [treenode] {$k_3$:\\ Wert $7$}
      child
      {
        node [treenode] {$k_6$:\\ Wert $5$}
      }
      child
      {
        node [treenode] {$k_7$:\\ Wert $2$}
      }
    };
  \end{tikzpicture}
\end{center}
Um $k_1$ mit $k_3$ zu vertauschen, muss der linke Nachfolger von $k_3$ auf $k_2$, der rechte auf $k_1$ und der Vorgänger auf \texttt{NULL} gesetzt werden.
Genauso muss der Vorgänger von $k_1$ auf $k_3$, und der linke bzw. rechte Nachfolger auf $k_6$ bzw. $k_7$. 
Im Pseudo-Code sähe das wie folgt aus
\begin{algorithmic}[1]
  \Procedure{swap}{\texttt{parent}, \texttt{child}}
  \State\textbf{InOutput}: \texttt{parent}, \texttt{child} vom Typ \texttt{element}
  \State \texttt{child.parent} $=$ \texttt{parent.parent}
  \State \texttt{parent.parent} $=$ \texttt{child}
  \If{\texttt{child} $==$ \texttt{parent.left}}
  \State \texttt{parent.left} $=$ \texttt{child.left}
  \State \texttt{child.left} $=$ \texttt{parent}
  \State \texttt{child.right} $\leftrightarrow$ \texttt{parent.right}
  \Else
  \State \texttt{parent.right} $=$ \texttt{child.right}
  \State \texttt{child.right} $=$ \texttt{parent}
  \State \texttt{child.left} $\leftrightarrow$ \texttt{parent.left}
  \EndIf
%  \State \texttt{child} $\leftrightarrow$ \texttt{parent}
  \EndProcedure
\end{algorithmic}
Hierbei ist es wichtig, das bekannt ist, welches Element der Vorgänger und welches der Nachfolger ist. 
Man beachte, dass all diese Operationen lediglich Operationen auf Zeigern sind. 
Es werden also keine Daten kopiert, wenn man die Parameter \texttt{parent} und \texttt{child} als Zeiger übergibt.
Das gilt für alle nun folgenden Funktionen: Sie sollten \emph{call-by-reference} verwenden.
Damit kann man das Versickern implementieren.
Erstellen Sie zunächst Pseudo-Code für das Versickern mit dem Datentyp \texttt{element}.
Eine solche \textsc{versickern}-Funktion bekommt nur noch das Wurzelelement des (Unter-)Baumes übergeben, keine weiteren Parameter.

Etwas schwieriger als mit Arrays ist es, den anfänglichen Heap zu erzeugen.
Zunächst erzeugt man dafür einen balancierten binären Baum mit $N$ Elementen.
Schreiben Sie eine Funktion, die einen solchen zufälligen balancierten binären Baum mit $N$ Elementen erzeugt.

Nehmen wir an, das Wurzelelement dieses zufälligen, balancierten binären Baumes ist \texttt{root}.
Diesen binären Baum kann man zu einem Heap machen, indem man rekursiv die Funktion \textsc{versickern} anwendet.
Das Vorgehen ist dabei sehr ähnlich wie beim Heapsort mit Arrays.
Man definiert folgende Funktion
\begin{algorithmic}[1]
  \Procedure{heapify}{\texttt{root}}
  \State\textbf{Input} \texttt{root}
  \If{\texttt{root.left}$==$\texttt{NULL}}
  \State\textbf{return}
  \Comment \texttt{root} hat keine Nachfolger
  \Else
  \State\textsc{heapify}(\texttt{root.left})
  \Comment \texttt{root} hat einen linken Nachfolger
  \EndIf
  \If{\texttt{root.right}$!=$\texttt{NULL}}
  \State\textsc{heapify}(\texttt{root.right})
  \Comment \texttt{root} hat einen rechten Nachfolger
  \EndIf
  \State\textsc{versickern}(\texttt{root})
  \EndProcedure
\end{algorithmic}
Wie man sieht, handelt es sich um eine rekursive Funktion.
Sie ruft sich selbst wieder auf, bis ein Vertex ohne Nachfolger erreicht wird.
Wird die Funktion \textsc{heapify} mit dem Wurzelelement \texttt{root} des zufälligen, balancierten binären Baumes aufgerufen, so wird \textsc{heapify} so lange wieder aufgerufen, bis kein linker Nachfolger mehr existiert.
Dann wird eine Ebene darüber \textsc{versichern} aufgerufen.
So wird sukzessive von den Blättern des Baumes an für jede Ebene mit der Funktion \textsc{versickern} die Heapeigenschaft hergestellt. 
Implementieren Sie die Funktion \textsc{versickern} und die Funktion \textsc{heapify} in \texttt{C}.

Nun fehlen nur noch Kleinigkeiten für den kompletten Heapsort Algorithmus.
\begin{enumerate}
\item Wir benötigen eine Funktion \textsc{findlast}, die das letzte Element im Baum findet. 
  Dabei sollte man verwenden, dass der Baum balanciert ist.
\item Weiterhin brauchen wir eine Funktion \textsc{remove}, die ein Element aus dem Baum entfernt.
  Es dürfen dabei nur Elemente entfernt werden, die keine Nachfolger haben.

  Sinnvollerweise kombiniert man \textsc{findlast} und \textsc{remove}. 
  Man schreibt also \textsc{remove} so, dass nur Elemente entfernt werden dürfen, wenn das Element keine Nachfolger hat und nach dem Entfernen der binäre Baum noch balanciert ist.
\item Und wir benötigen eine Funktion \textsc{replaceroot}, die das Wurzelelement durch ein anderes Element ersetzt.
\end{enumerate}
Diese drei, bzw. zwei Funktionen sollten ohne große Schwierigkeiten implementierbar sein.
Entwerfen Sie Pseudo-Code für die Funktionen und implementieren Sie sie anschließend in \texttt{C}.

\end{document}
