\documentclass{uebungszettel}
\begin{document}
\begin{aufg}
Das folgende Programm sollte ein Array mit $n$ Zahlen 
allokieren. In den $i$-ten Eintrag $1/(i+1)$ schreiben und das Array ausgeben.
Dieses Programm benutzt die Funktion \verb|printarray| aus dem Modul \verb|"arrayhelpers"|.
Leider haben wir dabei $5$ Fehler gemacht. 
Schnapp sie dir alle! 

\begin{codelisting}
\begin{lstlisting}[numbers=left,numberstyle=\tiny,frame=tlrb,showstringspaces=false]
/* Array erstellen, mit 1/(i+1) fuellen und ausgeben. 
 * (c) 2015 Clelia und Johannes */

#include <stdio.h>
#include "arrayhelpers.h"

int main () {
   double *array;
   int n,i;   
	
   n = 42;
   /* Hole Speicher fuer n Eintraege */
   array = malloc (n);  
   if (array == NULL) {
      printf ("Fehler, nicht genug Speicher.\n");
   }
   for (i = 0;i < sizeof (array);i++){
      /* Schreibe 1/(i+1) in Array */
      array[i] = 1/(i+1);   
   }
   printarray (array, n); /* Gebe Array aus */
   free (array);	
   return 0;
}
\end{lstlisting}
\end{codelisting}
\end{aufg} 

\begin{aufg}
\begin{enumerate}
\item
Schreibe ein Programm, welches eine Datei im folgenden Format ausliest:
In der ersten Zeile steht die Anzahl der folgende Zeilen.
Die Zeilen sehen dann so aus:
\medskip \begin{codelisting}
\begin{lstlisting}[numbers=left,numberstyle=\tiny,frame=tlrb,mathescape=true]
cos(0.4) = 0.921
cos(0.45) = 0.9014
cos(0.6) = 0.82533
\end{lstlisting}
\end{codelisting}
Wenn in einer Zeile das Ergebnis um mehr als $10^{-3}$ von dem tats\"achlichen Wert abweicht, 
so soll diese Differenz mit Zeilennummer auf der Konsole ausgegeben werden.
\item * Modifiziere das Programm so, dass in der ersten Zeile nicht mehr die Anzahl der Eintr\"age
	stehen muss.
\end{enumerate}
\end{aufg}

\begin{aufg}
Diese Aufgabe läuft auf die Implementierung des Merge-Sort Algorithmus hinaus.
\begin{enumerate}
\item Implementiere eine Funktion \verb|merge|, die zwei bereits sortierte (eventuell verschieden große) Arrays als Argumente erhält, diese zu einem sortieren Array kombiniert und dieses zurück liefert. 
\item Die Funktion \verb|mergesort| selbst soll ein Array in zwei (möglichst gleich große) Teilarrays zerlegen, sich für diese Teilarrays selbst aufrufen und danach die dann sortierten Teilarrays mit der \verb|merge|-Funktion kombinieren. Erhält die Funktion ein Array mit keinem oder einem Element so belässt es dieses Array wie es ist, dann ist es nämlich bereits sortiert.
\item Schreibe ein Programm, dass einen Dateinamen auf der Kommandozeile entgegen nimmt, den Inhalt der Datei in ein \verb|int|-Array einließt, dieses sortiert und es dann wieder in die gleiche Datei zurück schreibt.
\end{enumerate}

Hier als Tipp ein Vorschlag für die Signaturen der beiden Funktionen:
\begin{codelisting}
\begin{lstlisting}[numbers=left,numberstyle=\tiny,frame=tlrb]
int *merge(int *list1, int n, int *list2, int m);
void mergesort(int *list, int n);
\end{lstlisting}
\end{codelisting}
\end{aufg}
\end{document}
