\documentclass{uebungszettel}
\begin{document}
\begin{aufg} 
Lese noch einmal im Skript die Sektion 7.5 und implementiere doppelt verkettete Listen, die \verb|double|-Variablen speichern.

\medskip\begin{codelisting}
\begin{lstlisting}[numbers=left,numberstyle=\tiny,frame=tlrb]
/* Definiere hier angemessene Strukturen fuer einen
   einzelnen Listeneintrag und die Liste selbst. */

/* Leere Liste erstellen */
LIST *list_create();

/* Element hinter E einfuegen, NULL heisst am Anfang */
LISTNODE *list_insert(LIST *L, LISTNODE *E, double p);

/* Element am Anfang bzw. Ende einfuegen */
LISTNODE *list_unshift(LIST *L, double p);
LISTNODE *list_push(LIST *L, double p);

/* Element am Anfang bzw. Ende entfernen und 
   die Daten zurueck geben */
double list_shift(LIST *L);
double list_pop(LIST *L);

/* ein Element aus der Liste entfernen */
void list_delete(LIST *L, LISTNODE *E);

/* zwei Listen zusammenfuegen */
LIST *list_merge(LIST *L, LIST *M);

/* Liste inklusive allen Elementen frei geben */
void list_free(LIST *L);

\end{lstlisting}
\end{codelisting}
\end{aufg}

\newpage
\begin{aufg}
~\begin{enumerate}
\item
Implementiere die Addition, Multiplikation, Potenzen und Division komplexer Zahlen. Verwende dazu folgende Header-Datei:
\begin{codelisting}
\begin{lstlisting}[numbers=left,numberstyle=\tiny,frame=tlrb]
#ifndef _COMPLEX__H
#define _COMPLEX__H

typedef struct _COMPLEX {
	double real;
	double imag;
} COMPLEX;

COMPLEX cplx_add(COMPLEX a, COMPLEX b);
COMPLEX cplx_mul(COMPLEX a, COMPLEX b);
COMPLEX cplx_pow(COMPLEX a, unsigned long n);
COMPLEX cplx_div(COMPLEX a, COMPLEX b);

#endif
\end{lstlisting}
\end{codelisting}
\item Implementiere die Addition, Multiplikation und Division sowie das Kürzen rationaler Zahlen. Schreibe dazu erst 
die Header-Datei. Überlege dir dazu zuerst, wie du die Funktionen aufrufen möchtest.
\end{enumerate}
\end{aufg}

\end{document}
