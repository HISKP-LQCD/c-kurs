\documentclass{uebungszettel}
\begin{document}

\begin{aufg} Installiere GMP und kompiliere folgendes Beispielprogramm 
\begin{codelisting}
\begin{lstlisting}[numbers=left,numberstyle=\tiny,frame=tlrb]
/* Gebe einige Zufallszahlen aus. Anwendungsbeispiel
   fuer die Verwendung der GMP. */

#include <stdio.h>
#include <gmp.h>
#include <time.h>

int main() {
  mpz_t a,r;
  int i; 

  /* Zustand des Zufallszahlengenerators: */
  gmp_randstate_t state; 

  /* Initialisiere den Zufallszahlengenerator */
  gmp_randinit_default(state);
  gmp_randseed_ui(state, time(NULL));

  mpz_init(a); 
  mpz_init(r);
  mpz_set_str(a,"856490000",10);

  for (i=0; i<20; i++) {
    mpz_urandomm(r,state,a);
    mpz_out_str(stdout,10,r);
    putchar('\n');
  }

  mpz_clear(r);
  mpz_clear(a);
  return 0;
}
\end{lstlisting}
\end{codelisting}
oder ein ähnliches (in Eclipse).
\end{aufg}

\pagebreak[4]

\begin{aufg}
\begin{enumerate}
\item Implementiere den naiven Faktorisierungsalgorithmus vom ersten Tag mit GMP und faktorisiere die Zahl
\begin{center}
\verb|272963285971849714829857456457|
\end{center}
\item Verwende OpenMP um deinen Primzahlalgorithmus zu parallelisieren.
\item Implementiere den
\begin{algorithm}[H]
\caption{Miller--Rabin Test}
\label{alg2}
\algsetup{indent=1.5em}
\begin{algorithmic}[1]
\REQUIRE Zahlen $a,n \in \mathbb{N}$ mit $a < n$.
\ENSURE ``Ja'', falls $a$ Zeuge für $n$, andernfalls ``Nein''.
\IF{$n$ gerade oder $\mathrm{ggT}(a,n) \ne 1$}
  \RETURN ``Ja''
\ENDIF
\STATE Schreibe $n - 1 = 2^k \cdot q$ mit $q$ ungerade.
\STATE \SET $a := a^q \bmod n$
\IF{$a \equiv 1 \pmod n$} 
  \RETURN ``Nein''
\ENDIF
\FOR{$i=1$ \TO $k$}
  \IF{$a \equiv -1 \pmod n$}
    \RETURN ``Nein''
  \ELSE
    \STATE \SET $a := a^2 \bmod n$
  \ENDIF
\ENDFOR
\RETURN ``Ja''
\end{algorithmic}
\end{algorithm}

\noindent und verwende ihn, um eine Funktion zu schreiben, die zu einer gegebenen Bitlänge $n$ eine Primzahl mit mindestens $n$ Bits findet. 
\item* Schreibe ein Modul, welches RSA--Verschlüsselung von beliebigen Arrays ermöglicht. Es ist dir überlassen, ob du zum Finden der Primzahlen deine eigene Funktion oder die Funktionen der GMP verwendest.
\end{enumerate}
\end{aufg}

\end{document}
