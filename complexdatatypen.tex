\section{Komplexe Datentypen}

\subsection{Eigene Datentypen mit \texttt{typedef}}

\index{\texttt{typedef}}
Je nach Problemstellung ist es manchmal sehr hilfreich, Datenstrukturen selbst erzeugen zu können.
C bietet dafür die Möglichkeit, und zwar auf verschiedene Weisen.
Zunächst kann man in C einen neuen Typ definieren, z.B. einen eigenen Typ für reelle Zahlen:
\begin{lstlisting}
typedef float real;
\end{lstlisting}
Dieser kann im Quelltext danach wie native C Typen verwendet werden
\begin{lstlisting}
typedef float real;
int main()
{
  real x, y = 3.718;
  return 0;
}
\end{lstlisting}
Auf den ersten Blick mag eine eigene Definition für eine Datentyp für reelle Zahlen nicht besonders sinnvoll erscheinen.
Allerdings kann man mit dieser Typdefinition zu einem späteren Zeitpunkt sehr einfach die verwendete Genauigkeit für reelle Zahlen ändern.
Nämlich einfach, indem man die Zeile für die Typdefinition durch
\begin{lstlisting}
typedef double real;
\end{lstlisting}
ersetzt.

\subsection{Zusammengesetzte Datentypen}

\index{\texttt{struct}}
Eine weitere Art, einen neuen Datentyp zu definieren, ist zusammengesetzte Datentypen zu definieren.
Ein einfaches mathematisches Beispiel ist eine Datentyp für komplexe Zahlen.
Eine komplexe Zahl muss dabei eine Real- und Imaginärteil haben.
Genau für solche Fälle stellt C die Möglichkeit zur Verfügung, zusammengesetzte Datentypen zu definieren.
Das entsprechende Schlüsselwort ist \verb|struct|.
Solche zusammengesetzten Datentypen in C kann man sich wie Kontainer vorstellen.
Die allgemeine Benutzung von \verb|struct| ist wie folgt:
\begin{lstlisting}
  struct name
  {
    Type1 name1;
    Type2 name2;
    Type3 name3;
    ...
  };
\end{lstlisting}
Die Verwendung als Typ für eine Variable:
\begin{lstlisting}
  struct name variablename;
\end{lstlisting}
Als Beispiel betrachten wir einen Datentyp für kartesiche Koordinaten mit $x$, $y$ und $z$ Komponente.
Der Datentyp wird wie folgt definiert:
\begin{lstlisting}
struct coord
{
  double x, y, z;
};
\end{lstlisting}
mit drei \verb|double| Elementen für den $x$, $y$ und $z$ Komponente.
Der Zugriff auf die einzelnen Elemente erfolgt über den Namen des \verb|struct| und den Namen des Elements, getrennt durch einen Punkt.
Für den allgemeinen Fall also wie folgt:
\begin{lstlisting}
  struct name
  {
    Type1 name1;
    Type2 name2;
    Type3 name3;
    ...
  };
  name.name1 = value;
\end{lstlisting}
In unserem Beispiel für den Koordinatentyp sieht dies wie folgt aus
\begin{lstlisting}
#include <stdio.h>

struct coord
{
  double x, y, z;
};
int main()
{
  struct coord r;
  r.x = 1.0;
  r.y = 3.4;
  r.z = -0.55;
  printf("r hat Koordinaten (%e, %e, %e)\n", r.x, r.y, r.z);
  return 0;
}
\end{lstlisting}
Also, allgemein gesprochen greift man auf die Elemente über 
\begin{lstlisting}
structname.elementname = wert;
\end{lstlisting}
zu.
Elemente können selbst wieder ein \verb|struct| sein.
Einen Unterschied gibt es beim Zugriff auf Elemente, wenn man einen Zeiger auf einen \verb|struct| benutzt.
Dort kann man direkt mit dem Pfeil Operator \verb|->| auf die Elemente zugreifen:
\begin{lstlisting}
#include <stdio.h>

struct corrd
{
  double x, y, z;
};
int main()
{
  struct coord r;
  struct coord *v = &r;
  v->x = 1.0;
  v->y = 3.4;
  v->z = -0.55
  printf("r hat Koordinaten (%e, %e, %e)\n", v->x, v->y, v->z);
  return 0;
}
\end{lstlisting}
Elemente eines \verb|struct| können natürlich auch Zeiger sein.
Allerdings muss dann Speicher außerhalb der Deklaration des \texttt{struct} reserviert werden.

\subsection{Beispiel: Datentyp für kartesische Koordinaten}

Die Verwendung von \verb|struct| ist sehr hilfreich, wenn beispielsweise eine Liste von logisch zusammengehörigen Daten bearbeiten soll.
Es muss dann nicht mehr jedes Element einzeln übergeben werden, sondern nur noch einmal der Kontainer.
Der Zusammenhang der einzelnen Elemente bleibt damit erhalten.
Man könnte nun beispielsweise eine Funktion für die Quadratnorm einer Koordinate wie folgt schreiben:
\begin{lstlisting}
#include <stdio.h>

struct coord
{
  double x, y, z;
};
double sqnorm(struct coord r) {
  return(r.x*r.x + r.y*r.y + r.z*r.z);
}

int main()
{
  struct coord r;
  r.x = 1.0;
  r.y = 3.4;
  r.z = -0.55;
  printf("r hat Quadratnorm %e\n'', sqnorm(r));
  return 0;
}
\end{lstlisting}
Mit einem \verb|struct| wird ebenfalls einer neuer Typ bereitgestellt.
Der entsprechende Name enthält aber immer das Schlüsselwort \verb|struct|, in unserem Fall \verb|struct coord|. 
Man kann einen \verb|struct| mit einem \verb|typedef| kombinieren, um nicht immer \verb|struct| schreiben zu müssen.
Im Quelltext sähe das so aus:
\begin{lstlisting}
struct coord
{
  double x, y, z;
};
typedef struct coord coord;
\end{lstlisting}
Das sieht etwas verwirrend aus, weil \verb|coord| zweimal auftaucht.
Aber der Ausdruck wird klar, wenn man sich erinnert, dass \verb|struct coord| im Prinzip ein Typ ist.
Und damit ist der Name \verb|coord| noch nicht vergeben.
Die etwas gebräuchlichere Art obiger Typdefinition geht über die Möglichekeit eines anonymen \verb|struct|.
Man kann nämlich auch schreiben:
\begin{lstlisting}
typedef struct
{
  double x, y, z;
} coord;
\end{lstlisting}
Damit ist natürlich \verb|struct coord| nicht definiert, aber man kann \verb|coord| wie einen C Typ verwenden.


\endinput

Ein Beispiel aus der C Standardbibliothek \verb|time.h| ist das \verb|struct tm| für Datum und Zeit:
\begin{lstlisting}
struct tm
{
  int tm_sec; // Sekunde
  int tm_min; // Minute
  int tm_hour; // Stunde
  int tm_mday; // Tag
  int tm_mon; // Monat
  int tm_year; // Jahr
  int tm_wday; // Wochentag
  int tm_yday; // Tag im Jahr
  int tm_isdst; // Sommerzeit oder nicht
};
\end{lstlisting}
Damit kann über die Funktion
\begin{myexampleblock}{Funktion: \texttt{time}}
\begin{lstlisting}
time_t time(time_t *t)
\end{lstlisting}
  \verb|time| gibt die Zeit in Sekunden seit 1970-01-01 00:00:00 +0000 (UTC) zurück.
  Falls \verb|t| nicht \verb|NULL| ist, wird das Ergebnis zusätzlich in \verb|t| gespeichert.
  \verb|time_t| ist vom Typ \verb|long int|.
\end{myexampleblock}
und die Funktion
\begin{myexampleblock}{Funktion: \texttt{localtime}}
\begin{lstlisting}
struct tm *localtime(time_t *timer);
\end{lstlisting}
  Eingabeparameter: Zeiger auf eine Variable vom Typ \verb|time_t|.
  Rückgabeparameter: Zeiger auf ein \verb|struct tm|.
\end{myexampleblock}
die beide in \verb|time.h| deklariert sind.
Folgendes Beispiel illustriert deren Benutzung:
\begin{lstlisting}
#include <stdio.h>
#include <stdlib.h>
#include <time.h>

int main()
{
  time_t now;
  struct tm *local_date_time;
  char *tagarray[7] = {"Sonntag",    "Montag",  "Dienstag", "Mittwoch",
                       "Donnerstag", "Freitag", "Samstag"};
  now = time(NULL);
  local_date_time = localtime(&now);

  printf("Jahr: %d\n", local_date_time->tm_year + 1900);
  printf("Monat: %d\n", local_date_time->tm_mon);
  printf("Tag: %s\n", tagarray[local_date_time->tm_wday]);
  printf("Stunde: %d\n", local_date_time->tm_hour);
  printf("Minute: %d\n", local_date_time->tm_min);
  printf("Sekunde: %d\n", local_date_time->tm_sec);
  return (0);
}
\end{lstlisting}
Es gibt das aktuelle Datum und die aktuelle Zeit aus.
Dabei wird zunächst die Funktion \verb|time| aufgerufen, und desse Rückgabewert dann an die Funktion \verb|localtime| übergeben.
In beiden Funktionsaufrufen mussten wir nicht mit den $9$ Elementen von \verb|struct tm| hantieren, sondern konnten bequem den Container übergeben.
Da \verb|local_date_time| ein Zeiger ist, greifen wir auf dessen Elemente wieder mit dem \verb|->| Operator zu.
