\section{Complex Datatypen}
Wir haben gelernt, wie wir mehrere Objekte von dem gleichen Typ händeln
können. Aber often das ist nicht genug. Wenn in unserem Problem
wir zusammenhörende Variablen haben, müssen wir ihnen zusammen handeln
als ein Stukture. In der C Sprache wir können neuen Datentypen aus
den Elementaren herstellen, dafür müssen wir $struct$ Strichwort
verwenden.
\begin{myalertblock}{struct Typdefinierung}
\begin{lstlisting}
struct name {
    Type1 typename1;
    Type2 typename2;
    Type3 typename3;
    ...
}
\end {lstlisting}
\vspace{-0.4cm}
Verwendung, definieren Variable aus $name$ Typ:
\begin{lstlisting}
struct name variablename;
\end{lstlisting}
\vspace{-0.4cm} 
Die Typen in einer Strukture können auch Strukture sein.
Sie müssen geschlossen sein, was bedeutet, dass ihre Größe
(wievielen Speicherzellen verwenden Sie) in der Kompilierung
bekannt sein muss.
\end{myalertblock}
Ein Strukture besteht aus Objekten von verschidenen Typen. 
Mann kann auch ein Objejt in einer Structure mit $Record$ 
oder $Feld$ bezeichnet.

Wir können mit dem Punktoperator (.) zu den eigenen Felden
griffen. In dem unteren kleinen Quelltext wir zeigen ein Beispiel
für Strukture Datatyp.
\begin{lstlisting}
#include <stdio.h>
struct Student {
    int matrikelnummer;
    char *name;
    char klasse[5];
} int main() {
    struct Student erste, zweite;
    erste.matrikalnummer = 99;
    erste.name = "Martin Pfeindrich";
    erste.klasse = "5a";
    zweite = {101, "Andrea Tabarrios", "6b"};
}
\end{lstlisting}
In diesem Beispiel wir haben das Student stukture definiert in der Zeilen zwischen 2-7.
Das musst außerhalb des $main$ Funktions sein, um alle andere Funktionen zugriffen zu können.
In der neuenten Zeile wir haben zwei Variable von Student Typ definiert. Sie sind 
nicht initializiert noch. In der zehnten Ziele wir griffen zu den Recorden in der
Student Strukture mit dem Punkt operator. Allgemeinen die Zuweisung hat der folgende Form:
\begin{lstlisting}
Variablename.structurenecodname = wert;
\end{lstlisting}
Natürlich die structurenrecordname oben kann also ein Struktured Typ sein, d.h wir können
Strukturen schachteln. Wir können auch nicht nur mit dem Punktoperator initializieren, sondern
auch mit einer Aufzählung der Werten getrennt durch die (,) Operator.

Natürlich wir können nicht alle Elementare operatoren auf Selbst
erstellte Strukturen verwenden. Wir können zwischen zwei Elementen zuweisung machen zum Beispiel.
In diesem Fall die Werten von alle Speicherzellen, die die Strukture aufbauen werden in der
anderen Strukturevariable copyieren. Wir können auch Arrays  Elementen von Strukturen typ, 
und deswegen die Dereferenzierung und die adresse Operator verwendet werden.

Strukuren sind hilfreich auch wenn ein Fuktion mehr ($n$) zusammenhörige Änderung machen soll. 
Anstatt $n$ Eingabeparameter zu verwenden, genügt es, einen zu verwenden. Im Standardbibliothek
C also hat Complex Stukture Datatypen. Zum Beispiel, die Funktions, die die Zeit verarbeiten.

\begin{myexampleblock}{Definition: \texttt{struct tm}}
\begin{lstlisting}
struct tm {
    int tm_sec; /*Sekundums im Minute*/
    int tm_min; /*Minutes im Uhr*/
    int tm_hour; /*Uhr am Tag*/
    int tm_mday; /*Tag im Monat*/
    int tm_mon; /*Monat im Jahr*/
    int tm_year; /*Jahren seit 1900*/
    int tm_wday; /*Tage seit Sonntag*/
    int tm_yday; /*Tage seit Januar 1*/
    int tm_isdst; /*Sommerzeit ist oder nicht*/
}
\end{lstlisting}
Das Definition befindet sich im Standardbibliothek Funktion time.h.
\end{myexampleblock}
\begin{myexampleblock}{Funktion:\texttt{time}}
\begin{lstlisting}
time_t time(time_t *t)
\end{lstlisting}
\vspace{-0.4cm}
Kehrt die Zeit in Sekundum seit (00:00:00, January 1, 1970) zurück.
Eingabeparameter ist die Adresse ein Variable von long int Typ, in der
das Ergebnis gespeichert wird, oder $NULL$ Pointer.
\end{myexampleblock}
\begin{myexampleblock}{Funktion:\texttt{localtime}}
\begin{lstlisting}
struct tm *localtime(time_t *timer);
\end{lstlisting}
\vspace{-0.4cm}
Eingabeparameter :Adresse einer Variable von long int Typ (often die Rückgabeparameter vom Fuktion $time$).
Mit den Werten von $timer$ füllt dieser Funktion die felder in der $tm$ Strukture aus, und kehrt ihn zurück.
\end{myexampleblock}
\begin{lstlisting}
#include <stdio.h>
#include <stdlib.h>
#include <time.h>
int main(void) {
    time_t now;
    struct tm *local_date_time;
    char *tagarray[7] = {"Sonntag",    "Montag",  "Dienstag", "Mittwoch",
                         "Donnerstag", "Freitag", "Samstag"};
    now = time(NULL);
    local_date_time = localtime(&now);
    printf("Jahr ist: %d\n",
           local_date_time->tm_year +
               1900); /*Auch (*local_date_time).tm_year+1900 ist ok*/
    printf("Monat ist: %d\n", local_date_time->tm_mon);
    printf("Tage ist: %s\n", tagarray[local_date_time->tm_wday]);
    printf("Uhr ist: %d\n", local_date_time->tm_hour);
    printf("Min ist: %d\n", local_date_time->tm_min);
    printf("Sek ist: %d\n", local_date_time->tm_sec);
}
\end{lstlisting}
In diesem Beispiel wir drücken das aktuellen Datum auf Monitor aus. Zuerst rufen wir den $time$ Funktion
an damit wir  im $now$ das Datum als Sekunden in einem Long int Zahl haben. Danach graben wir von diesem Zahl 
verständnisvoll Informationen aus mit dem $localtime$ Funktion. Ihre Rückgabe ist eine Pointer auf $tm$ Strukture. 
Wir haben gelernt, wie wir zu den Strukture Elementen griffen können. Wir können auch Direkt vom Pointer
zu den Elementen griffen mit dem $->$ Operator. Auch in der Achten Zeile wir haben ein Variable mit komplizierten
Typ definiert als $tagarray$. Wie der Name schon sagt, das ist ein Array von Pointers auf "char". Wir verwenden
die Struktureelement $tm\_wday$, als ein Index für diesen Array in der dreizehnten Zeile.
