\section{Datein verarbeiten}

Ein- und Ausgabe in Dateien, oder allgemein Ein- und Ausgabe auf Geräten ist nicht Teil von C selbst.
Aber die Standardbibliothek stellt diese Funktionalität zur Verfügung.
Sie basiert auf dem Konzept von sogenannten \emph{streams}.
Der Name kommt daher, dass Ein- und Ausgabe immer ein serieller Prozess ist.
D.h., in Einheiten bestimmter kleinster Elemente werden die Daten sukzessive eingelesen oder ausgegeben.
Man kann es sich also als einen Datenstrom vorstellen.

Ein Strom in \verb|stdio.h| ist immer vom Typ \verb|FILE*|.
Es gibt einige vordefinierte streams:
\begin{enumerate}
\item \verb|stdin| : Standardeingabe (haben wir schon implizit mit \texttt{scanf} verwendet) 
\item \texttt{stdout} : Standardausgabe (haben wir schon implizit mit \texttt{printf} verwendet) 
\item \texttt{stderr} : Standardfehler
\end{enumerate} 
Standard Ein- und Ausgabe haben wir schon implizit mit \texttt{scanf} und \texttt{printf} benutzt.
Funktionen, die ebenfalls direkt auf \verb|stdin| und \texttt{stdout} arbeiten, sind \verb|getchar| and \verb|putchar|.
Sie lesen bzw. schreiben genau ein Zeichen von \verb|stdin| bzw. \texttt{stdout}.
Ihre Deklaration in \verb|stdio.h| sieht wie folgt aus:
\begin{lstlisting}
  int getchar ();
  int putchar (int c);
\end{lstlisting}
In \verb|putchar| wird ein \emph{cast} von \verb|c| nach \verb|unsigned char| durchgeführt.
\verb|getchar| liest ein Zeichen als \verb|unsigned char| vom \verb|stdin| und führt dann einen \emph{cast} nach \verb|int| durch.
Damit kann man beispielsweise ein Programm schreiben, dass alle eingegenen Zeichen direkt wieder auf \texttt{stdout} ausgibt:
\begin{lstlisting}
  #include<stdlib.h>
  #include<stdio.h>
  
  int main(){
    int c;
    while( (c=getchar()) != EOF) {
      putchar(c);
    }
    return(0);
  }
\end{lstlisting}
Wiederum wird solange eingelesen, bis das Sonderzeichen \verb|EOF| gefunden wird.

Wenn wir direkt aus einer Datei lesen wollen, so müssen wir einen \emph{stream} bekanntmachen.
Dies geht mit Hilfe der Funktion \verb|fopen|.
\begin{myexampleblock}{Funktion: \texttt{fopen}}
  \begin{lstlisting}
    FILE *fopen( const char *filename, char mode);
  \end{lstlisting}
  \vspace{-0.4cm}
  Öffnet eine Datei. Dabei gibt \texttt{mode} an, für welchen Zweck die Datei geöffnet werden soll:
  \begin{itemize} 
    \itemsep0.5ex
  \item \texttt{"w"}: Datei zum Schreiben öffnen. Wenn sie schon existiert, wird sie überschrieben. Wenn sie nicht existiert, wird die Datei erzeugt.
  \item \texttt{"r"}: Öffnen einer Datei ausschließlich zum lesen. Der \emph{stream} zeigt auf den Anfang der Datei.
  \item \texttt{"{}a"}: Öffnen einer Datein zum Schreiben. Wenn sie schon existiert, wird am Ende der Datei hinzugefügt.
  \item \texttt{"w+"}: Öffnen zum Schreiben und Lesen. Wenn die Datei schon existiert, wird sie überschrieben. Der \emph{stream} zeigt auf den Anfang der Datei.
  \item \texttt{"r+"}: Öffnen einer Datei zum Schreiben und Lesen. Der \emph{stream} zeigt auf den Anfang der Datei.
  \item \texttt{"{}a+"}: Öffnen einer Datei zum Lesen und Hinzufügen. Zum Lesen zeigt der \emph{stream} auf den Anfang der Datei. Geschriebenes wird immer hinzugefügt.
  \end{itemize}
\end{myexampleblock}  
Eine Datei, die mit \texttt{fopen} geöffnet wurde, muss mit \texttt{fclose} wieder geschlossen werden:
\begin{myexampleblock}{Funktion: \texttt{fclose}}
  \begin{lstlisting}
    int fclose(FILE *stream);
  \end{lstlisting}
  \vspace{-0.4cm}
  Der Rückgabewert ist \texttt{0}, wenn die Datei erfolgreich geschlossen werden konnte und \texttt{EOF}, wenn nicht.
\end{myexampleblock}
Nur, wenn \texttt{fclose} aufgerufen wurde ist sichergestellt, dass die Daten auch in die Datei geschrieben wurden.

Hat man einmal eine Datei geöffnet, muss man sich entscheiden, wie man sie ausliest oder in sie schreibt.
Wir führen hier zunächst die sogenannte formatierte Ein- und Ausgabe ein.
Dabei wird jedes gelesene Byte als ein ASCII Zeichen interpretiert.
Wir haben diese Art von Ein- und Ausgabe bereits mit \verb|printf| und \verb|scanf| kennengelertn.
Die Verallgemeinerung von \verb|scanf| für beliebige \emph{streams} ist \verb|fscanf|
\begin{lstlisting}
  int fscanf(FILE * stream, char *format, ...);
\end{lstlisting}
\verb|fscanf| bekommt als ersten Parameter den \emph{stream} übergeben.
\verb|format| ist eine Zeichekette, wie wir sie schon von \verb|scanf| kennen.
Daher kennen wir auch schon, dass man Variablen über spezielle Zeichfolgen, die mit \% beginnen, aus dem \emph{stream} zuweisen kann.
Dabei stehen unter anderem die folgenden speziellen Zeichen zur Verfügung
\begin{center}
  \begin{tabular}{lr}
    \hline
    \texttt{Specifizierungszeichen} & Bedeutung \\\hline
    \texttt{\%d}	&  \texttt{int} \\
    \texttt{\%ld}  &  \texttt{long int} \\
    \texttt{\%ud}  &  \texttt{unsigned int} \\
    \texttt{\%f,\%e}   & \texttt{float} \\
    \texttt{\%lf,\%le}  & \texttt{double} \\
    \texttt{\%c}  & \texttt{char} \\
    \texttt{\%s}  & eine Zeichenkette\\
    \hline
  \end{tabular}
\end{center}
Alle anderen Zeichen im Formattierungstext werden eingelesen, aber nicht gespeichert. 
Nehmen wir an, wir müssen die Zahlen aus folgender ASCII Datei einlesen:
\begin{lstlisting}
1212
        1222
999             12212
888
\end{lstlisting}
Dies kann mit folgendem Code gemacht werden
\begin{lstlisting}
  #include<stdio.h>
  #include<stdlib.h>

  int main(int argc, char *argv[]){
    if (argc>2 ) {                       // wurde ein Dateiname uebergeben?
      fprintf(stderr,"Useage: ./read_int_ascii filename\n");
      return(-3);
    }
    FILE * in = fopen(argv[1], "r");     // Oeffne die Datei
    if(in == NULL) {                     // Ueberpruefe auf Fehler 
      fprintf(stderr, "Error opening the file\n");
      return(-2);
    }
    printf("Lese aus der Datei %s\n", argv[1]);
    int d;
    while( fscanf(in, "%d", &d) == 1 ) { // lese ein int nach dem anderen ein
      printf("%d\n", d);
    }
    if (fclose(in) != 0) {               // Schliesse die Datei wieder
      printf("Error closing the File\n");
      return(-3);
    }
    return(0);
  }
\end{lstlisting}
Es fällt auf, dass wir als Formatierungszeichenkette lediglich \verb|%d| verwenden, und keine Leerzeichen.
Der Grund dafür ist, dass Leerzeichen (und Zeilenumbrüche, Tabs etc.) als Trennzeichen interpretiert werden.
Wie viele Leerzeichen und Zeilenumbrüche zwischen den Zahlen eingefügt sind, spielt für die formatierte Eingabe keine Rolle.
Die Ausgabe des Programms sieht dann wie folgt aus:
\begin{verbatim}
Lese aus der Datei numbers.txt
1212
1222
999
12212
88
\end{verbatim}
Die formatierte Ausgabe funktioniert ähnlich, nur mit der Funktion \verb|fprintf|
\begin{lstlisting}
  int fprintf(FILE * stream, char *format, ...);
\end{lstlisting}
Die Spezifizierungszeichen sind die gleichen wie für \verb|fscanf|.
Mit der Ausname von \verb|%lf,%le|, die für \verb|fprintf| nicht zur Verfügung stehen.
\verb|%e| gibt eine Fließkommazahl auf \verb|double| Genauigkeit gerundet im wissenschaftlichen Format aus, \verb|%f| ebenfalls auf \verb|double| Genauigkeit gerundet als Dezimalzahl.
Weiterhin kann man die Anzahl von Stellen etc beeinflussen, zum Beispiel
\begin{lstlisting}
  #include<stdio.h>
  #include<stdlib.h>

  int main(){
    FILE *out;
    out = fopen("quadrats.txt", "w"); // Oeffne die Datei zum Schreiben
    if (out == NULL){
      fprintf(stderr, "Error opening the file\n");
      return(-1);
    }
    for (int i = 1; i <= 10; ++i) {
      fprintf(out, "%.6d\n", i*i);    // Schreibe i^2 in die Datei
    }
    if (fclose(out) != 0){            // Schliese die Datei wieder
      printf("Error in closing the File\n");
      return(-2);
    }
    return(0);
  }
\end{lstlisting}
Nach dem Aufruf enthält die Datei \verb|quadrats.txt| folgendes
\begin{verbatim}
000001
000004
000009
000016
000025
000036
000049
000064
000081
000100
\end{verbatim}
Die Spezifizierung \verb|%.6d| bedeutet, dass die ganze Zahl mit sechs Stellen ausgegeben werden soll.
Dementsprechend werden führende Nullen hinzugefügt.

Formatierte Ausgabe ist sinnvoll, wenn die Dateien für Menschen lesbar sein sollen und wenn nicht viele Daten gespeichert werden müssen.
Für den Fall von vielen Daten sollte man allerdings auf nichtformatierte Ausgabe zurückgreifen (manchmal auch als binäre Ausgabe bezeichnet).
Dies kann man in C beispielsweise mit den Funktione \texttt{fread} und \texttt{fwrite} bewerkstelligen:
\begin{lstlisting}
  size_t fread ( void * ptr, size_t size, size_t count, FILE * stream );
  size_t fwrite ( const void * ptr, size_t size, size_t count, FILE * stream );
\end{lstlisting}
Dabei werden \verb|count*size| bytes direkt gelesen und im Speicherbereich abgelegt, auf den \verb|ptr| zeigt, bzw. \verb|count*size| aus dem Speicherbereich direkt geschrieben, ohne, dass dazwischen eine Konvertierung und Interpretation stattfinden würde.
Beide Funktionen liefern die Anzahl der gelesen bzw. geschriebenen Elemente der Größe \verb|size| zurück.
Um den Erfolg der Operationen zu testen, sollte man also prüfen, ob der Rückgabewert dem Wert der Variablen \verb|count| entspricht.

Um mehrfach aus einer Datei lesen zu können, kann man mit 
\begin{lstlisting}
  void rewind(FILE *stream)
\end{lstlisting}
zum Anfang der Datei zurückkehren, oder mit
\begin{lstlisting}
  int fseek(FILE *stream, long int offset, int whence)
\end{lstlisting}
zu einem bestimmten Punkt in der Datei gehen.
Bei letzterer Funktion wir der \verb|offset| in Bytes relativ zu \verb|whence| gesetzt.
\verb|whence| kann auf \verb|SEEK_SET|, \verb|SEEK_CUR| or \verb|SEEK_END| gesetzt werden, was Anfang, momentane Position oder Ende in der Datei bedeutet.
Im folgenden ein Beispiel:
\begin{lstlisting}
  #include<stdio.h>
  #include<stdlib.h>

  int main(){
    FILE * outin;
    int * squares;
    const int n=10;
    // Oeffne die Datei
    outin = fopen("temporarystorage", "w+");
    if (outin == NULL){
      fprintf(stderr, "Error in opening the file\n");
      return(-1);
    }
    // Speicher fuer squares reservieren
    squares = (int *)malloc(sizeof(int)*n);
    if (squares == NULL){
      fprintf(stderr, "Error in allocating memory\n");
      return(-2);
    }
    // Berechne Quadrate und speichere sie in squares
    for (int i = 0; i < n; ++i) {
      squares[i]=(i+1)*(i+1);
    }
    // Schreibe squares in die Datei
    if ( fwrite((void*) squares, sizeof(int), n, outin) != n){
      printf("Error in writing\n");
      return(-3);
    }
    for (int i = 0; i < n; ++i) {
      squares[i]=0;
    }
    rewind(outin);    // Geh wieder zum Anfang der Datei
    // Lese die Datei wieder ein
    if ( fread ((void*) squares, sizeof(int), n, outin) != n){
      printf("Error in read\n");
      return(-4);
    }
    for (int i = 0; i < n; ++i) {
      printf("%d\n", squares[i]);
    }
    // Datei wieder schliessen
    if (fclose(outin) != 0){
      printf("Error in closing file\n");
      return(-5);
    }
    return(0);
  }   
\end{lstlisting}
Die Datei wird zum Schreiben und Lesen geöffnet.
Dabei wird eine eventuell existierende Datei gleichen Namens überschrieben.
Danach werden die Quadrate von $i=1,...,n$ zuerst in die Datei geschrieben.
Dann gehen wir wieder zum Anfang der Datei und lesen sie dann wieder ein.
Die eingelesenen Zahlen werden dann auf der Kommandozeile ausgegeben.
