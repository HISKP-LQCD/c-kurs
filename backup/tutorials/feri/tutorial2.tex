\documentclass{article}[12pt]
\usepackage[utf8]{inputenc}
\usepackage[T1]{fontenc}
\usepackage[ngerman]{babel}
\usepackage[dvipsnames]{xcolor}
\usepackage{lipsum}
\usepackage{algorithm}
\usepackage{algorithmicx}
\usepackage{algpseudocode}

\usepackage{amsfonts}
\usepackage[intlimits]{amsmath}
\usepackage{cite}
\usepackage{epsfig}

\usepackage[usenames,dvipsnames]{pstricks}
\usepackage{pstricks-add}
\usepackage{epsfig}
\usepackage{wasysym}
\usepackage{pst-grad} % For gradients
\usepackage{pst-plot} % For axes

\addtolength{\hoffset}{-1.5cm}
\addtolength{\textwidth}{3cm}
\usepackage{listings}
\usepackage{color}
\definecolor{mygreen}{rgb}{0,0.6,0}
\definecolor{mygray}{rgb}{0.5,0.5,0.5}
\definecolor{mymauve}{rgb}{0.58,0,0.82}
\PassOptionsToPackage{svgnames}{xcolor}
\usepackage{tcolorbox}
\usepackage{lipsum}
\tcbuselibrary{skins,breakable}
\usetikzlibrary{shadings,shadows}
\usepackage{siunitx}
\lstset{ %
  backgroundcolor=\color{white},   % choose the background color; you must add \usepackage{color} or \usepackage{xcolor}; should come as last argument
  basicstyle=\footnotesize,        % the size of the fonts that are used for the code
  breakatwhitespace=false,         % sets if automatic breaks should only happen at whitespace
  breaklines=true,                 % sets automatic line breaking
  captionpos=b,                    % sets the caption-position to bottom
  commentstyle=\color{mygreen},    % comment style
  deletekeywords={...},            % if you want to delete keywords from the given language
  escapeinside={\%*}{*)},          % if you want to add LaTeX within your code
  extendedchars=true,              % lets you use non-ASCII characters; for 8-bits encodings only, does not work with UTF-8
  frame=single,                    % adds a frame around the code
  keepspaces=true,                 % keeps spaces in text, useful for keeping indentation of code (possibly needs columns=flexible)
  keywordstyle=\color{blue},       % keyword style
  language=C,                      % the language of the code
  morekeywords={*,...},            % if you want to add more keywords to the set
  numbers=left,                    % where to put the line-numbers; possible values are (none, left, right)
  numbersep=5pt,                   % how far the line-numbers are from the code
  numberstyle=\tiny\color{mygray}, % the style that is used for the line-numbers
  rulecolor=\color{black},         % if not set, the frame-color may be changed on line-breaks within not-black text (e.g. comments (green here))
  showspaces=false,                % show spaces everywhere adding particular underscores; it overrides 'showstringspaces'
  showstringspaces=false,          % underline spaces within strings only
  showtabs=false,                  % show tabs within strings adding particular underscores
  stepnumber=1,                    % the step between two line-numbers. If it's 1, each line will be numbered
  stringstyle=\color{mymauve},     % string literal style
  tabsize=2,                       % sets default tabsize to 2 spaces
  title=\lstname                   % show the filename of files included with \lstinputlisting; also try caption instead of title
}

\usepackage{amssymb}

\newenvironment{mydefinitionblock}[1]{%
    \tcolorbox[beamer,%
    noparskip,breakable,
    colback=White,colframe=ForestGreen,%
    colbacklower=LimeGreen!75!White,%
    title=#1]}%
    {\endtcolorbox}

\newenvironment{myalertblock}[1]{%
    \tcolorbox[beamer,%
    noparskip,breakable,
    colback=White,colframe=Bittersweet,%
    colbacklower=Peach!75!White,%
    title=#1]}%
    {\endtcolorbox}

\newenvironment{myblock}[1]{%
    \tcolorbox[beamer,%
    noparskip,breakable,
    colback=White,colframe=RoyalBlue,%
    colbacklower=TealBlue!75!White,%
    title=#1]}%
    {\endtcolorbox}

\newenvironment{myexampleprogram}[1]{%
    \tcolorbox[beamer,%
    noparskip,breakable,
    colback=White,colframe=Goldenrod,%
    colbacklower=Yellow!75!White,%
    title=#1]}%
    {\endtcolorbox}
%--------
%\usepackage[magyar]{babel}
\title{Conway's Spiel des Lebens}
\begin{document}
\maketitle


\noindent Betrachten wir ein \glqq Universum\grqq, dass aus einem zweidimensionalen Gitter besteht.
Jede Gitterzelle kann zwei Zustände einnehmen:
\begin{itemize}
\item lebendig,
\item tot.
\end{itemize}
Im Spiel entwickelt sich das Universum nach bestimmten Regeln in diskreten Zeitschritten.
Der Satz von Regeln $R$ sieht wie folgt aus:
\begin{enumerate}
\item Eine lebendige Zelle stirbt, wenn sie weniger als zwei lebendige
  Nachbarzellen hat. (Einsamkeit)
  \begin{center}
\Huge
\begin{tabular} { l c r }
  \begin{tabular}{| l | c | r | }
    \hline
    {\color{blue}\frownie}   & {\color{blue}\frownie} & {\color{blue}\frownie} \\ \hline
    {\color{blue}\frownie}   & {\color{red}\smiley}   & {\color{blue}\frownie} \\ \hline
    {\color{black}\smiley}   & {\color{blue}\frownie} & {\color{blue}\frownie} \\ \hline
  \end{tabular}

& $\Longrightarrow$ &
  \begin{tabular}{| l | c | r | }
    \hline
    {\color{blue}\frownie} & {\color{blue}\frownie} & {\color{blue}\frownie} \\ \hline
    {\color{blue}\frownie} & {\color{red}\frownie}  & {\color{blue}\frownie} \\ \hline
    {\color{black}\smiley} & {\color{blue}\frownie} & {\color{blue}\frownie} \\ \hline
  \end{tabular}
\\
\end{tabular}
\end{center}
\item Eine lebendige Zelle mit zwei oder drei lebendigen Nachbarn lebt
  weiter. 
  \begin{center}
\Huge
\begin{tabular} { l c r }
  \begin{tabular}{| l | c | r | }
    \hline
    {\color{black}\smiley}   & {\color{blue}\frownie} & {\color{blue}\frownie} \\ \hline
    {\color{blue}\frownie}   & {\color{red}\smiley}   & {\color{blue}\frownie} \\ \hline
    {\color{black}\smiley}   & {\color{blue}\frownie} & {\color{blue}\frownie} \\ \hline
  \end{tabular}

& $\Longrightarrow$ &
  \begin{tabular}{| l | c | r | }
    \hline
    {\color{black}\smiley} & {\color{blue}\frownie} & {\color{blue}\frownie} \\ \hline
    {\color{blue}\frownie} & {\color{red}\smiley}   & {\color{blue}\frownie} \\ \hline
    {\color{black}\smiley} & {\color{blue}\frownie} & {\color{blue}\frownie} \\ \hline
  \end{tabular}
\\
\end{tabular}
\end{center}
\begin{center}
\Huge
\begin{tabular} { l c r }
  \begin{tabular}{| l | c | r | }
    \hline
    {\color{black}\smiley}   & {\color{blue}\frownie} & {\color{black}\smiley} \\ \hline
    {\color{blue}\frownie}   & {\color{red}\smiley}   & {\color{blue}\frownie} \\ \hline
    {\color{black}\smiley}   & {\color{blue}\frownie} & {\color{blue}\frownie} \\ \hline
  \end{tabular}

& $\Longrightarrow$ &
  \begin{tabular}{| l | c | r | }
    \hline
    {\color{black}\smiley} & {\color{blue}\frownie} & {\color{black}\smiley} \\ \hline
    {\color{blue}\frownie} & {\color{red}\smiley}   & {\color{blue}\frownie} \\ \hline
    {\color{black}\smiley} & {\color{blue}\frownie} & {\color{blue}\frownie} \\ \hline
  \end{tabular}
\\
\end{tabular}
\end{center}
\item Eine lebendige Zelle mit mehr als drei lebenden Nachbarzellen
  stirbt. (Überbevölkerung)
  \begin{center}
\Huge
\begin{tabular} { l c r }
  \begin{tabular}{| l | c | r | }
    \hline
    {\color{black}\smiley}   & {\color{blue}\frownie} & {\color{black}\smiley} \\ \hline
    {\color{black}\smiley}   & {\color{red}\smiley}   & {\color{blue}\frownie} \\ \hline
    {\color{black}\smiley}   & {\color{blue}\frownie} & {\color{blue}\frownie} \\ \hline
  \end{tabular}

& $\Longrightarrow$ &
  \begin{tabular}{| l | c | r | }
    \hline
    {\color{black}\smiley} & {\color{blue}\frownie} & {\color{black}\smiley} \\ \hline
    {\color{black}\smiley} & {\color{red}\frownie}   & {\color{blue}\frownie} \\ \hline
    {\color{black}\smiley} & {\color{blue}\frownie} & {\color{blue}\frownie} \\ \hline
  \end{tabular}
\\
\end{tabular}
\end{center}
\item Eine tote Zelle wird wiederbelebt, wenn sie genau drei lebende Nachbarzellen hat.
\begin{center}
\Huge
\begin{tabular} { l c r }
  \begin{tabular}{| l | c | r | }
    \hline
    {\color{blue}\frownie}   & {\color{black}\smiley} & {\color{blue}\frownie} \\ \hline
    {\color{black}\smiley}   & {\color{red}\frownie}   & {\color{black}\smiley} \\ \hline
    {\color{blue}\frownie}   & {\color{blue}\frownie} & {\color{blue}\frownie} \\ \hline
  \end{tabular}

& $\Longrightarrow$ &
  \begin{tabular}{| l | c | r | }
    \hline
    {\color{blue}\frownie} & {\color{black}\smiley} & {\color{blue}\frownie} \\ \hline
    {\color{black}\smiley} & {\color{red}\smiley}   & {\color{black}\smiley} \\ \hline
    {\color{blue}\frownie} & {\color{blue}\frownie} & {\color{blue}\frownie} \\ \hline
  \end{tabular}
\\
\end{tabular}
\end{center}
\end{enumerate}
Dabei wird davon ausgegangen, dass diese Regeln auf alle Zellen gleichzeitig angewendet werden.
Es gibt also einen Zustand des Universums $Z(\tau)$ zum Zeitpunkt $\tau$.
Dann wird jede Zelle nach obigen Regeln weiterentwickelt basierend auf dem Zustand $Z(\tau)$.
Wir können also auch schreiben, dass die Entwicklung nach den Regeln $R(Z(\tau))$ stattfindet. 
Der neue Zustand ist dann $Z(\tau+1)$. 
Der entsprechende Pseudocode könnte wie folgt aussehen
\begin{algorithmic}[1]
  \Procedure{Entwicklung}{$Z(\tau)$}
  \State \textbf{input}: $Z(\tau)$
  \State \textbf{output}: $Z(\tau+1)$
  \For{$z_i\in Z(\tau)$}
  \State $z_i(\tau+1)\ \stackrel{R(Z(\tau))}{\longleftarrow}\ z_i(\tau)$
  \EndFor
  \State \textbf{return} $Z(\tau+1)$
  \EndProcedure
\end{algorithmic}
Aufgabe ist ein Programm zu schreiben, das dieses Spiel implementiert.
Jede Zelle kann zwei Zustände haben, die wir auf $0$ (tot) und $1$ lebendig abbilden wollen.
Es gilt also $z \in \{0,1\}\ \forall z\in Z$.
Das zweidimensionale Gitter soll quadratisch sein und $L\times L$ Zellen $z$ haben.
D.h., insgesamt gibt es Zellen
\begin{equation}
  z(x,y; \tau) = \begin{cases}
    1 & \mathrm{Zelle~ist~lebendig} \\0  & \mathrm{Zelle~ist~tot} \\
  \end{cases}
  \,,\quad (x\in[0,1,\ldots L-1], y\in[0,1,\ldots L-1])\,.
\end{equation}
Obiger Pseudocode sieht damit wie folgt aus
\begin{algorithmic}[1]
  \Procedure{Entwicklung}{$Z(\tau)$}
  \State \textbf{input}: $Z(\tau)$
  \State \textbf{output}: $Z(\tau+1)$
  \For{$x\in 0,1,\ldots L-1$}
  \For{$y\in 0,1,\ldots L-1$}
  \State $z(x,y; \tau+1)\ \stackrel{R(Z(\tau))}{\longleftarrow}\ z(x,y; \tau)$
  \EndFor
  \EndFor
  \State \textbf{return} $Z(\tau+1)$
  \EndProcedure
\end{algorithmic}
Es bietet sich an, das Gitter im Programm als ein zweidimensionales Array darzustellen.
Der Datentyp des Arrays sollte es erlauben, die zwei Zustände darzustellen.
Die Größe des Arrays ist durch $L$ bestimmt.
Wir haben zwar noch nicht über zweidimensionale Arrays gesprochen, trotzdem können wir diese Aufgabe erledigen. 
Wir beschränken uns auch zunächst auf \emph{statische} arrays.
Ein zweidimensionale $L\times L$ Array kann durch folgende Abbildung $g$ auf ein eindimensionales Array der Größe $L^2$ abgebildet werden:
\begin{equation}
  g(x,y)\ =\ xL + y\,,\qquad (x\in[0,1,\ldots L-1], y\in[0,1,\ldots L-1])\,.
\end{equation}
Wir müssen noch sagen, wie wir am Rand des Gitters verfahren wollen.
Hier werden wir periodische Randbedingungen annehmen, also
\begin{equation}
  z(x+L,y)= z(x,y)\,,\quad (x\in[0,1,\ldots L-1], y\in[0,1,\ldots L-1])
\end{equation}
und
\begin{equation}
  z(x,y+L)= z(x, y)\,,\quad (x\in[0,1,\ldots L-1], y\in[0,1,\ldots L-1])\,.
\end{equation}
Im Quelltext benötigt man entsprechend zwei Arrays gleicher Größe für Zeitschritte $\tau$ und $\tau+1$.
Zum Beispiel
\begin{lstlisting}
  const unsigned int L = 12;     // feste Groesse des Gitters
  unsigned int Z1[L*L], Z2[L*L]; // statische Speicheranforderung
  unsigned int *Ztau = Z1;                // Zeiger fuer den Zugriff
  unsigned int *Ztaup1 = Z2;
\end{lstlisting}
Der Zugriff kann dann über die Zeiger \texttt{Z1} und \texttt{Z2} erfolgen.
So kann man $Z(\tau)$ und $Z(\tau+1)$ vertauschen, ohne die gesamten Inhalte kopieren zu müssen.

In zwei Dimensionen hat eine Zelle $z(x,y)$ acht Nachbarn:
\begin{itemize}
\item  $z(x-1,y)$
\item  $z(x-1,y-1)$
\item  $z(x-1,y+1)$
\item  $z(x,y-1)$
\item  $z(x,y+1)$
\item  $z(x+1,y)$
\item  $z(x+1,y-1)$
\item  $z(x+1,y+1)$
\end{itemize}
Für die Implementation müssen wir (mindestens) drei verschiedenen Funktionen schreiben:
\begin{enumerate}
\item Entwicklung: Gibt ein neues Gitter zurück, nachdem es nach den obigen Regeln entwickelt wurde
\item Ausgabe: Gibt das Gitter auf Monitor aus
\item Spiel: Initialisiert die Zellen zufällig und \glqq spielt\grqq, bis alle Zellen den Zustand tot angenommen haben oder bis eine obere Schranke für die Zeit erreicht ist.
\end{enumerate}
Für die zufällige Initialisierung der Zellen könne wir den Zufallszahlengenerator aus der vorigen Aufgabe verwenden.
Für jede Zelle erzeugt man einen Zufallszahl $u$.
Falls $u<\gamma$, wird die Zelle als tot initialisiert, sonst als lebendig.
Wählen Sie zu Beginn $\gamma=0,5$. 
Später können Sie untersuchen, wie die Überlebensrate der Population vom Wert von $\gamma$ abhängt.

%Beim Ausdrücken wir können die Folgenden Steuerzeichen als Argument vom \texttt{printf()} verwenden:
%\begin{enumerate}
%\item \textbackslash$e[1;1H$\textbackslash$e[2J$: Löschen den Bildschirm
%\item \textbackslash$033[H$: Positionierung des Cursers auf die links obene Ecke
%\item \textbackslash$033[E$; Beginn ein neue Zeile
%\item \textbackslash$033[107m$\textbackslash$033[34m$: Hintergrund Weiß, Zeichen in Blau
%\item \textbackslash$033[m$: Zurück zum Standard
%\end{enumerate}

\end{document}
