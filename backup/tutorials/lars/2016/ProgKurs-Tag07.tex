\documentclass{uebungszettel}
\begin{document}
\begin{aufg} 
Schau dir die Folien zu Vorlesung 7 nochmal an und implementiere doppelt verkettete Listen, die \verb|double|-Variablen speichern.

\medskip\begin{codelisting}
\begin{lstlisting}[numbers=left,numberstyle=\tiny,frame=tlrb]
/* Definiere hier angemessene Strukturen fuer einen
   einzelnen Listeneintrag und die Liste selbst. */

/* Leere Liste erstellen */
LIST *list_create();

/* Element hinter E einfuegen, NULL heisst am Anfang */
LISTNODE *list_insert(LIST *L, LISTNODE *E, double p);

/* Element am Anfang bzw. Ende einfuegen */
LISTNODE *list_unshift(LIST *L, double p);
LISTNODE *list_push(LIST *L, double p);

/* Element am Anfang bzw. Ende entfernen und 
   die Daten zurueck geben */
double list_shift(LIST *L);
double list_pop(LIST *L);

/* ein Element aus der Liste entfernen */
void list_delete(LIST *L, LISTNODE *E);

/* zwei Listen zusammenfuegen */
LIST *list_merge(LIST *L, LIST *M);

/* Liste inklusive allen Elementen frei geben */
void list_free(LIST *L);

\end{lstlisting}
\end{codelisting}
\end{aufg}

\newpage
\begin{aufg} Implementiere eine Funktion die zu einem gegebenen Funktionenpointer $f:\R \rightarrow \R$, einen Dateinamen, einer Schrittweite $s \in \R$, einer Startstelle $x_1$ und einer Endstelle $x_2$ die Wertetabelle der Funktion zwischen $x_1$ und $x_2$ zur Schrittweite $s$ speichert. Dabei sollen $x$ und $f(x)$ durch einen Tabulator getrennt werden und jedes Paar $(x, f(x))$ in einer eigenen Zeile stehen.
\end{aufg}

\begin{aufg}
In dieser Aufgabe geht es um numerische Integration.
\begin{enumerate}
\item Implementiere eine Integrationsfunktion, die das Intervall $[a, b]$ in $n$ gleich große Teile aufteilt, für diese jeweils die Trapezsumme (aus der Vorlesung) berechnet und diese aufsummiert:

\begin{codelisting}
\begin{lstlisting}[numbers=left,numberstyle=\tiny,frame=tlrb]
double integrate(double a, double b, 
  double (*f)(double), unsigned int n); 
\end{lstlisting}
\end{codelisting}

\item Schreibe nun eine Funktion, die nicht die Anzahl der Teilintervalle erhält, sondern eine ``Fehlertoleranz'' $e$. Die Funktion die Aufteilung solange verfeinern, bis sich der approximierte Wert für das Integral durch eine Verfeinerung nur noch um weniger als $e$ ändern würde. 
\end{enumerate}
\end{aufg}

\end{document}
