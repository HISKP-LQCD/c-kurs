\documentclass{uebungszettel}
\begin{document}

\begin{aufg}
In dem folgenden Hallo-Welt-Programm befinden sich
$4$ Fehler.
Finde sie alle.

\begin{codelisting}
\begin{lstlisting}[numbers=left,numberstyle=\tiny,frame=tlrb,showstringspaces=false]
 * Hello World Program.
 * (c) 2015 Clelia und Johannes */

#include <stdio.h>

double main () {
    pritnf ("Hallo Welt\n")
    return 0;
}
\end{lstlisting}
\end{codelisting}
\end{aufg}

\begin{aufg}
Installiere einen Compiler auf deinem Computer und kompiliere ein Hallo-Welt-Programm. Informationen dazu und einen Download-Link für Cygwin findest du im Skript auf dem USB-Stick oder der Webseite des Kurses:
\begin{center}
	\verb|http://www.ah-effect.net/ |
\end{center}
\end{aufg}

\begin{aufg}
Schreibe ein Programm, dass den Wert der folgenden Funktion ausgibt (für eine fest in den Quellcode geschriebene \verb|int|-Variable):
\[
	f(n) = \left\{ \begin{array}{ll}
	\frac{n}{2} & \text{wenn } n \text{ gerade} \\
	\frac{n+1}{2} & \text{wenn } n \text{ ungerade} \\
	\end{array}
	\right.
\]
Und das geht natürlich nur mit Wissen aus der Vorlesung.
\end{aufg}

\begin{aufg}
Was machen folgende Algorithmen (kein C-Code)?

\begin{algorithm}[H]
\caption{}
\algsetup{indent=1.5em}
\begin{algorithmic}[1]
\REQUIRE Ganze Zahl $c\in\mathbb{N}$
\ENSURE Entweder \verb|Ja| oder \verb|Nein|.
\STATE \SET $n := 2$.
\IF{$n>\sqrt{c}$} \label{1Start}
\RETURN \verb|Ja|
\ENDIF
\IF{$n$ teilt $c$}
\RETURN \verb|Nein|
\ENDIF
\STATE \SET $n := n + 1$
\STATE \GOTO \ref{1Start}
\end{algorithmic}
\end{algorithm}

\begin{algorithm}[H]
\caption{}
\algsetup{indent=1.5em}
\begin{algorithmic}[1]
\REQUIRE Ganze Zahlen $a,b\in\mathbb{N}$
\ENSURE Eine ganze Zahl $k\in\mathbb{N}$
\IF{$a=0$} 
\RETURN $b$
\ENDIF
\IF{$b=0$} \label{3Start}
\RETURN $a$
\ENDIF
\IF{$a>b$}
\STATE \SET $a = a - b$
\ELSE 
\STATE \SET $b = b - a$
\ENDIF
\STATE \GOTO \ref{3Start}
\end{algorithmic}
\end{algorithm}

\begin{algorithm}[H]
\caption{}
\algsetup{indent=1.5em}
\begin{algorithmic}[1]
\REQUIRE Reelle Zahl $a\in\mathbb{R}_{\ge 0}$
\ENSURE Reine reelle Zahl $x\in\mathbb{R}$
\STATE \SET $x := 2$ und $y := 1$.
\IF{$|x-y|\le 10^{-10}$} \label{2Start} 
\RETURN $x$
\ENDIF
\STATE \SET $x := y$
\STATE \SET $y := \frac{1}{2} \cdot \left(x+\frac{a}{x}\right)$
\STATE \GOTO \ref{2Start}
\end{algorithmic}
\end{algorithm}
\end{aufg}

\begin{aufg}
Wettbewerb: Gegeben ist folgender Programmrumpf:
\begin{codelisting}
\begin{lstlisting}[numbers=left,numberstyle=\tiny,frame=tlrb]
#include <stdio.h>
int main(int argc, char **argv) {
	int x = 2;
	/* dein Code hier */
	printf("%i\n", x);
	return 0;
}
\end{lstlisting}
\end{codelisting}

Füge an der markierten Stelle C-Code ein, sodass der Wert von
$2^{\left(3^3\right)}$ ausgegeben wird. Erlaubt sind aber nur folgende Zeichen:
\begin{center}
    \texttt{x \quad + \quad - \quad * \quad / \quad = \quad ( \quad ) \quad ;}
\end{center}
Zeilenumbrüche und Leerzeichen sollen genutzt werden, um das Programm möglichst
übersichtlich zu gestalten. Die Benutzung von Klammern kann ebenfalls
genutzt werden, um die Lesbarkeit zu erhöhen.
\end{aufg}


\end{document}
