\documentclass{uebungszettel}
\begin{document}

\begin{aufg} Implementiere eine Funktion die zu einem gegebenen Funktionenpointer $f:\R \rightarrow \R$, einem Dateinamen, einer Schrittweite $s \in \R$, einer Startstelle $x_1$ und einer Endstelle $x_2$ die Wertetabelle der Funktion zwischen $x_1$ und $x_2$ zur Schrittweite $s$ speichert. Dabei sollen $x$ und $f(x)$ durch einen Tabulator getrennt werden und jedes Paar $(x, f(x))$ in einer eigenen Zeile stehen. Etwa wäre die Ausgabe für $f=\cos$ zwischen $x_1=0$ und $x_2=0$ mit Schrittweite $s=0.1$ die folgende:
\begin{codelisting}
\begin{lstlisting}[numbers=left,numberstyle=\tiny,frame=tlrb]
0.0 1.0
0.1 0.995004165278
0.2 0.980066577841
0.3 0.955336489126
0.4 0.921060994003
0.5 0.87758256189
0.6 0.82533561491
0.7 0.764842187284
0.8 0.696706709347
0.9 0.621609968271
1.0 0.540302305868
\end{lstlisting}
\end{codelisting}
\end{aufg}

\begin{aufg}
In dieser Aufgabe geht es um numerische Integration.
\begin{enumerate}
\item Implementiere eine Integrationsfunktion, die das Intervall $[a, b]$ in $n$ gleich große Teile aufteilt, für diese jeweils die Trapezsumme (aus der Vorlesung) berechnet und diese aufsummiert:

\begin{codelisting}
\begin{lstlisting}[numbers=left,numberstyle=\tiny,frame=tlrb]
double integrate(double a, double b, 
  double (*f)(double), unsigned int n); 
\end{lstlisting}
\end{codelisting}

\item Schreibe nun eine Funktion, die nicht die Anzahl der Teilintervalle erhält, sondern eine ``Fehlertoleranz'' $e$. Die Funktion die Aufteilung solange verfeinern, bis sich der approximierte Wert für das Integral durch eine Verfeinerung nur noch um weniger als $e$ ändern würde. 
\end{enumerate}
\end{aufg}

\begin{aufg}
Implementiere folgende Funktion, die zu einem gegebenen String einen längsten Teilstring findet, der ein Palindrom ist. Speichere diesen Teilstring wieder in s und gib s zurück.
\begin{codelisting}
\begin{lstlisting}[numbers=left,numberstyle=\tiny,frame=tlrb]
/* Beschreibung, selber machen */
char *str_glsp(char *s);
\end{lstlisting}
\end{codelisting}
Das Problem ist in polynomieller Laufzeit lösbar und auf die Idee kann man auch kommen. 
\end{aufg}

\begin{aufg} Auf der Homepage gibt es ein Modul \verb|rationals|, in dem rationale Zahlen implementiert sind. Binde es in ein neues Projekt ein und deklariere ein Array von Brüchen, in etwa so:
\begin{codelisting}
\begin{lstlisting}[numbers=left,numberstyle=\tiny,frame=tlrb]
RATIONAL arr[7] = { 
 {-1,2}, {1,2}, {3,4}, {9,7}, {10,1}, {7,3}, {11,2} };
\end{lstlisting}
\end{codelisting}
Nun sortiere \verb|arr| mit Hilfe von \verb|qsort|.
\end{aufg}

\begin{aufg}
Brainfuck ist eine sogenannte esoterische Programmiersprache, das sind Sprachen, die meist zu wissenschaftlichen oder theoretischen Zwecken, oder einfach zum Spaß entwickelt wurden. 

Brainfuck besteht nur aus $8$ Befehlen: $>$ $<$ $+$ $-$ $,$ $.$ $[$ $]$ alle anderen Zeichen werden als Kommentar interpretiert. Diese Befehle werden, wie bei C auch, nacheinander ausgeführt. Sie operieren auf einem (potentiell unendlich langen) Band (welches aus Zellen besteht in denen jeweils ein \verb|char| steht) indem sie einen Lese-/Schreibkopf über das Band bewegen und Zeichen lesen / schreiben lassen. Das Band ist überall mit \verb|'\0'| vorinitialisiert und der Lese-/Schreibkopf startet an ``Position $0$'' des Bandes. Die Befehle haben folgendee Bedeutung:

\begin{tabular}{|c|p{10cm}|} \hline
$>$ bzw. $<$ & schiebt den Lese-/Schreibkopf eins nach rechts bzw. links \\\hline
$+$ bzw. $-$ & in- bzw. dekrementiert den Bandwert unter dem Lese-/Schreibkopf um $1$ \\\hline
$.$ & gibt den Wert unter dem Lese-/Schreibkopf aus \\\hline
$,$ & liest ein Zeichen vom Benutzer ein und schreibt es unter den Lese-/Schreibkopf \\\hline
$[$ & springt zum zugehörigen $]$-Befehl, wenn der Wert unter dem Lese-/Schreibkopf $0$ ist, sonst soll nichts passieren\\\hline
$]$ & springt zum zugehörigen $[$-Befehl, wenn der Wert unter dem Lese-/Schreibkopf verschieden von $0$ ist\\\hline
\end{tabular}

\newpage
So sieht ein ``Hallo-Welt''-Programm in Brainfuck aus:

\begin{codelisting}
\begin{lstlisting}[numbers=left,numberstyle=\tiny,frame=tlrb,mathescape=true]
++++++++++
[
   >+++++++>++++++++++>+++>+<<<<-
]
>++.
>+.
+++++++..
+++.>++.
<<+++++++++++++++.
>.+++.
------.--------.
>+.>.
\end{lstlisting}
\end{codelisting}
Deine Aufgabe ist es nun, ein Programm zu schreiben, welches Brainfuck-Programme einlesen und ausführen kann. 
\end{aufg}

\end{document}
