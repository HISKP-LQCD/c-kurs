\documentclass{article}[12pt]
\usepackage[utf8]{inputenc}
\usepackage[T1]{fontenc}
\usepackage[ngerman]{babel}
\usepackage[dvipsnames]{xcolor}
\usepackage{lipsum}

\usepackage{amsfonts}
\usepackage[intlimits]{amsmath}
\usepackage{cite}
\usepackage{epsfig}

\usepackage[usenames,dvipsnames]{pstricks}
\usepackage{pstricks-add}
\usepackage{epsfig}
\usepackage{pst-grad} % For gradients
\usepackage{pst-plot} % For axes

\addtolength{\hoffset}{-1.5cm}
\addtolength{\textwidth}{3cm}
\usepackage{listings}
\usepackage{color}
\definecolor{mygreen}{rgb}{0,0.6,0}
\definecolor{mygray}{rgb}{0.5,0.5,0.5}
\definecolor{mymauve}{rgb}{0.58,0,0.82}
\PassOptionsToPackage{svgnames}{xcolor}
\usepackage{tcolorbox}
\usepackage{lipsum}
\tcbuselibrary{skins,breakable}
\usetikzlibrary{shadings,shadows}
\usepackage{siunitx}
\lstset{ %
  backgroundcolor=\color{white},   % choose the background color; you must add \usepackage{color} or \usepackage{xcolor}; should come as last argument
  basicstyle=\footnotesize,        % the size of the fonts that are used for the code
  breakatwhitespace=false,         % sets if automatic breaks should only happen at whitespace
  breaklines=true,                 % sets automatic line breaking
  captionpos=b,                    % sets the caption-position to bottom
  commentstyle=\color{mygreen},    % comment style
  deletekeywords={...},            % if you want to delete keywords from the given language
  escapeinside={\%*}{*)},          % if you want to add LaTeX within your code
  extendedchars=true,              % lets you use non-ASCII characters; for 8-bits encodings only, does not work with UTF-8
  frame=single,                    % adds a frame around the code
  keepspaces=true,                 % keeps spaces in text, useful for keeping indentation of code (possibly needs columns=flexible)
  keywordstyle=\color{blue},       % keyword style
  language=C,                      % the language of the code
  morekeywords={*,...},            % if you want to add more keywords to the set
  numbers=left,                    % where to put the line-numbers; possible values are (none, left, right)
  numbersep=5pt,                   % how far the line-numbers are from the code
  numberstyle=\tiny\color{mygray}, % the style that is used for the line-numbers
  rulecolor=\color{black},         % if not set, the frame-color may be changed on line-breaks within not-black text (e.g. comments (green here))
  showspaces=false,                % show spaces everywhere adding particular underscores; it overrides 'showstringspaces'
  showstringspaces=false,          % underline spaces within strings only
  showtabs=false,                  % show tabs within strings adding particular underscores
  stepnumber=1,                    % the step between two line-numbers. If it's 1, each line will be numbered
  stringstyle=\color{mymauve},     % string literal style
  tabsize=2,                       % sets default tabsize to 2 spaces
  title=\lstname                   % show the filename of files included with \lstinputlisting; also try caption instead of title
}

\usepackage{amssymb}

\newenvironment{codelisting}{\fontfamily{pcr}\selectfont%
\lstset{commentstyle=\textit}\lstset{language=c}}{\fontfamily{ptm}\selectfont}


\usepackage{algorithm,algorithmic}

\floatname{algorithm}{Algorithmus}
\newcommand{\SET}{\textbf{set}\ }
\newcommand{\CHOOSE}{\textbf{choose}\ }
\newcommand{\GOTO}{\textbf{goto}\ }
\renewcommand{\algorithmicrequire}{\textbf{Input:}}
\renewcommand{\algorithmicensure}{\textbf{Output:}}
\renewcommand{\listalgorithmname}{Algorithms}
\renewcommand{\algorithmiccomment}[1]{\\/* #1 */}

\newenvironment{mydefinitionblock}[1]{%
    \tcolorbox[beamer,%
    noparskip,breakable,
    colback=White,colframe=ForestGreen,%
    colbacklower=LimeGreen!75!White,%
    title=#1]}%
    {\endtcolorbox}

\newenvironment{myalertblock}[1]{%
    \tcolorbox[beamer,%
    noparskip,breakable,
    colback=White,colframe=Bittersweet,%
    colbacklower=Peach!75!White,%
    title=#1]}%
    {\endtcolorbox}

\newenvironment{myblock}[1]{%
    \tcolorbox[beamer,%
    noparskip,breakable,
    colback=White,colframe=RoyalBlue,%
    colbacklower=TealBlue!75!White,%
    title=#1]}%
    {\endtcolorbox}

\newenvironment{myexampleprogram}[1]{%
    \tcolorbox[beamer,%
    noparskip,breakable,
    colback=White,colframe=Goldenrod,%
    colbacklower=Yellow!75!White,%
    title=#1]}%
    {\endtcolorbox}
%--------
%\usepackage[magyar]{babel}
\title{Solutions to Lars's Übung $2$}
\begin{document}
\maketitle
\section{Aufgabe 1}
\begin{codelisting}
\begin{lstlisting}[numbers=left,numberstyle=\tiny,frame=tlrb,showstringspaces=false]
/* Summe der ersten n Zahlen.
 * (c) 2015 Clelia und Johannes */

#include <stdio.h>

int main () {
    int n = 10;      /* Addiere bis zu dieser Zahl */
    int i;
    int summe;       /* speichert Zwischenergebnis */   
    
    i = 0;
    
    while (i < n) {
      summe =+ i     /* addiere ite Zahl auf summe */
    }    
    printf ("Das Ergebnis ist %f.\n", summe);
    return 0;
}
\end{lstlisting}
\end{codelisting}
First of all we have to check whether the actual implementations compile. Using \texttt{gcc} we get the following 
error messages:
\begin{lstlisting}
temp2.c: In function 'main':
temp2.c:15:5: error: expected ';' before '}' token
     }
     ^
temp2.c:16:13: warning: format '%f' expects argument of type 'double', but argument 2 has type 'int' [-Wformat=]
     printf ("Das Ergebnis ist %f.\n"
\end{lstlisting}
The first error message misses the end of the instruction at the 15-th line. The second message is only a warning.
It warns us that we print with \texttt{\%f} a variable of \texttt{int} typ. This should be \texttt{\%d}.
The source code, that can be successfully compiled look like the following:
\begin{codelisting}
\begin{lstlisting}[numbers=left,numberstyle=\tiny,frame=tlrb,showstringspaces=false]
/* Summe der ersten n Zahlen.
 * (c) 2015 Clelia und Johannes */

#include <stdio.h>

int main () {
    int n = 10;      /* Addiere bis zu dieser Zahl */
    int i;
    int summe;       /* speichert Zwischenergebnis */

    i = 0;

    while (i < n) {
      summe =+ i;     /* addiere ite Zahl auf summe */
    }
    printf ("Das Ergebnis ist %d.\n", summe);
    return 0;
}
\end{lstlisting}
\end{codelisting}
However, even if the code compiles it does not guarantee a meaningfull output. Running the code, it seems it runs
forever. This is because the statements in \texttt{while} actually never became falls, since \texttt{i} is not incremented
inside \texttt{while}.This should be done with \texttt{i++;}.Besides this, we have to initialize all variables correctly. 
This can be done together with the declaration. Comitting the above changes our code will give meaningfull results.
\begin{codelisting}
\begin{lstlisting}[numbers=left,numberstyle=\tiny,frame=tlrb,showstringspaces=false]
/* Summe der ersten n Zahlen.
 * (c) 2015 Clelia und Johannes */

#include <stdio.h>

int main () {
    int n = 10;      /* Addiere bis zu dieser Zahl */
    int i=0;         /* initializierung */
    int summe=0;      /* speichert Zwischenergebnis */

    while (i < n) {
      summe =+ i;     /* addiere ite Zahl auf summe */
      i++;
    }
    printf ("Das Ergebnis ist %d.\n", summe);
    return 0;
}
\end{lstlisting}
\end{codelisting}
Although our code now gives meaningfull results, this does not mean, that it gives back the corrct results. First of all
at the twelwth line we assign \texttt{i} to the \texttt{summe} variable, instead of incrementing it with \texttt{i}. Second
in the sum we also have to the into account \texttt{n} itself, so the statements in the \texttt{while} should be less or equal
to \texttt{n}.
\begin{codelisting}
\begin{lstlisting}[numbers=left,numberstyle=\tiny,frame=tlrb,showstringspaces=false]
/* Summe der ersten n Zahlen.
 * (c) 2015 Clelia und Johannes */

#include <stdio.h>

int main () {
    int n = 10;      /* Addiere bis zu dieser Zahl */
    int i=0;         /* initializierung */
    int summe=0;     /* speichert Zwischenergebnis */

    while (i <= n) {
      summe += i;     /* addiere ite Zahl auf summe */
      i++;
    }
    printf ("Das Ergebnis ist %d.\n", summe);
    return 0;
}
\end{lstlisting}
\end{codelisting}
This can be of course faster, using the well-known closed formula for the summ of the first \texttt{n} numbers.
\begin{codelisting}
\begin{lstlisting}[numbers=left,numberstyle=\tiny,frame=tlrb,showstringspaces=false]
/* Summe der ersten n Zahlen.
 * (c) 2015 Clelia und Johannes */

#include <stdio.h>

int main () {
    int n = 10;      /* Addiere bis zu dieser Zahl */
    int i=0;         /* initializierung */
    int summe=0;     /* speichert Zwischenergebnis */

    summe=n*(n+1)/2;
    printf ("Das Ergebnis ist %d.\n", summe);
    return 0;
}
\end{lstlisting}
\end{codelisting}
\section{Aufgabe 2}
\begin{lstlisting}
#include<stdio.h>
#include<math.h>//should be included (sqrt), compiled with -lm
int primetest(int c){
   int n=2;
   while  ( ((double)n<sqrt((double)c)) && (c%n != 0)){
//until we have not reach one divisor of c, or its square root
//we increase n
      n++;
   }
//if we have not found any divisor, then it is a prime
   if (c%n != 0)
     return 1;
//if we have then it is not, we give back 0
   else
     return 0;
}
//looking for the first prime after n
int firstprimeaftern(int n){
   int c=n;
//start a loop at n until the number gets a prime
   while( primetest(c) == 0 ){
//if it is not a prime yet, increase it by one
     ++c;
   }
   printf("First prime number after %d is %d\n", n, c);
   return c;
}
int main(){
   int m=31;
   int testres=primetest(m);
   int a,b,c;
   a=firstprimeaftern(20000);
   b=firstprimeaftern(30000);
   c=firstprimeaftern(40000);
   printf("%d\n", testres);
   printf("Solution b1 %d\n",a);
   printf("Solution b2 %d\n",b);
   printf("Solution b3 %d\n",c);
}
\end{lstlisting}
\section{Aufgabe 3}
\begin{lstlisting}
//Implementing the square root
#include<stdio.h>
#include<math.h>//should be included because fabs
double sqrtown( double a){
  double x=2;
  double fx=1;
//looking for the fix point of the mapping f:0.5*(x+a/x)
//fabs(x-fx) = | x-fx | : measure how far we are from the fixpoint
  while ( fabs(x-fx)>1e-10){
     x=fx;
     fx=0.5*(x+a/x);
  }
//When we habe found it, we just print it out and give it back
  printf("Root is %f\n", x);
  return x;
}
int main(){
  double x=8.;
  double root=sqrtown(x);
  printf("Root of %e is %e\n", x, root);
}
\end{lstlisting}
\section{Aufgabe 4}
\begin{lstlisting}
#include<stdio.h>
int gcr( int a, int b){
//do a loop "forever"
   while (1){
//if one of the numbers get zero: we end the loop and 
//return the other one: that will be the Greatest common factor.
     if (a==0) return b;
     if (b==0) return a;
//if not we make a new iteration with b, a%b, when a>b
//                            or with a, b%a in the other case
     if (a>b)
       a= a%b;
     else
       b= b%a;
   }
}
int main(){
   int a=15;
   int b=20;
   int fn=gcr(a,b);
   printf("n=%d\n",fn);
}
\end{lstlisting}
\section{Aufgabe 5}
You can find it in the script.
\section{Aufgabe 6}
\begin{lstlisting}
//we calculate an approximation to the summ of the 1/k**2 infinite series
#include<stdio.h>
#include<math.h>
int main(){
  int k=1;//we summ from 1
  double sum=0.;//very important initialization
  //we summ the one over squares till the 
  //successive elements differ larger
  //then 10**-8
  while( fabs(1./(k+1)-1./(double)k)>1e-8){
    sum+=1./(double)(k*k);
    ++k;
  }
  printf("%e\n",sum);
}
\end{lstlisting}
\end{document}
