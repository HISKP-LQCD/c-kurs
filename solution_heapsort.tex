\documentclass{article}[12pt]
\usepackage[utf8]{inputenc}
\usepackage[T1]{fontenc}
\usepackage[ngerman]{babel}
\usepackage[dvipsnames]{xcolor}
\usepackage{lipsum}

\usepackage{amsfonts}
\usepackage[intlimits]{amsmath}
\usepackage{cite}
\usepackage{epsfig}

\usepackage[usenames,dvipsnames]{pstricks}
\usepackage{pstricks-add}
\usepackage{epsfig}
\usepackage{pst-grad} % For gradients
\usepackage{pst-plot} % For axes

\addtolength{\hoffset}{-1.5cm}
\addtolength{\textwidth}{3cm}
\usepackage{listings}
\usepackage{color}
\definecolor{mygreen}{rgb}{0,0.6,0}
\definecolor{mygray}{rgb}{0.5,0.5,0.5}
\definecolor{mymauve}{rgb}{0.58,0,0.82}
\PassOptionsToPackage{svgnames}{xcolor}
\usepackage{tcolorbox}
\usepackage{lipsum}
\tcbuselibrary{skins,breakable}
\usetikzlibrary{shadings,shadows}
\usepackage{siunitx}
\lstset{ %
  backgroundcolor=\color{white},   % choose the background color; you must add \usepackage{color} or \usepackage{xcolor}; should come as last argument
  basicstyle=\footnotesize,        % the size of the fonts that are used for the code
  breakatwhitespace=false,         % sets if automatic breaks should only happen at whitespace
  breaklines=true,                 % sets automatic line breaking
  captionpos=b,                    % sets the caption-position to bottom
  commentstyle=\color{mygreen},    % comment style
  deletekeywords={...},            % if you want to delete keywords from the given language
  escapeinside={\%*}{*)},          % if you want to add LaTeX within your code
  extendedchars=true,              % lets you use non-ASCII characters; for 8-bits encodings only, does not work with UTF-8
  frame=single,                    % adds a frame around the code
  keepspaces=true,                 % keeps spaces in text, useful for keeping indentation of code (possibly needs columns=flexible)
  keywordstyle=\color{blue},       % keyword style
  language=C,                      % the language of the code
  morekeywords={*,...},            % if you want to add more keywords to the set
  numbers=left,                    % where to put the line-numbers; possible values are (none, left, right)
  numbersep=5pt,                   % how far the line-numbers are from the code
  numberstyle=\tiny\color{mygray}, % the style that is used for the line-numbers
  rulecolor=\color{black},         % if not set, the frame-color may be changed on line-breaks within not-black text (e.g. comments (green here))
  showspaces=false,                % show spaces everywhere adding particular underscores; it overrides 'showstringspaces'
  showstringspaces=false,          % underline spaces within strings only
  showtabs=false,                  % show tabs within strings adding particular underscores
  stepnumber=1,                    % the step between two line-numbers. If it's 1, each line will be numbered
  stringstyle=\color{mymauve},     % string literal style
  tabsize=2,                       % sets default tabsize to 2 spaces
  title=\lstname                   % show the filename of files included with \lstinputlisting; also try caption instead of title
}

\usepackage{amssymb}

\newenvironment{codelisting}{\fontfamily{pcr}\selectfont%
\lstset{commentstyle=\textit}\lstset{language=c}}{\fontfamily{ptm}\selectfont}


\usepackage{algorithm,algorithmic}

\floatname{algorithm}{Algorithmus}
\newcommand{\SET}{\textbf{set}\ }
\newcommand{\CHOOSE}{\textbf{choose}\ }
\newcommand{\GOTO}{\textbf{goto}\ }
\renewcommand{\algorithmicrequire}{\textbf{Input:}}
\renewcommand{\algorithmicensure}{\textbf{Output:}}
\renewcommand{\listalgorithmname}{Algorithms}
\renewcommand{\algorithmiccomment}[1]{\\/* #1 */}

\newenvironment{myexampleblock}[1]{%
    \tcolorbox[beamer,%
    noparskip,breakable,
    colback=White,colframe=ForestGreen,%
    colbacklower=LimeGreen!75!White,%
    title=#1]}%
    {\endtcolorbox}

\newenvironment{myalertblock}[1]{%
    \tcolorbox[beamer,%
    noparskip,breakable,
    colback=White,colframe=Bittersweet,%
    colbacklower=Peach!75!White,%
    title=#1]}%
    {\endtcolorbox}

\newenvironment{myblock}[1]{%
    \tcolorbox[beamer,%
    noparskip,breakable,
    colback=White,colframe=RoyalBlue,%
    colbacklower=TealBlue!75!White,%
    title=#1]}%
    {\endtcolorbox}

\newenvironment{myexampleprogram}[1]{%
    \tcolorbox[beamer,%
    noparskip,breakable,
    colback=White,colframe=Goldenrod,%
    colbacklower=Yellow!75!White,%
    title=#1]}%
    {\endtcolorbox}
%--------
%\usepackage[magyar]{babel}
\title{Solutions to Heapsort with pointers}
\author{Ferenc Pittler}
\begin{document}
\maketitle
We implement the heap sort algorithm using a balanced binary tree.
Every heap node has three pointers:
\begin{enumerate}
\item \texttt{parent}: points to the node's parent
\item \texttt{left}: points to it's right child
\item \texttt{right}: points to its left child
\end{enumerate}
The definition of the heap is actually quite simple (the realization 
is not as simple as it sounds): it is a balanced binary tree
with the a relationship between a child and it's parent. 
For example if the nodes contains some numerical value,
the relationship can be, that the value from the parents
is always larger as the value from its child.

Actually this implementation of the binary tree is quite similar to 
that of the double linked list. Therefore here we also use a specific 
typ, that points to the first and the last element of our binary tree.
To summarize the types we are using are the followings in \texttt{c}:
\begin{lstlisting}
typedef struct heapnode {
   int element;
   struct heapnode *left;
   struct heapnode *right;
   struct heapnode *parent;
} heapnode;
typedef struct binaryheap {
   struct heapnode *last;
   struct heapnode *first;
} binaryheap;
\end{lstlisting}
First we have to create an empty tree.
For this we have to allocate memoryspace for a binaryheap.
Then set its first and last pointers to \texttt{NULL}. This
will be actually an empty binary tree.
\begin{lstlisting}
binaryheap *createleertree(void){
   //erzeugt ein neues binaere Baum
   binaryheap *ret=(binaryheap *)malloc(sizeof(binaryheap));
   if (ret == NULL){
     printf("Memory reservierung fuer leer binary heap war nicht erfolgreich\n");
     exit(1);
   }
   //Unsere leere Baum
   ret->last=NULL;
   ret->first=NULL;
   return ret;
}
\end{lstlisting}
Having created a binary tree, we can start to fill it with nodes.
The way we fill it actually is using a two dimensional array of heapnodes
as helpers. We have now seen explicitely how a heap can be implemented by arrays, so we put the elements
in arrays and set the connection between them  (the parent, left, right)
pointers with the help of the array index. For example
the left of the node with arrayindex \texttt{i} has to be the node
with  arrayindex \texttt{2i+1}. We have to be careful, that for even
number of elements the last parent does not have a right child!

\begin{lstlisting}
void createheapnode(binaryheap **tree, heapnode **root, int *list, int n){
   int i;
//Zeiger array fuer die heap elements
   heapnode **array=(heapnode **)malloc(sizeof(heapnode*)*n);
   if (array == NULL){
     printf("Memory reservierung fuer Zeigers auf heap elements war nicht erfolgreich\n");
     exit(1);
   }
   for (i=0; i<n; ++i){
     array[i]=(heapnode *)malloc(sizeof(heapnode));
     if (array[i] == NULL){
       printf("Nicht genug speicher fuer die %d element im heap\n", i);
       exit(1);
     }
   }
//erfuellt alle elementen mit den Werten
   for (i=0; i<n; ++i){
     array[i]->element=list[i];
     array[i]->left=NULL;
     array[i]->right=NULL;
     array[i]->parent=NULL;
   }
//machen die eltern kind beziehungen
//linkes Kind fuer i wird 2*i+1 sein
//rechtes kind fuer i wird 2*i+2 sein
//achten wenn n ist gerade wir haben am
//ende ein linkes Kind
   for (i=0; i<n/2;++i){
     array[i]->left=array[2*i+1];
     array[2*i+1]->parent=array[i];
     if (2*i+2 != n){
       array[i]->right=array[2*i+2];
       array[2*i+2]->parent=array[i];
     }
   }
   *root=array[0];
   (*tree)->first=array[0];
   (*tree)->last=array[n-1];
   free(array);
}
\end{lstlisting}
In this way we only have filled it with random elements, so our
we are not able to call our binary tree as a binary heap.
We must restore the heap property.
\begin{lstlisting}
void versickern(binaryheap **b, heapnode **q){
   int temp1;
   int temp2;
   int orig;
//wenn sie kein Kind hat wir sind fertig
   if ((*q)->left == NULL){
      return;
   }
//wenn sie ein Kind hat es muss link sein
//uberprufen wir die heapnode eigenschaft
//wenn es notig, tauschen wir die Wert
//mit dem linken Kind
   if ((*q)->right == NULL){
      if ( (*q)->element < (*q)->left->element ){
         swap(b,q,*q, (*q)->left);
      }
      return;
   }
//wenn es zwei Kinder hat
//wir mussen auch das heapnode eigenschaft uberpruefen
   temp1=(*q)->left->element;
   temp2=(*q)->right->element;
   orig=(*q)->element;
//wenn das root ist die groesste wir sind fertig
   if ((orig > temp1 ) && (orig > temp2))
     return;
//wenn nicht wir suchen fuer die groesste
//von rechten und linken Kind und tauschen
//aber wir sind nicht fertig, wir muessen
//auch fuer die rechtes beziehungseise
//der linkes Kind auch uberpruefen
   if ( temp1 > temp2) {
     swap(b,q,*q, (*q)->left);
     versickern(b,&((*q)->left));
     return;
   }
   swap(b,q,*q, (*q)->right);
   versickern(b,&((*q)->right));
   return;
}
const int MAX=100;
void heapify(binaryheap **b,heapnode **q){
   if ((*q)->left == NULL)
     return;
   else
     heapify(b,&((*q)->left));
   if ((*q)->right != NULL){
     heapify(b,&((*q)->right));
   }
   versickern(b,q);
}

void heapify(binaryheap **b,heapnode **q){
   if ((*q)->left == NULL)
     return;
   else
     heapify(b,&((*q)->left));
   if ((*q)->right != NULL){
     heapify(b,&((*q)->right));
   }
   versickern(b,q);
}
\end{lstlisting}
Here the most important routine is the swap, which
actually swap a parent with its child in the tree.
We have to set the following pointers (Figure FIXME).
\begin{lstlisting}
void swap(binaryheap **b,heapnode **root, heapnode * const parent, heapnode * const child) {
  heapnode *p = parent;
  heapnode *c = child;
  if (child == parent->left) {
     c->parent=p->parent;
     p->parent=c;
     p->left=c->left;
     c->left=p;
     if (p->right!=NULL){
      p->right->parent=c;
     }
     if (c->right!=NULL){
      c->right->parent=p;
     }
     swap_elem(&(p->right),&(c->right));
  }
  else if (child == parent->right) {
     c->parent=p->parent;
     p->parent=c;
     p->right=c->right;
     c->right=p;
     if (p->left!=NULL){
      p->left->parent=c;
     }
     if (c->left!=NULL){
      c->left->parent=p;
     }
     swap_elem(&(p->left),&(c->left));
  }
  else {
    printf("Cannot swap elements which are not direct reletives\n");
    abort();
  }
  if (parent == *root) {
    *root = child;
  }
  if ((*b)->first == parent){
    (*b)->first=child;
  }
  if ((*b)->last == child ){
    (*b)->last = parent;
  }
}
\end{lstlisting}
\section{Remove the first from the tree}
Here we free the first pointer and change the pointers in such a way that
the last element will be a new root of the tree. Of course, this will 
ruin the heap property, which we have to restore with  \texttt{versickern}. 
In this part we need a code to find actually the new last element of the tree. 
In order to get the new last element we implement the following trick:
\begin{enumerate}
\item If the original last element is a right child then it is easy.
The left child of its parent will be the new last element. 
\item If it is a left child we make the following trick.
We go up and store the actual path. In order to go to the new last element
we follow exactly the opposite path in the opposite order from the root till the 
left child of the actual node is not equal to zero.
\end{enumerate}
\begin{lstlisting}
void removelast(binaryheap **b,heapnode **q){
   heapnode *lastelem=(*b)->last;
   heapnode *firstelem=(*b)->first;
   if ( (*b)->first == NULL ){
     printf("Fehler es gibt kein element im Baum\n");
     exit(1);
   }
   if (firstelem == lastelem ){
    //nur ein element
    printf("%d\n", lastelem->element);
    //wir geben es frei
    free(lastelem);
    (*q)=NULL;
    (*b)->last = NULL;
    (*b)->first= NULL;
    return;
   }
   if (firstelem->left == lastelem){
    //gubt es nur zwei element
    printf("%d\n", firstelem->element);
    free(firstelem);
    //last elem parent zeiger to NULL
    lastelem->parent=NULL;
    (*b)->first=lastelem;
    (*q)=lastelem;
    return;
   }
   if (firstelem->right == lastelem){
    //nur drei element
    //nachdem tauschen wir ueberpruefen 
    //das heap eigenschaft
    printf("%d\n", firstelem->element);
    lastelem->left=firstelem->left;
    free(firstelem);
    lastelem->parent=NULL;
    (*b)->first=lastelem;
    (*b)->last=lastelem->left;
    lastelem->left->parent=lastelem;
    (*q)=lastelem;
    versickern(b,q);
    return;
   }
   if (lastelem->parent->left == lastelem){
      //spaicher das Weg zum root
      int *path=(int *)malloc(sizeof(int)*100);
      heapnode *temp2=lastelem->parent;
      printf("%d\n",firstelem->element);
      heapnode *temp=lastelem;
      heapnode *temp3;
      int i=0,j=0;
      if (path == NULL){
        printf("Memory allocation error in path\n");
        exit(1);
      }
      for (i=0; i<100; ++i)
        path[i]=0;
      i=0;
      while(temp->parent != NULL){
        temp3=temp->parent;
        if (temp3->right == temp){
          path[i]=1;
        }
        else{
          path[i]=0;
        }
        temp=temp3;
        i++;
      }
      //wir machen das gegenteil
      //vom Weg in der andere richtungs
      for(j=0; j<i;++j){
        if (path[i-j-1]==0){
          temp3=temp->right;
          if (temp3 == NULL)
            break;
          temp=temp3;
          }
         else{
          temp3=temp->left;
          if (temp3 == NULL)
           break;
          temp=temp3;
         }
      }
      lastelem->right=firstelem->right;
      lastelem->left=firstelem->left;
      firstelem->left->parent=lastelem;
      firstelem->right->parent=lastelem;
      free(firstelem);
      temp2->left=NULL;
      lastelem->parent=NULL;
      (*q)=lastelem;
      (*b)->first=lastelem;
      (*b)->last=temp;
      //wir muessen das heap eigenschaft ueberpruefen
      versickern(b,q);
      return;
   }
   else{
      printf("%d\n",firstelem->element);
      heapnode *temp2=lastelem->parent;
      heapnode *temp=lastelem->parent->left;
      lastelem->right=firstelem->right;
      lastelem->left=firstelem->left;
      firstelem->left->parent=lastelem;
      firstelem->right->parent=lastelem;
      free(firstelem);
      temp2->right=NULL;
      lastelem->parent=NULL;
      (*q)=lastelem;
      (*b)->first=lastelem;
      (*b)->last=temp;
      versickern(b,q);
   }

}
\end{lstlisting}
\section{main}
\begin{lstlisting}
int main(int argc, char *argv[]){
   int list[]={2,3,4,8,5,6,1,7};
   heapnode *Q;
   binaryheap *B;
   B=createleertree();
   createheapnode(&B,&Q,list,8);

   heapify(&B,&Q);
   removelast(&B,&Q);
   removelast(&B,&Q);
   removelast(&B,&Q);
   removelast(&B,&Q);
   removelast(&B,&Q);
   removelast(&B,&Q);
   removelast(&B,&Q);
   removelast(&B,&Q);

\end{lstlisting}
\end{document}
