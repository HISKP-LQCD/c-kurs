\section{Variablen}

Wie schon angedeutet werden Daten in C in Variablen bzw. Objekten gespeichert.
Auch dies ist am einfachsten an Hand eines Beispiels zu verstehen
\begin{lstlisting}{Erste Variablen Deklaration und Zuweisung}
#include<stido.h>

int main(){
  int n;
  n = 4;
  printf("Wir werden n zahlen sortieren:\n");
}
\end{lstlisting}
Im Vergleich zu unserem letzten Beispiel sind zwei Zeilen hinzugekommen. 
\begin{itemize}
\item \texttt{int n;}\\
  Diese Anweisung stellt eine Deklaration dar. 
  Sie teilt dem Compiler mit, ab jetzt den entsprechenden Speicherplatz für eine ganze Zahl vom Typ \texttt{int} bereitzustellen, also $4$ byte.
  Außerdem kennt der Compiler ab jetzt den Namen \texttt{n} innerhalb des Blockes, der durch \texttt{\{\}} begrenzt wird.
  Streng genommen findet gleichzeitig auch eine Definition statt, weil der entsprechende Speicherplatz reserviert wird. 
  Man unterscheidet Deklaration und Definition, weil es auch Deklarationen ohne Bereitstellung von Speicher gibt.
  In einem solchen Fall wird nur das Objekt bekannt gemacht.

\item \verb|n = 4;|\\
  Diese Anweisung stellt eine Zuweisung dar.
  Der Variable \texttt{n} wird der Wert $4$ zugewiesen.
  An der entsprechenden Stelle im Speicher wird dieser Wert abgelegt.
  Bei einer Definition führt C keine Initialisierung durch.
  Nach der reinen Definition (ohne Zuweisung) ist der Wert der Variablen \texttt{n} also rein zufällig.
\end{itemize}
Wie oben ersichtlich, haben Variablen einen Sichtbarkeitsbereicht und damit auch eine Lebensdauer.
Innerhalb eines Blockes kann man nicht zwei Variablen mit gleichem Namen deklarieren, dies führt zu einer Fehlermeldung des Compilers.
Deklariert man in einem Unterblock eine Variable mit dem Namen einer Variablen aus dem darüberliegenden Block, so ist die Variable aus dem darüberliegenden Block verdeckt.
In unserem Beispiel von oben heißt das, dass die Variable \texttt{n} nur in der Funktion \texttt{main} sichtbar ist. 
Es ist wichtig, dass jeder Benutzung einer Variablen, beispielsweise in einer Zuweisung, die Deklaration der Variablen vorangehen muss.
Deklaration und Zuweisung können auch in einer Anweisung geschehen, man kann also auch
\begin{lstlisting}{Erste Variablen Deklaration und Zuweisung}
#include<stido.h>

int main(){
  int n = 4;
  printf(``Wir werden n zahlen sortieren:\n'');
}
\end{lstlisting}
schreiben. 
Man spricht dann auch von Initialisierung der Variablen \texttt{n}.
Variablen, die außerhalb aller Funktionen deklariert werden, bezeichnet man als \emph{global}.
Sie sind in allen Funktionen, die nach der globalen Deklaration definiert werden, sichtbar und zugreifbar.

%\textbf{das ist zu früh hier!}
%Es gibt zwei verschiedene Möglichkeiten eine globale Variable zu definieren.
%\begin{enumerate}
%\item Stichwort static: In diesem Fall mann kann nutzten die Variable in ganzen File, wo es definiert war. Anderen
%Files in unserem Code kann natürlich nutzten eine andere Variable mit dem gleichen Name.
%\item Stichwort extern: In diesem Fall mann kann nutzten die Variable in dem ganzen Programm. Aber das
%variable muss deklariert werden in allen Files, wo wir ihn nutzten wollen.
%\end{enumerate} die für alle
%Wir haben die Sichtbarkeitbereich der Variablen 
%in der Abbildung  \ref{sicht} zusammen gefasst.  Das bedeutet, das wir können Variablen mit 
%gleichen Namen in verschiedenen Funkcionen nutzten, wenn wir definieren sie als lokalen Variablen. 
%
%% Generated with LaTeXDraw 2.0.8
%% Tue Feb 21 10:59:44 CET 2017
%% \usepackage[usenames,dvipsnames]{pstricks}
%% \usepackage{epsfig}
%% \usepackage{pst-grad} % For gradients
%% \usepackage{pst-plot} % For axes
%%\scalebox{0.5} % Change this value to rescale the drawing.
%%{
%\begin{figure}[!ht]
%\centering
%\scalebox{0.5}
%{
%\begin{pspicture}(2,-9.1)(17.3,9.12)
%\psframe[linewidth=0.04,dimen=outer](12.6,9.1)(2.0,-9.1)
%%\usefont{T1}{ptm}{m}{n}
%\rput(3.6692188,8.61){Source code}
%\psline[linewidth=0.04](5.2,9.1)(5.2,8.1)(2.0,8.1)
%%\usefont{T1}{ptm}{m}{it}
%\rput(7.8229685,-7.565){globalen Variablen: Deklarierten mit dem Stichwort: extern}
%\psframe[linewidth=0.04,dimen=outer](11.6,7.1)(3.2,0.1)
%\psframe[linewidth=0.04,dimen=outer](11.6,-0.5)(3.2,-7.1)
%%\usefont{T1}{ptm}{m}{n}
%\rput(4.0503125,6.81){File 1}
%%\usefont{T1}{ptm}{m}{n}
%\rput(4.0489063,-0.79){File 2}
%\psline[linewidth=0.04](5.0,7.1)(5.0,6.5)(3.2,6.5)(3.2,6.5)
%\psline[linewidth=0.04](4.8,-0.5)(4.8,-1.1)(3.2,-1.1)
%%\usefont{T1}{ptm}{m}{n}
%\rput(6.7009373,0.81){variable mit static Stichwort}
%%\usefont{T1}{ptm}{m}{n}
%\rput(6.9809375,-6.59){variablen mit static stichwort}
%\psframe[linewidth=0.04,dimen=outer](6.8,5.7)(3.8,2.7)
%\psframe[linewidth=0.04,dimen=outer](11.0,5.5)(8.0,2.7)
%\psframe[linewidth=0.04,dimen=outer](7.0,-1.7)(3.6,-5.1)
%\psframe[linewidth=0.04,dimen=outer](11.2,-1.9)(8.0,-5.3)
%%\usefont{T1}{ptm}{m}{n}
%\rput(4.9203124,5.41){Funkcion 1}
%%\usefont{T1}{ptm}{m}{n}
%\rput(9.3189063,5.21){Funkcion 2}
%%\usefont{T1}{ptm}{m}{n}
%\rput(4.9203124,-1.99){Funkcion 1}
%%\usefont{T1}{ptm}{m}{n}
%\rput(9.3189063,-2.19){Funkcion 2}
%%\usefont{T1}{ptm}{m}{n}
%\rput(4.7434375,4.61){Lokale }
%%\usefnt{T1}{ptm}{m}{n}
%\rput(5.4034376,4.21){variablen}
%%\usefont{T1}{ptm}{m}{n}
%\rput(9.3434377,4.41){Lokale}
%%\usefont{T1}{ptm}{m}{n}
%\rput(9.8173437,4.01){Variablen}
%%\usefont{T1}{ptm}{m}{n}
%\rput(4.7434375,-3.39){Lokale}
%%\usefont{T1}{ptm}{m}{n}
%\rput(5.2034376,-3.79){variablen}
%%\usefont{T1}{ptm}{m}{n}
%\rput(9.3434377,-3.39){Lokale}
%%\usefont{T1}{ptm}{m}{n}
%\rput(9.8034377,-3.79){variablen}
%\end{pspicture}
%}
%\caption{\label{sicht} Sichtbarkeitbereich der Variablen}
%\end{figure}

Variablen haben immer einerseits einen Datentyp und einen Wert. 
Der Datentyp entscheidet, welche Werte eine Variable annehmen kann und wie viel Arbeitsspeicher dafür reserviert wird.
In der Tabelle~\ref{tab:PPer} sind die elementaren C-Datentypen mit ihren Wertebereichen aufgelistet.

\begin{table}[t]
\caption{Elementare Datentypen\label{tabelle1}}  % title name of the table
\centering
  % centering table
\begin{tabular}{|l c c rrr|}
  % creating 10 columns
\hline
Name & & Varianten & Größe in Byte & Minimaler Wert & Maximaler Wert
  % inserting double-line Audio &Audibility & Decision & \multicolumn{7}{c}{Sum of Extracted Bits} 
\\[0.5ex]   
\hline % inserts single-line % Entering 1 st row
                       & & int &4 & $-2,147,483,648$ & $2,147,483,647$ \\[-0.0ex]
                       & & short & 2 & $-32,768$ & $32,767$ \\[-0.0ex]
\raisebox{1ex}{int}  & & unsigned short& 2 & $0$ & $65535$ \\[-0.0ex]
                       & &unsigned& 4 & $0$ & $ +4,294,967,295$ \\[1ex]
                       & &long& 4 &  $-2,147,483,648$ & $2,147,483,647$ \\
\hline
% Entering 2nd row
                            & &signed & 1 & $-128$ & $127$ \\[-1ex]
\raisebox{1.5ex}{char} &    & unsigned &1 & $0$ & $255$  \\[1ex]
\hline
% Entering 3rd row
float & & & 4 &  &  \\
double& & & 8 &  &  \\
long double& & &8 &  &  \\[1ex]

% [1ex] adds vertical space
\hline                          % inserts single-line
\end{tabular}
\label{tab:PPer}
\end{table}

C Compiler führen im Prinzip eine strenge Typenkontrolle durch.
Das ist eine sehr nützliche Eigenschaft der Compiler, wenn es auch manchmal etwas mühsam ist. 
Man kann dies durch einen expliziten \emph{cast} umgehen.
Dafür sollte man aber sehr genau wissen, was man tut.
Leider ist die Typkontrolle vom Compiler abhängig und meist wird bei einer Zuweisung ein impliziter \emph{cast} durchgeführt.
Beispielsweise wird folgender Code ohne Beanstandung übersetzt
\begin{lstlisting}{Erste Variablen Deklaration und Zuweisung}
  #include<stido.h>
  
  int main(){
    // so etwas sollte man nicht schreiben!
    int n = 4.5;
    printf(``Wir werden n Zahlen sortieren:\n'');
  }
\end{lstlisting}
obwohl hier implizit die reelle Zahl $4.5$ durch Abschneiden in eine ganze Zahl umgewandelt wird.
\texttt{n} hat den Wert $4$.

Es gibt einige Regeln für die Namen von Objecten in C. 
C eigene Schlüsselworte, wie z.B. \texttt{main} dürfen nicht verwendet werden.
Auch dürfen die Namen nicht mit einer Zahl beginnen, auch wenn Zahlen generell erlaubt sind.
Operatornamen, wie \verb|+| oder \verb|-| dürfen ebenfalls nicht verwendet werden.
Es ist ratsam, Variablen mit sinnvollen Namen zu versehen.
Das macht den Quelltext lesbarer und erhöht die Verständlichkeit.
Für den Algorithmus Einfügesortieren sollte man beispielsweise die beiden Liste mit \texttt{sortiert} und \texttt{unsortiert} benennen.
Im folgenden Quelltext sind einige Beispiele für richtige und falsche Variablendeklarationen zu finden:
\begin{lstlisting}
  int main(){
    int m1=4, n1=5, l1=6; // Richtig
    int m2=4, char n2='a', float m2=4. // Falsch
    char m3='a'; double n3=18.9; // Richtig
    float 4m=1.; // Falsch
    return(0);
  }
\end{lstlisting} 
Wie man sieht kann man mehrere Variable in einer Anweisung deklarieren, definieren und initialisieren, wenn sie den gleichen Typ haben.
Die Variablen werden dabei durch ein Komma getrennt.
Wie schon oben erwähnt, können mehrere Anweisung in der gleichen Zeile stehen, solange sie mit dem Semikolon abgeschlossen werden.

Variablen können mit Hilfe von Operationen manipuliert werden.
Natürlich hängt es vom Variablentyp ab, welche Operationen dafür definiert sind.
Man unterscheidet drei verschiedene Typen von Operationen:
\begin{itemize}
\item Infix:\\
  Derx Operator steht zwischen den Variablen. Zum Bespiel: \verb|a+b|. 
  Dieser Ausdruck nimmt die jeweiligen Werte von \verb|a| und \verb|b|, summiert sie und gibt das Ergebnis zurück.
\item Präfix:\\
  Der Operator steht vor der Variablen. Zum Bespiel: \verb|++a|. 
  Dieser Ausdruck erhöht den Wert von \verb|a| um $1$ und gibt danach den neuen Wert von \verb|a| zurück.
\item Postfix:\\
  Der Operator steht nach der Variablen. Zum Beispiel: \verb|a--|. 
  Dieser Ausdruck reduziert den Wert von \verb|a| um $1$, aber gibt den originalen Wert vom \verb|a| zurück.
\end{itemize}
Wieder sieht man es am einfachsten an einem Beispiel:
\begin{lstlisting}
#include<stdio.h>

int main(){
  int a = 2;
  printf("%d\n", a++);
  printf("%d\n", a);
  printf("%d\n", ++a);
  printf("%d\n", a);
  return(0);  
}
\end{lstlisting}
%% stopped here...
Im obenen Beispiel zuerst wir definieren eine Variable mit dem Anfangswert zwei. Danach drücken wir die wirkung des Postfix operators ($++$) auf $a$ aus 
in der vierten Reihe. Das Ergebnis wird auch zwei sein, weil wir das Postfix operator genutzten haben. Aber wenn wir gleich danach im fünften Reihe
das Wert von $a$ ausdrücken, das Ergebnis wird drei sein. In der sechsten Reihe wir drücken die wirkung des Präfix operators ($++$) auf $a$ aus, und 
das Ergebnis wird vier sein, und als Nebeneffect, das Variable $a$ wird inkrementiert.

Wir haben also drei Vershieden Operator Typen:
\begin{itemize}
\item binärer: Das operator hat zwei Argumente
\item unärer: Das operator hat nur ein Argumente
\item ternärer; Nur eine, der drei Argumente hat: $?:$
\end{itemize} 

Die operatoren könnten bitwise, oder logicalische sein. Die bitwise operator sind binäre operatoren. Sie ausführen die operation mit jedem bits der Arguments.
Die Logical operatoren geben zurück Logical Wert: nein, oder falsh. In dem unteren Tabelle wir fassen zusammen die Wichtige operatoren der Sprache.

\begin{table}
\caption{Arithmetic operatoren \label{oper}}
\centering
\begin{tabular}{|l c c|}
\hline
Operator & Expression & Wert der Expression \\
\hline
Zuweisung & $a$ = $b$ & Werte von $b$ \\
Addition & $a$ + $b$ & Summe von $a$ und $b$ \\
Subraktion & $a$ - $b$ & Differenz von $a$ und $b$ \\
Multiplikation & $a$ * $b$ & Produkt von $a$ und $b$ \\
Division & $a$ / $b$ & Quotient von $a$ und $b$ \\
Modulo & $a$ \% $b$ & Rest eine Ganzzahldivision von $a$ durch $b$ \\
Inckrement & $++a$,$a++$ & Präfix: $a$+1, Postfix: $a$ \\
Dekrement & $--a$, $a--$ & Präfix: $a$-1, Postfix: $a$ \\
Positiver Vorzeichenoperator & $+a$ & Wert von $a$ \\
Negativer Vorzeichenoperator & $-a$ & Wert von $a$ aber mit umgekehrte Vorzeichen \\
\hline
\end{tabular}
\end{table}

\begin{table}
\caption{Vergleichs operatoren \label{vergoper}}
\centering
\begin{tabular}{|l c|}
\hline
Operator & Expression \\
\hline
Prüfen auf Gleichheit & $a == b$  \\
Prüfen auf Ungleichheit & $a != b$ \\
Prüfen, ob $a$ echt größer als $b$ ist & $a>b$ \\
Prüfen, ob $a$ echt kleiner als $b$ ist & $a<b$ \\
Prüfen, ob $a$ größer oder gleich $b$ ist & $a>=b$ \\
Prüfen, ob $a$ kleiner oder gleich $b$ ist & $a<=b$ \\
\hline
\end{tabular}
\end{table}

\begin{table}
\caption{Logischen operatoren \label{vergoper}}
\centering
\begin{tabular}{|l c c|}
\hline
Operator & Expression & Wert \\
\hline
                                                &                                &   Wenn $a$ und $b$ beide waren wahr, \\
                                                &                                &   dann der Rückgabe ist wahr,  \\
\raisebox{1.5ex}{Operator für das Logische UND} & \raisebox{1.5ex}{$a \&\& b$ }  &   in jeden anderen Fallen falsch \\
\hline
                                                &                                &   Wenn $a$ oder $b$ war wahr, \\
                                                &                                &   dann der Rückgabe ist wahr, \\
\raisebox{1.5ex}{Opatoren für das logische ODER}& \raisebox{1.5ex}{$a ||    b$}  &   in den anderen Fall  falsch \\
\hline
                                                &                                &   Wenn $a$ ist falsch, die Rückgabe is wahr, \\
\raisebox{1.5ex}{Negationsopeator}              & \raisebox{1.5ex}{$!a$}         &   und umgekehrt \\
\hline
\end{tabular}
\end{table}
