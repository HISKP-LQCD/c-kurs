\section{Felder (arrays)}

Bisher haben wir uns ausschließlich mit skalaren Datentypen beschäftigt, d.h. also einzelne Elemente eines bestimmten Typs.
\texttt{C} kennt allerdings auch Tupel eines bestimmten Typs, sogenannte Felder oder im Englischen \emph{arrays}.
Allerdings muss die Länge des Tupels zum Zeitpunkt der Deklaration bekannt sein.
Und diese Länge kann auch dann nicht mehr verändert werden.

\subsection{Deklaration und Initialisierung eindimensionaler Felder}

Die Deklaration eines Feldes ist gany analog zu anderen Variablen
\begin{lstlisting}
  int a[5]; // array declaration
\end{lstlisting}
was in diesem Fall ein Feld der Länge $5$ mit ganzen Zahlen erzeugt.

Für die Initialisierung gibt es zwei Möglichkeiten.
Man kann ersten Deklaration und Initialisierung in einem Schritt vornehmen, was wie folgt aussieht
\begin{lstlisting}
  int a1[5] = {1, 2, 3, 4, 5};
  int a2[] = {1, 2, 3, 4, 5};
\end{lstlisting}
was jeweils ein Feld der Länge $5$ erzeugt, aber diesmal initialisiert mit den Werten $1,2,3,4,5$.
Bei der zweiten Zeile bestimmt der Compiler aus der Initialisierungsliste automatisch die Länge des Feldes.
Die zweite Möglichkeit besteht darin, jedes Element einzeln zuzuweisen:
\begin{lstlisting}
  int a[5];
  for(int i = 0; i < 5; i++) {
    a[i] = i+1;
  }
\end{lstlisting}
was zum exakt gleichen Ergebnis führt.
An diesem Beispiel sieht man bereits, dass in \texttt{C} die Indizierung von $0$ bis $n-1$ läuft.

Es sei nocheinmal darauf hingewiesen, dass Felder wie wir sie bisher eingeführt haben, konstante Länge haben.
So führt folgender Quelltext zu einer Fehlermeldung
\begin{lstlisting}
  int n = 5;
  int a[n];
\end{lstlisting}
da die Variable \texttt{n} nicht konstant ist.
Dagegen ist folgendes korrekt
\begin{lstlisting}
  const int n = 5;
  int a[n];
  for(int i = 0; i < n; i++) {
    a[i] = i+1;
  }
\end{lstlisting}
was es erlaubt die Feldlänge in einer Variablen zu speichern.
Wenn sich nun im Quelltext diese Länge ändert, so muss man diese Änderung nur noch an einer Stelle vornehmen.

\subsection{Eindimensionale Felder als Funktionenargumente}

Wie schon im Abschnitt über Funktionen angedeutet, kann man auch Felder als Funktionenargument verwenden.
Die Schreibweise wird am folgenden Beispiel für eine Funktion erklärt, die die Quadratsumme aller Elemente eines Feldes erzeugt (das Quadrat der Norm)
\begin{lstlisting}
  double sqrsum(double a[], const int n) {
    double sum = 0;
    for(int i = 0; i < n; i++) {
      sum += a[i]*a[i];
    }
    return sum;
  }
\end{lstlisting}
Die Schreibweise \texttt{a[]} teilt dem Compiler mit, dass es sich um ein Feld handelt.
Es ist sehr wichtig hier zu verstehen, dass der Compiler aber keine Möglichkeit hat zu überprüfen, ob innerhalb der Funktion nur auf die vorhandenen Elemente von \texttt{a} zugegriffen wird.
Diese Information ist sogar nicht verfügbar, wenn man nicht als Programmierer selbst sicherstellt, dass in diesem Fall \texttt{n} die richtige Länge als Wert enthält.
Folgender Quelltext ist korrekter \texttt{C} Quelltext
\begin{lstlisting}
int list[5];
int i = 5;
list[i] = 3;
\end{lstlisting}
und wird vom Compiler anstandslos übersetzt.
Im besten Fall erhält man dann bei der Ausführung dieses Codes einen \emph{segmentation fault}.
Im schlechtesten Fall ist \verb|list[5]| Speicher, auf den das Programm zugreifen kann.
Dann erhält man keinen Laufzeitfehler und modifiziert ungewollt Speicher, den man nicht modifizieren will.
Dies kann zu sehr seltsamem Verhalten des Programms führen, und das Finden eines solchen Fehlers ist sehr schwierig.
Deshalb sollte man Indizierungen immer mit großer Sorgfalt überprüfen.

Es gibt noch einen weiteren wichtigen Punkt, der später noch genauer diskutiert wird.
Für Felder, die man wie oben beschrieben an Funktionen übergibt, wird keine Kopie angelegt.
Das heißt insbesondere, dass sich das Originalfeld ändert, wenn man innerhalb der Funktion das Feld modifiziert.
Genauer wird das im Abschnitt über Zeiger diskutiert.

\subsection{Mehrdimensionale Felder}

Ganz analog zu eindimensionalen Feldern lassen sich auch zweidimensionale Felder (Matrizen) erzeugen
\begin{lstlisting}
  int A[3][5]; // Deklaration
  for(int i = 0; i < 3; i++) {
    for(int j = 0; j < 5; j++) {
      A[i][j] = i+j;  // Initialisierung
    }
  }
\end{lstlisting}
Und wie im eindimensionalen Fall kann man auch Deklaration und Initialisierung kombinieren
\begin{lstlisting}
  int A1[2][3] = {{1,2,3}, {4,5,6}, {7,8,9}};
  int A2[][3]  = {{1,2,3}, {4,5,6}, {7,8,9}};
  int A3[][3] = {1,2,3,4,5,6,7,8,9}};
\end{lstlisting}
Die Anzahl der Zeilen kann der Compiler aus der Initialisierungsliste bestimmen.
Die Anzahl der Spalten muss aber immer angegeben sein.
Die Dimensionalität eines Feldes kann dann in analoger Weise zu drei oder mehr erweitert werden.

\subsection{Zeichenketten oder Strings}

In C gibt es keinen elementaren Datentyp für Zeichenketten, sogenannte \emph{strings}.
Zeichenkette werden mit Hilfe von Arrays abgebildet.
Ein Zeichen kann in einer Variablen vom Typ \verb|char| gespeichert werden, siehe ASCII Zeichensatz.\footnote{%
    Die Größe des Datentyps \texttt{char} ist immer ein Byte, also 8 Bit.
    Eigentlich hätte man diesen Typ \texttt{byte} nennen sollen. Es können $2^8
    = 256$ verschiedene Zeichen gespeichert werden. Dies reicht für Englisch
    und ein paar Steuerzeichen, jedoch nicht für alle Sprachen dieser Welt. Der
    ASCII Zeichensatz nutzt die ersten 128 Zustände (also die ersten 7 Bit) für
    das im Englischen genutzte Alphabet. Die restlichen 128 Zustände werden
    abhängig vom  \emph{Encoding} interpretiert. Für Deutsch kann man
    \texttt{latin-1} nutzen. Je nach Encoding wird ein Zeichen mit Wert 204
    interpretiert. Ist das Encoding nicht die richtige, erscheinen die Umlaute
    nicht korrekt und scheinbar willkürliche Zeichen stehen an ihrer Stelle.
    Sprachen, die mehr als 128 verschiedene Zeichen brauche, können im oberen
    Teil von ASCII überhaupt nicht dargestellt werden. Die Einsicht, dass es
    mehr als 256 verschiedene Zeichen gibt, wurde im Unicode Standard
    manifestiert. Die Konsequenz ist jedoch, dass jetzt mehr als ein Byte pro
    Zeichen benötigt wird. UTF-8 ist inzwischen das Standard-Encoding, sodass
    beliebig viele verschiedene Zeichen in einer Textdatei genutzt werden
    können. Der Preis ist jedoch, dass ein Buchstabe jetzt beliebig viele Byte
    (meist eins) nutzt. Da hier der Fokus allerdings auf Numerischen Methoden
    liegt, wird nicht weiter auf die vielfältigen Probleme mit Encodings
    eingegangen.
}
Eine Zeichenkette kann also durch eine Array von Elementen vom Typ \verb|char| erzeugt werden.
Das Ende einer Zeichenkette wird durch das Zeichen \verb|\0| angegeben.
Im nächsten Beispiel lesen wir eine Zeichenkette von der Standardeingabe ein, bis wir das Zeichen für das Ende des Strings finden und testen, ob die Kette eine Zahl enthält:
\begin{lstlisting}
#include <stdio.h>

int main()
{
  char string[] = "sdfk99225kljsdfs\0";
  int i = 0;
  int ergebnis = 0;

  while(1) {
    if(string[i] == '\0') {
      break;
    }
    if((string[i]) >= '0' && (string[i] <= '9')) {
      ergebnis = 1;
      break;
    }
    i++;
  }
  if (ergebnis)
    printf("Der String %s enthaelt mindestens eine Zahl\n", string);
  else
    printf("Der String %s enthaelt keine Zahlen\n", string);
  return 0;
}
\end{lstlisting}
Wir deklarieren und initialisieren zunächst die Variable \texttt{string} mit einer Zeichenkette.
Dann nutzen wir eine \verb|while| Schleife mit konstant wahrem logischem Ausdruck.
Die Schleife wird dann mit \verb|break| abgebrochen, wenn das Endstring-Zeichen gefunden wurde.
In der Schleife wird dann jedes Element der Kette darauf überprüft, ob es eine Zahl ist.
Dementsprechend wird die Variable \verb|ergebnis| gesetzt.
Beim abschließenden \texttt{printf} wird eine wichtige Eigenschaft von \texttt{C} Feldern deutlich:
Man benötigt keine Referenzierungsoperator \verb|&|.
Im nächsten Abschnitt über Zeiger wird klar werden, warum dies so ist.

\subsubsection{\texttt{sprintf} und \texttt{snprintf}}

Die Manipulation von Zeichenketten ist oft ein wichtiger Bestandteil eines Programms, zum Beispiel um Ausgabedateinamen abh\"{a}ngig vom Wert einer Variable zu machen.

Hierf\"{u}r stehen in C die beiden Funktionen \texttt{sprintf} aund \texttt{snprintf} zur Verf\"{u}gung, welche wie die schon bekannte \texttt{printf} Funktion funktioniern.
Im Gegensatz zu \texttt{printf}, schreiben diese Funktionen jedoch direkt in eine sich im Speicher befindliche Zeichenkette.

\begin{lstlisting}
#include <stdio.h>
const int MAX_LENGTH = 1000;
const int SHORT_LENGTH = 50;

int main()
{
  char string[MAX_LENGTH];
  const int i = 42;
  sprintf(string, "The Answer to the Ultimate Question of Life, The Universe, and Everything is %d.\n\n", i);
  printf("%s", string);
  
  int rval;
  char short_string[SHORT_LENGTH];
  
  rval = snprintf(short_string, SHORT_LENGTH-1, "This is a test string.\n\n");
  if(rval >= SHORT_LENGTH) 
    printf("snprintf: The string was not completely written!\n");
  else 
    printf("snprintf: wrote %d characters\n", rval);
  printf("%s",short_string);
  
  /* ueberlange Zeichenkette wird in short_string geschrieben, Rueckgaberwert wird ueberprueft */
  rval = snprintf(short_string, SHORT_LENGTH-1, "The Answer to the Ultimate Question of Life, The Universe, and Everything is %d.\n", i);
  if(rval >= SHORT_LENGTH) 
    printf("snprintf: The string was not completely written!\n");
  
  printf("%s\n\n",short_string);
  
  // ACHTUNG: hier wird fremder Speicher ueberschrieben 
  sprintf(short_string, "The Answer to the Ultimate Question of Life, The Universe, and Everything is %d.\n", i);
  printf("%s",short_string); 
  
  return 0;
}
\end{lstlisting}

Ebenso wie bei \texttt{scanf}, besteht jedoch die Gefahr eines buffer overflows, wenn mit \texttt{sprintf} mehr Zeichen geschrieben werden, als eigentlich allokiert wurden.
Dies resultiert im schlimmsten Fall \emph{nicht} in einem Programmabsturz sondern darin, dass irgendeine Speicherstelle überschrieben wird.

Es ist deshalb Vorsicht geboten und \texttt{snprintf} sollte \texttt{sprintf} vorgezogen werden, da ersteres es erlaubt, die L\"{a}nge der geschriebenen Zeichenkette zu beschr\"{a}nken.
Desweiteren kann man durch \"{U}berpr\"{u}fung des R\"{u}ckgabewertes der Funktion feststellen, ob die Zeichenkette in den vorhandenen Speicher gepasst hat und so direkt auf den Fehler reagieren.

\noindent Ausgabe:
\begin{verbatim}
$ ./test
The Answer to the Ultimate Question of Life, The Universe, and Everything is 42.

snprintf: wrote 24 characters
This is a test string.

snprintf: The string was not completely written!
The Answer to the Ultimate Question of Life, The

The Answer to the Ultimate Question of Life, The Universe, and Everything is 42.
\end{verbatim}

Die Benutzung von Arrays wird in folgendem Beispiel illustriert:
\begin{myexampleprogram}{Beispiel: Berechnung des Mittelwerts und der Varianz}
  Die Berechnung des Mittelwerts einer Datenreihe ist ein gutes Beispiel für die Benutzung von Arrays.
  Nehmen wir an, wir haben die folgend Daten gegeben:
\begin{lstlisting}
double data[] = {1.3, 2.4, 5.3, 2.4, 6.7, 3.5, 6.9, 1.3, 1.4, 4.5,
                 5.5, 5.3, 6.7, 2.1, 2.4, 3.3, 7.9, 0.3, 3.3, 1.5};
\end{lstlisting}
  und wir wollen den Mittelwert dieser Daten berechnen.
  Folgendes Program übernimmt diese Aufgabe:
\begin{lstlisting}
#include <stdio.h>

int main()
{
  const int size = 20;
  double data[] = {1.3, 2.4, 5.3, 2.4, 6.7, 3.5, 6.9, 1.3, 1.4, 4.5,
                   5.5, 5.3, 6.7, 2.1, 2.4, 3.3, 7.9, 0.3, 3.3, 1.5};

  // initialisiere mean zu 0
  double mean = 0.;
  for (int i = 0; i < size; i++)
    {
      mean += data[i];
    }
  mean /= (double)size;
  printf("Der Mittelwert ist %e\n", mean);
  return (0);
}
\end{lstlisting}
  Für die Varianz müssen wir auch noch die Quadrate aufsummieren.
  Wir modifizieren dafür die Schleife wie folgt:
\begin{lstlisting}
double mean = 0., xsq = 0.;
for (int i = 0; i < size; i++)
  {
    mean += data[i];
    xsq += data[i] * data[i];
  }
mean /= (double)size;
\end{lstlisting}
  Die Varianz können wir dann wie folgt berechnen und ausgeben:
\begin{lstlisting}
double var = xsq / (double)size - mean * mean;
printf("Die Varianz ist %e\n", var);
\end{lstlisting}
\end{myexampleprogram}


\endinput
