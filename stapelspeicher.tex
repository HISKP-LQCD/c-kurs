\section{Verketette Liste, der Kellerspeicher}
Wir haben gelernt, wie wir die Memoryverwaltung verwenden können. Aber was passiert, wenn wir ein Element löschen wollten, oder
wir nochein zwischen zwei schon existiert Elementen hinzufügen möchten. In anderen Worten, wir möchten ein Dateistrukture, das
dynamisch erhöht, oder schrumpft. Das können wir ausführen mit einer Strukture, die die Adresse des nächsten Elements enthält.
Beispielweise in der C Sprache ein Struktureelement kann auch ein Pointer auf den nächsten Element sein.
\begin{lstlisting}
struct list{
  int element;
  struct list *nachste;
}
\end{lstlisting}
\begin{figure}[!ht]
% Generated with LaTeXDraw 2.0.8
% Sun Mar 05 19:02:52 CET 2017
% \usepackage[usenames,dvipsnames]{pstricks}
% \usepackage{epsfig}
% \usepackage{pst-grad} % For gradients
% \usepackage{pst-plot} % For axes
\scalebox{1} % Change this value to rescale the drawing.
{
\begin{pspicture}(0,0.)(16.0,4.)
\psframe[linewidth=0.04,dimen=outer](3.,1)(0.0,3.0)
\rput(1.2, 2.5){element: 4}
\rput(1.2, 2.0){nachste:} 
\rput(1.2, 1.5){0x1b464c0}
\rput(1.5, 0.5){Adresse: 0x1b464a0} 

\psline[linewidth=0.04cm](3.5,2.)(4.5,2.)
\psline[linewidth=0.04cm](4.25,2.5)(4.5,2)
\psline[linewidth=0.04cm](4.25,1.5)(4.5,2)
\psframe[linewidth=0.04,dimen=outer](8.,1)(5.0,3.0)
\rput(6.2, 2.5){element: 6}
\rput(6.2, 2.0){nachste:}
\rput(6.2, 1.5){0x1b46480}
\rput(6.2, 0.5){0x1b464c0}
\psline[linewidth=0.04cm](8.5,2.)(9.5,2.)
\psline[linewidth=0.04cm](9.25,2.5)(9.5,2)
\psline[linewidth=0.04cm](9.25,1.5)(9.5,2)

\psframe[linewidth=0.04,dimen=outer](13.,1)(10.0,3.0)
\rput(11.2, 2.5){element: 6}
\rput(11.2, 2.0){nachste:}
\rput(11.2, 1.5){NULL}
\rput(11.2, 0.5){0x1b46480}
\psline[linewidth=0.04cm](13.,2.)(14.,2.)
\psline[linewidth=0.04cm](13.75,2.5)(14.,2)
\psline[linewidth=0.04cm](13.75,1.5)(14.,2)
\rput(15, 2.){\LARGE NULL}
\end{pspicture} 
}
\caption{Einmal Verketette Liste\label{verklist}}
\end{figure}
In der Abbildung \ref{verklist} wir zeigen, wie es in der Praxis
funkzioniert. Wir werden den NULL pointer für das Ende der verketetten
Liste verwenden. Viele wichtigen Datenstukturen können mit 
verketetten Listen darstellen. Beispielweise den Stapelspeicher.
In Assembly Programmiersprache beim Funktionanrufen die Parameters werden 
zum Stapelspeicher gestellt. Wenn wir kommen von einem Funktion zurück, 
wir nehmen ein Element vom Stapelspeicher aus. Das Grundprinzip im
Stapelspeicher ist: Den zuletzt gespeicherten Element werden wir zuerst
zurückgeben. Wir zeigen es in der Abbildung \ref{stapspeicher}.
\begin{figure}[!ht]
\begin{center}
% Generated with LaTeXDraw 2.0.8
% Sun Mar 05 20:17:29 CET 2017
% \usepackage[usenames,dvipsnames]{pstricks}
% \usepackage{epsfig}
% \usepackage{pst-grad} % For gradients
% \usepackage{pst-plot} % For axes
\scalebox{0.7} % Change this value to rescale the drawing.
{
\begin{pspicture}(0,-3)(12.0,3)
\psframe[linewidth=0.04,dimen=outer](3,3)(0.0,2)
\psframe[linewidth=0.04,dimen=outer](11.0,3)(8,2)
\pscustom[linewidth=0.04]
{
\newpath
\moveto(3.1,2.5)
%\lineto(7.5,1.5)
\curveto(3.3,2.5)(3.5,2.4)(4.5,1.7)
}
\rput(3.7, 2.8){\LARGE push}
\psline[linewidth=0.04cm](4.55,1.9)(4.5,1.7)
\psline[linewidth=0.04cm](4.4,1.6)(4.5,1.7)


\pscustom[linewidth=0.04]
{
\newpath
\moveto(7.5,2.5)
%\lineto(7.5,1.5)
\curveto(7.5,2.5)(6.8,2.5)(6.5,1.6)
}
\rput(6.7, 2.8){\LARGE pop}
\psline[linewidth=0.04cm](7.4,2.7)(7.5,2.5)
\psline[linewidth=0.04cm](7.4,2.3)(7.5,2.5)


\psframe[linewidth=0.04,dimen=outer](7,1.5)(4,0.5)
\psframe[linewidth=0.04,dimen=outer](7,0.2)(4,-0.8)
\psframe[linewidth=0.04,dimen=outer](7,-1.1)(4,-2.1)
\end{pspicture} 
}
\caption{Der Stapelspeicher\label{stapspeicher}}
\end{center}
\end{figure}

Wir werden jetzt implementieren diesen wichtigen Datenstruktur mit 
den verketetten Listen. Zuerst stellen wir der Datenstruktur, den wir verwenden
vor:
\begin{lstlisting}
typedef struct list {
   struct list* next;
   int data;
} list;
\end{lstlisting}
Das ist natürlich eine verketette Liste. Aber wir haben hier diesen Typ also definiert
mit dem Strichwort $typedef$.
\begin{myexampleblock}{Strichwort: \texttt{typedef}}
\begin{lstlisting}
typedef neuename altename;
\end{lstlisting}
\vspace{-0.4cm}
Stellt ein neues Name ($neuename$) für den Typ von $altename$ her.
\end{myexampleblock}
Der größte Vorteil in diesem Fall ist, dass wir den Strichwort $struct$ 
vor allen Variable definition verwenden müssen. Der Quelltext wird durchsichtiger
sein. Jetzt wir haben das Definition der Datenstrukture, wir können die Implementierung
anfangen. Zuerst besprechen wir, wie können wir eine neue Stapelelement herstellen.
Wie können wir eine neue Variable von Typ $list$ herstellen.
\begin{lstlisting}
list * get_a_list( int x ){
   list *q=(list *)malloc(sizeof(list));
   if (q == NULL){
     fprintf(stderr, "Error in Memory allocation\n");
     exit(1);
   }
   q->data=x;
   q->next=NULL;
   return q;
}
\end{lstlisting}
Der obene Codeteil gib eine $list$ Element zurück und hat die
speichernde ganze Zahl als Eingabeparameter. Dieser Teil ist ganz einfach.
Wir reservieren Speicherzellen für den neuen Listelement und zuweisen
den Eingabewert. Zum Ende wir müssen die nächste Element als ein NULL 
Pointer festlegen. Die nächste Teil ist das push() Funktion. Speichern 
ein Element im Stapelspeicher.
\begin{lstlisting}
void push( int x, list ** first){
   if ((*first) == NULL){
       (*first)=get_a_list(x);
       return;
   }
   list *temp=get_a_list(x);
   temp->next=(*first);
   (*first)=temp;
}
\end{lstlisting}
Wir müssen die Adresse von dem Pointer auf den Anfang des Speichers  und den speichernde 
Element abgeben. Wenn den List leer ist, wir legen die Anfangsadresse mit der adresse 
des  neuen Elements fest. In anderem Fall wir legen den next Pointer von dem neuen Element 
mit dem originalle Anfangsadresse und die neue Anfangsadresse wird die Adresse vom neuen Element 
sein. Darum müssen wir nicht die Anfangsadresse als Parameter abgeben, sondern ihre Adresse,
weil wir den Wert von ihm im Funktion ändern wollen. Den logischen nächsten Schritt
ist die implementierung des $pop$ Funktions. Der Rückgabewert muss den letzten gespeicherte
Element sein. Der Eingabeparameter muss den Anfangsadresse den Stapelspeicher sein.
\begin{lstlisting}
int pop( list **first){
   list *temp;
   int ret;
   if ((*first) == NULL){
     printf("Es gibt kein Element im Speicher\n");
     return 0;
   }
   ret=(*first)->data;
   temp=(*first)->next;
   free(*first);
   (*first)=temp;
   return ret;
}
\end{lstlisting}
Wenn das Stapelspeicher leer war, wir werden eine Nachricht ausdrücken. 
Im anderen Fall wir legen den Rückgabewert mit dem Wert von dem ersten 
Element fest. Wir müssen auch die ersten Element freigeben, und die neue
Anfangsadresse mit dem nächste der originalen festlegen.

Wir können auch prüfen die aktuellen Inhalt des Speichers. 
Dafür müssen wir einer Funktion, der dursch die Listelementen geht
schreiben.
\begin{lstlisting}
void print_list( list *first){
   list *temp=(first);
   if (temp == NULL)
      return;
   for (;;){
      printf("%d\t", temp->data);
      temp=temp->next;
      if (temp == NULL)
        break;
   }
   printf("\n");
}
\end{lstlisting}
Dieser Funktion zuerst prüft ob der Speicher leer ist. Wenn der
Speicher nicht leer ist, wir machen eine Schleife um durch die
Listelementen gehen. In jeden iteration wir springer zur 
nächsten Element (Zeile 7). Wenn wir am Ende des Speichers
sind, beenden wir die Schleife.

Es fehlt jetzt nur noch ein Gebrauchanweisung für unseren Stapelspeichern.
Dies wird der $main$ Funktion.
\begin{lstlisting}
int main(){
   list *erste=NULL;
   push(2, &erste);  print_list(erste);
   push(4, &erste);  print_list(erste);
   push(6, &erste);  print_list(erste);
   pop( &erste );    print_list(erste);
   push(8, &erste);  print_list(erste);
   pop( &erste );    print_list(erste);
}
\end{lstlisting}
Wir definieren zuerst einer Pointer um die 
Anfangsadresse des Speichers zu speichern (Zeile 2). Dann wir machen eine
Folgen aus push und pop Anweisungen, und im jeden Schritt wir drücken
den aktuellen Zustand des Speichers aus. Die Ausgebe siehst du unter.
\begin{lstlisting}
2	
4	2	
6	4	2	
4	2	
8	4	2	
4	2	
\end{lstlisting}
Wie du siehst, wenn wir die erste $pop$ Anweisung ausführen, wurde die letzte 
eingegebene Wert (6) aus dem Speicher gelöst. 

