\section{Funktionen}

Es gibt viele Quelltextabschnitte, die wiederholt benutzt werden.
Dafür existiert das Konzept von Funktionen.
Eine Funktion haben wir schon kennen gelernt, nämlich die Funktion \verb|main|. 
In C sind Funktionen Variablen sehr ähnlich.
Eine Funktion kann wie folgt deklariert werden
\begin{lstlisting}
  Rueckgabetyp Funktionsname ( Parameterliste );
\end{lstlisting}
Man spricht von einem sogenannten Funktionsprototypen.
Die Parameterliste besteht aus durch Kommata getrennte Variablendeklarationen
\begin{lstlisting}
  Typ1 name1, Typ2 name2, ...
\end{lstlisting}
Der Rückgabetyp kann jeder C Typ (und jeder selbst definierte Typ sein).

Die Definition einer Funktion muss dann natürlich einen Block von Anweisungen enthalten, also
\begin{lstlisting}
  Rueckgabetyp Funktionsname ( Parameterliste ) {
    Rueckgabetyp x;
    Anweisung1;
    Anweisung2;
    ...;
    return(x);
  }
\end{lstlisting}
Die in der Parameterliste deklarierten Variablen sind dann innerhalb dieses Blocks definiert und unter ihrem Namen sichtbar.
Außerdem sind alle \emph{global} deklarierten Variablen im Funktionsblock sichtbar.
Für den Funktionsnamen gelten die gleiche Regeln, wie für Variablennamen.

Als Beispiel für eine Funktion schreiben wir einen Abschnitt, der die Fakultät einer ganzen Zahl berechnet:
\begin{lstlisting}
  unsigned long int Fakultaet(const unsigned int zahl) {
    unsigned long int fak = 1;
    for(int i = 2; i <= zahl; i++) {
      fak *= i;
    }
    return(fak);
  }
\end{lstlisting}
Der Rückgabetyp ist als \verb|unsigned long int| gewählt, da die Fakultaet immer positiv ist, aber auch sehr groß werden kann.
\endinput
