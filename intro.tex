\section{Einführung}
In diesem Kurz wir werden die C Programmier sprache kennenlernen. Programmierung besteht aus Code schreiben und die übersetztung dieser Code in
ausführbaren Maschinecode.
Es gibt Programmier Sprache, die jede Befehle nacheinander übersetzt und ausführt. C ist anders, hier die ganze Code muss zuerst schreiben und
kompiliert werden,  danach kann diene Programm ausführen. Diese Sprachen gehören zu Imperative Sprachen. Wir nutzten Compiler unser Code zu Maschinencode zu übersetzten. Für ersten Programmier Sprache C ist ideal
von mehreren Gründen:
\begin{itemize}
\item C ist sehr effizient
\item C hat high level konstrukte.
\end{itemize}
Es ist effizient, weil es sehr gute Compiler gibt. Wir müssen nicht die Teile des CPU-s wissen um eine gute Code zu schreiben,
weil es high level konstrukte hat. Aber M\"oglicherweise das beste Antwort ist das C Kenntnisse ist unbedingt in Forschungsrechnungen. Die meisten Programme,
der zum Stand der Technik geh\"oren sind C programme.


Hier wir beschreiben kurz den Inhalt des Buches. In dem ersten Teil des Buches wir erklären, wie man eine einfache Aufgabe mithilfe der C
sprache erledigen kann. Wir werden die Grundelementen der Sprache durch einem Beispiel (Einfügesortieren) vorstellen. Zuerst wir fassen
zusammen was in der Sprache inbegriffen ist. Du wirst dich verwundern, das die einfachste funktion, was nur etwas auf deinem Monitor zeigt,
geh\"ort nicht zu der Sprache. Wir fangen mit dem k\"orper eines durchschnittlichen C programm an  und erkl\"aren sein Teilen.
Um ein C programm zu verstehen, wir müssen Zwei wichtige Konzept wissen:
\begin{enumerate}
\item Data Typen
\item Funkcionen
\end{enumerate}

Zum Bespiel wenn wir die Zahlen sortieren wollen, wir müssen entscheiden ob wir ganze oder reelle Zahlen nutzten wollen. Zu jeden Data
Typen gehören Operationen, in denen wir die nutzen k\"onnen. Wir werden alle elementare Data Type kennenlernen. Mit Variablen aus diesen
Typen dann k\"onnen wir Operationen machen, die unsere Ergebnisse herstellen werden. Aber es kann sein, das unserer aktuelle Befehl hängt
von dem Ergebnis des vorherigen Befehl ab. In diesem Fall die Sprache bietet uns Statements und Expressions, mit denen wir kleinen Aufgaben
lösen könnnen.

Die andere wichtige Konzept is die Funkcionen. Die Funkcionen arbeiten wie schwarze K\"asten aus der Sicht des Benutzers. Sie stellen von
den Eingangsparameter ein neues Wert her. Auch wir können Parameters zu unserem Programm nur durch Funkcionen geben.  Zu genießen
die Fähigkeiten der Sprache, wir müssen zuerst verstehen, wie mann Funkcionen angerufen kann. Wir zeigen,
wie man die Standart Eingabe und Ausgabe Bibliothek nutzten kann. Nachdem wir diese F\"ahigkeiten verstehen, werden wir etwas
komplizierte \"Ubungen schreiben.

In dem zweiten Teil des Kurzes wir werden unsere Coden verbessern. Dafür gibt es zwei vershiedene
Richtungen:
\begin{enumerate}
\item Memorie verwaltung
\item Zusammengesetzte Datenstruktur
\end{enumerate}

Verstehen wie mann Memorie zuweisen, oder freien kann sehr hilfbereit sein, um eine reine Code schreiben zu können.
Die Höchstwahrscheinlich auftretene Fehler der Anfängers ist Segmentation Fault. Du kannst diese Fehler vermeiden nur, wenn
du weisst wie addressierung funkcionert. Das wird eine der wichtigesten Teil der Buch. Auf nebenplatz, wir werden
vorstellen, wie man strings in C verwenden kann.

Die Andere Richtung führt uns zur Zusammesgesetzte Datenstrukturs. Zuerst werden wir lernen, wie mann eigenes Daten strukturen herstellen
kann. Neue Datenstrukturen entdecken ist wichtig, weil sie können unsere Programme schneller machen. Zum Beispiel wenn mann nach
einem Field in einer List sucht, kann mann binarische Bäumen herstellen, um die Suchen schneller zu lassen. Um unseren Code
nach ein Jahr später auch verstehen zu können, es ist wichtig das Code verteilen zu können. Wir werden lernen wie man
von verschiedenen Files das Programm Aufbauen kann.

