\section{Das math Bibliothek}
Zur Standard \texttt{ANSI C} Sprache gehören auch mathematische Funktions and Konstanten.
Ihre deklaration befindet sich im \texttt{math.h}. Beispielweise wir möchten 
die Kosinunfunktion (Im Folgenden nur \texttt{cos}) eines Winkels berechnen.
Die \texttt{cos} Funktion wird auf ganz $\mathbb{R}$ durch ihre 
Taylorreihe mit Entwicklungsstelle 0 dargestellt:
\begin{equation}
cos\left(x\right)=\sum_{l=0}^{\infty} \left(-1\right)^{l} \dfrac{x^{2l}}{\left(2l\right)!}
\end{equation}
Natürlich im Computer wir haben eine endliche Genauigkeit mit dem wir Zahlen können
(Sieh die Einleitung). Den Funktion 
\begin{lstlisting}{cosineexample.c}
double cos(double x);
\end{lstlisting}
wird den \texttt{cos} in \texttt{double} Genauigkeit berechnen.
In dem unteren kleinen Quelltext wir zeigen es in der Praxis:
\begin{lstlisting}
#include<stdio.h>
#include<stdlib.h>
#include<math.h>
int main(int argc, char *argv[]){
  double winkle;
  double coswinkle;
  int n;
  //Wenn nicht genug Parameter gibt, druecken wir einen Nachricht aus
  if (argc < 2 ){
     printf("Vewendung: ./print_cos angle [n]; n ist freiwillig\n");
     exit(1);
  }
  winkle=atof(argv[1]);//Macht ein Fliesskommazahl aus einer Zeichenkette
  coswinkle=cos(winkle);//Wir verwenden den StandardBibliothek Funktion
  printf("Winkle=%e \t cos(Winkle)=%e\n", winkle, coswinkle);
  //Wir koennen mit dem kennengelernte Kontrollstrukturen die ernaherung
  //selbst erlidegen
  if (argc==3){
    int n;
    double coswinkleapprox;
    double winkletmp=winkle;//wir sammeln hier die Zaehler
    int denumtmp=1;//wir sammeln hier das Nenner
    n=atoi(argv[2]);//Macht ein ganze Zahl aus einer Zeichenkette
    coswinkleapprox=1.;//Wir sammeln hier das Ergebnis
    for (int i=1; i<n; ++i){
      int sign=( ( i % 2) == 0) ? 1 : -1; //wenn es gerade ist denn
                                          //das vorzeichen ist 1
                                          //andersfalls -1
      winkletmp=winkletmp*winkletmp;
      denumtmp=denumtmp*(2*i-1)*2*i;
      coswinkleapprox += sign*winkletmp/denumtmp;
    }
    printf("coswinkleapprox=%e\n", coswinkleapprox);
  }
}
\end{lstlisting}
In diesem Code wir verwenden den Standardfunktion und (wenn es nötig ist) erledigen
die ernäherung des \texttt{cos} Funktions selbst mit summieren ihren Taylor Reihe
bis \texttt{n}
\begin{equation}
cos_{approx}\left(x,n\right)=\sum_{l=0}^{n} \left(-1\right)^{l} \dfrac{x^{2l}}{\left(2l\right)!}
\end{equation}
Wir haben ein Eingabeparameter (der Winkel) und eine freiwillige Eingabeparameter
was die grad der Ernäherung bestimmt. Wir müssen die Winkel als ein Parameter eingeben.
Obwohl das Parameter eine Zeichenkette ist, können wir es in eine, Fließkommazahl konvertieren
mit Hilfe von \texttt{double atof(char *zeichenkette)} Standardbibliothekfunktion. 
Wir verwenden den \texttt{cos} Standardbibliothekfunktion in der Zeile 14. Wenn
den Anwänder wollte, erledigen wir die Ernäherung im Zeilen von 19-33. Wir bekommen 
das Grad \texttt{n} als auch ein Eingebeparameter. In Berechnung der Ernäherung wir machen 
eine Schleife von 1 bis das aktuellen Grad. Wir müssen nicht in jedem Schritt die 
Potenz ($x^{2i}$) und den Fakultät ($(2l)!$) berechnen, sondern auch wir können 
ihnen in jedem Schritt in zwei Helfenvariablen (\texttt{winkletmp,denumtmp}) speichern.
Das ist auch ein generall Konzept im Hand des Programmiers: Besser etwas mehr speicher
wervenden um wenige Berechnungen zu erledigen. 
Unseres Programm kann nachdem Folgenden Kompilieren:\\
\begin{verb}
>$  gcc -std=c99 cosineexample.c -o cos.exe -lm
\end{verb}\\
Mann muss achten um das \texttt{-lm} nicht zu vermeiden.
Die Weitere erreichbare  Funktionen zählen wir auf im Tabelle \ref{math}. 
\begin{table}[t]
\caption{Elementare Datentypen\label{tabelle1}}  % title name of the table
\centering
  % centering table
\begin{tabular}{|l l|}
\hline
\texttt{Deklaration} & \texttt{Rückgabewert} \\
  % creating 10 columns
\hline
\texttt{double acos(double x);} & Arcuscosinus von \texttt{x}\\
\texttt{double asin(double x);} & Arcussinus von \texttt{x}\\
\texttt{double atan(double x);} & Arcustangenz von \texttt{x}\\
\texttt{double atan2(double y,double x);} & Arcustangenz von \texttt{x/y}\\
\texttt{double ceil(double x);} & Kleinste ganze Zahl, welche $\ge$ \texttt{x} ist \\
\texttt{double cos(double x);} & Cosinus von \texttt{x}\\
\texttt{double cosh(double x);} & Cosinus hyperbolicus von \texttt{x}\\
\texttt{double exp(double x);} & $e^{x}$ wobei $e=2.7182..$\\
\texttt{double fabs(double x);} & $\vert x\vert$\\
\texttt{double floor(double x);} & Größte ganze Zahl, welche $\le$ \texttt{x} ist \\
\texttt{double ldexp(double x, double y);} & $xe^{y}$ \\
\texttt{double log(double x);} & Natürlicher Logarithmus von \texttt{x}\\
\texttt{double log10(double x);} & Zehnerlogarithmus von \texttt{x}\\
\texttt{double pow(double x, double y);} & $x^{y}$\\
\texttt{double sin(double x);} & Sinus von \texttt{x}\\
\texttt{double sinh(double x);} & Sinus hyperbolicus von \texttt{x}\\
\texttt{double sqrt(double x);} & $\sqrt{x}$\\
\texttt{double tan(double x);} & Tangens von \texttt{x}\\
\texttt{double tanh(double x);} & Tangens hyperbolikus von \texttt{x}\\
\hline
\end{tabular}
\caption{Funktionen von \texttt{math.h}\label{math}}
\end{table}

