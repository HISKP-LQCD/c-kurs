\section{Die Bibliotheken \texttt{math.h} und \texttt{complex.h}}

Die Sprache \texttt{C} hat neben den Sprachbestandteilen auch eine umfangreiche Liste von Standardbibliotheken.
Sie sind im \texttt{ANSI C} Standard festgelegt.
Diejenige für die Ein- und Ausgabe haben wir schon kennengelernt.
Wir werden jetzt noch zwei weitere kurz einführen.

\subsection{Mathematische Bibliothek \texttt{math.h}}

Erstens die Bibliothek \texttt{math.h}.
Sie stellt mathematische Funktionen, wie beispielsweise trigonometrische Funktionen, aber auch eine Funktion für den Betrag zur Verfügung.
Natürlich berechnen diese Funktionen das Ergebnis nur zur gewünschten Genauigkeit.
Wir illustrieren dies am Bespiel vom Cosinus.
Die Cosinus Funktion ist auf ganz $\mathbb{R}$ durch eine Taylorreihe um 0 definiert.
\begin{equation}
  \label{eq:cos}
  \cos\left(x\right)\ =\ \sum_{l=0}^{\infty} \left(-1\right)^{l} \dfrac{x^{2l}}{\left(2l\right)!}\,.
\end{equation}
Die Funktion 
\begin{lstlisting}
  double cos(double x);
\end{lstlisting}
berechnet den Cosinus in \texttt{double} Genauigkeit.
Im folgenden Quelltext berechnen wir den Cosinus einmal mit Hilfe der Funktion aus \texttt{math.h} und einmal über die Reihenentwicklung, die wir zu einer in der Kommandozeile angegebenen Ordnung abbrechen:
\begin{lstlisting}[caption={Beispiel zur Verwendung des Cosinus}, belowcaptionskip=0.3em]
#include <math.h>
#include <stdio.h>
#include <stdlib.h>

int main(int argc, char *argv[])
{
  double winkel;
  double coswinkel;
  // genuegend Kommandozeilenparameter?
  if (argc < 2) {
    printf("Usage: ./print_cos Winkel Naeherungsordnung[optional]\n");
    return(-1);
  }
  winkel = atof(argv[1]);  // Konvertiert eine Zeichenkette in eine Fliesskommazahl
  coswinkel = cos(winkel); // cos aus math.h
  printf("Winkel=%e \t cos(Winkel) = %e\n", winkel, coswinkel);
  // nun selbst, wenn die Naeherungsordnung gegeben wurde
  if (argc > 2) {
    int n;
    double coswinkelapprox;
    double zaehler = winkel; // Zaehler
    int nenner = 1;          // Nenner
    n = atoi(argv[2]);       // atoi konvertiert eine Zeichenkette in eine ganze Zahl
    coswinkelapprox = 1.;    // Das Ergebnis zu fuehrender Ordnung
    for (int i = 1; i < n; ++i) {
      int sign = 1;          // (-1)^l ist entweder 1 oder -1, wenn l gerade
      if( (i % 2) == 1 ) sign = -1; // oder ungerade ist
      zaehler = zaehler * zaehler;
      nenner = nenner * (2 * i - 1) * 2 * i;
      coswinkelapprox += sign * zaehler / nenner;
    }
    printf("cos(Winkel) approx %e\n", coswinkelapprox);
  }
  return(0);
}
\end{lstlisting}
Wenn auf der Kommandozeile die Ordnung der Näherung als zweiter Kommandozeilenparameter angegeben wurde\footnote{die Bedeutung von \texttt{argc} und \texttt{argv} wird in \cref{subsec:CommandLineArguments} erläutert}, berechnen wir auch die Näherung selbst nach der Formel Gleichung~\ref{eq:cos}.
Wie schon vorher erklärt, sind die Kommandozeilenparameter in der \texttt{main} Funktion nur als Zeichenkette verfügbar.
Die Konvertierung der Ordnung in eine ganze Zahl kann man mit der standard Funktion \texttt{atoi} (ASCII to integer) durchführen.
Genauso verwenden wir \texttt{atof} (ASCII to float) um den eingegebenen Winkel in the Fließkommazahl umzuwandeln.
Man beachte, dass die standard Cosinus Funktion das Argument im Bogenmaß erwartet.

Bei der Berechnung der Näherung müssen wir nicht in jedem Schritt die Potenz $x^{2i}$ und Fakultät $(2i)!$ neu berechnen, sonder müssen lediglich die Variablen \texttt{zaehler, nenner} entsprechend auffrischen.
Dies ist ein generelles Konzept: Man verwendet mehr Speicher um Rechnung zu sparen.

\subsubsection{Übersetzen und Linken mit \texttt{math.h}}

Das Programm kann man wie folgt übersetzen

\vspace*{0.5cm}
\begin{verbatim}
>$  gcc -std=c99 -Wall -pedantic cosineexample.c -o cos.exe -lm
\end{verbatim}
\vspace*{0.5cm}

\noindent Man beachte, dass man nun explizit die Bibliothek \texttt{math.h} dazulinken muss.
Dies geschieht mit dem Argument \texttt{-lm}.
In Tabelle~\ref{math} listen wir einige wichtige Funktionen auf, die \texttt{math.h} bereitstellt.

\subsection{Komplexe Zahlen mit \texttt{complex.h}}

Als zweite Standardbibliothek weisen wir noch auf \texttt{complex.h} hin.\index{\texttt{complex.h}}
Sie stellt einen komplexen Datentyp zur Verfügung, der leider nicht Teil von C selbst ist.
Die Benutzung ist dann aber gleich zu \texttt{C} Datentypen, wie man an folgendem Beispiel sieht:\index{\texttt{I}}\index{\texttt{complex}}\index{\texttt{cexp}}\index{Datentypen!\texttt{complex}}\index{Datentypen!\texttt{double complex}}
\begin{lstlisting}[caption={Beispiel für komplexe Zahlen}, belowcaptionskip=0.3em]
  #include <stdlib.h>
  #include <stdio.h>
  #include <math.h>
  #include <complex.h>
  
  int main(int argc, char *argv[])
  {
    double complex z,v,w;

    // I ist die rein komplexe Zahl i
    z = 1 + I*3;
    v = 0.3*I;
    // komplexe Multiplikation
    w = z*v;
    // Zugriff auf Real- und Imaginaerteile
    printf(``w = %e + i*%e\n'', creal(w), cimag(w));
    // Anwendung der komplexen Exponetialfunktion
    z = cexp(w);
    printf(``w = %e + i*%e\n'', creal(z), cimag(z));
    return 0;
  }
\end{lstlisting}

\subsubsection{Mathematische Funktionen für komplexe Zahlen}

In \texttt{math.h} sind wiederum die Funktionen aus Tabelle~\ref{math} für den Typ \texttt{double complex} definiert.\index{\texttt{math.h}}
Meistens unterscheiden sie sich von den reellwertigen Funktionen nur duch ein vorgestelltes \texttt{'c'}, wie zum Beispiel bei \texttt{cexp}.\index{\texttt{cexp}}
Auf Real- und Imaginärteil kann mit \texttt{creal} und \texttt{cimag} zugegriffen werden.
\texttt{complex.h} stellt auch Typen \texttt{float complex} und \texttt{long double complex} bereit.
Die entsprechenden Exponentialfunktionen heißen dann \texttt{cexpf} und \texttt{cexpl}.
Das komplex konjugierte erhält man mit \texttt{conj}, bzw. \texttt{conjf} und \texttt{conjl}.
\index{\texttt{complex.h}}\index{\texttt{creal}}\index{\texttt{cimag}}\index{\texttt{conj}}

\begin{table}
  \centering
  % centering table
  \begin{tabular}{|l l|}
    \hline
    \texttt{Deklaration} & \texttt{Rückgabewert} \\
    % creating 10 columns
    \hline
    \texttt{double acos(double x);} & Arcuscosinus von \texttt{x}\\
    \texttt{double asin(double x);} & Arcussinus von \texttt{x}\\
    \texttt{double atan(double x);} & Arcustangenz von \texttt{x}\\
    \texttt{double atan2(double y,double x);} & Arcustangenz von \texttt{x/y}\\
    \texttt{double ceil(double x);} & Kleinste ganze Zahl, welche $\ge$ \texttt{x} ist \\
    \texttt{double cos(double x);} & Cosinus von \texttt{x}\\
    \texttt{double cosh(double x);} & Cosinus hyperbolicus von \texttt{x}\\
    \texttt{double exp(double x);} & $e^{x}$ wobei $e=2.7182..$\\
    \texttt{double fabs(double x);} & $\vert x\vert$\\
    \texttt{double floor(double x);} & Größte ganze Zahl, welche $\le$ \texttt{x} ist \\
    \texttt{double ldexp(double x, double y);} & $xe^{y}$ \\
    \texttt{double log(double x);} & Natürlicher Logarithmus von \texttt{x}\\
    \texttt{double log10(double x);} & Zehnerlogarithmus von \texttt{x}\\
    \texttt{double pow(double x, double y);} & $x^{y}$\\
    \texttt{double sin(double x);} & Sinus von \texttt{x}\\
    \texttt{double sinh(double x);} & Sinus hyperbolicus von \texttt{x}\\
    \texttt{double sqrt(double x);} & $\sqrt{x}$\\
    \texttt{double tan(double x);} & Tangens von \texttt{x}\\
    \texttt{double tanh(double x);} & Tangens hyperbolikus von \texttt{x}\\
    \hline
  \end{tabular}
  \caption{Einige Funktionen, die in \texttt{math.h} deklariert sind.\label{math}}
\end{table}
